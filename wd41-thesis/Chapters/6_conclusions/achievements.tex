%**************************************************************
\section{Main Achievements}\label{sec:conclusions_achievements}
%**************************************************************

% Publications
The thesis proposed the application of methods adapted from applied literature with the ultimate goal to quantify the uncertainty of model parameters in a \gls[hyper=false]{th} system code, 
The application of each of the proposed methods was illustrated and demonstrated on the basis of a reflood experiment simulation model in the \gls[hyper=false]{trace} code.
During the course of the present doctoral research, four papers were presented in international conferences \cite{Wicaksono2014,Wicaksono2014a,Wicaksono2015,Wicaksono2016}, a journal article was published \cite{Wicaksono2016b}, and two contributions were submitted \cite{Wicaksono2016a,Zerkak2016} to the \gls[hyper=false]{premium} project and included in the NEA reports \cite{Reventos2016,Sanz2017}.

% Contributions to PREMIUM
The works related to the contribution to the \gls[hyper=false]{premium} project comprises the bulk of Chapter~\ref{ch:trace_reflood}.
The \gls[hyper=false]{trace} model of \gls[hyper=false]{feba} was successfully developed within that context and became the basis for several follow-up studies.
\marginpar{TRACE model of FEBA}
The model was stable and was relatively quick to run allowing even a relatively brute force sensitivity analysis method to be applied.
It is now part of the in-house \gls[hyper=false]{trace} code validation database at LRS.
Still within the context of \gls[hyper=false]{premium} a python scripting tool was developed to assist in conducting computer experiment on the \gls[hyper=false]{trace} model of \gls[hyper=false]{feba}.
\marginpar{\texttt{trace-simexp}}
The tool \texttt{trace-simexp} has reached a stable version, is well documented, and has been applied for several follow-up studies within and without the scope of the present doctoral research.

The prior uncertainties of the input parameters were quantified with a close supervision of an expert at \glsentryshort{psi} \cite{Zerkak2016}.
\marginpar{Contribution to PREMIUM, prior uncertainty quantification}
The quantified uncertainties were then propagated both on the \gls[hyper=false]{trace} models of \gls[hyper=false]{feba} and PERICLES (another reflood facility not presented in this thesis).
The result of the propagation submitted to the \gls[hyper=false]{premium} project were deemed satisfactory as it served the purpose of the prior quantification.
That is, the prediction uncertainties of both facilities were wide but covered the experimental data well, confirming that the prior range was not underestimated.

% Sensitivity Analysis, Morris
Three conference papers \cite{Wicaksono2014,Wicaksono2014a,Wicaksono2015} and a journal article \cite{Wicaksono2016b} made up Chapter~\ref{ch:gsa}.
The size of the initial selection of input parameters, as exemplified in \gls[hyper=false]{premium}, can be large. 
\marginpar{Implementation and application of screening methods}
Lacking prior knowledge, the selection should also include all the parameters that are vaguely perceived as important. 
The implementation and the application of screening methods (Morris screening method and Sobol' total-effect indices), as demonstrated in this thesis for the \gls[hyper=false]{trace} model of \gls[hyper=false]{feba},
allows for a quick, systematic, and quantitative screening of the initial set of input parameters in a global manner (i.e. simultaneous perturbation over the whole range of parameter uncertainties).
In the case studied here, more than half of the initial selection were found to be non-influential to the reflood simulation.

In conjunction with that, \gls[hyper=false]{fda} techniques to characterize the variation of functional data set was investigated and successfully applied to analyze the variation in reflood curves.
In essence, the application of the \gls[hyper=false]{fda} techniques resulted in the representation of the time-dependent output in a reduced space.
Indeed, time- and space-dependent outputs are ubiquitous in the \gls[hyper=false]{th} analysis thus dimension reduction techniques are worth investigating.
\marginpar{Application of GSA coupled with FDA}
\gls[hyper=false]{gsa} methods and \gls[hyper=false]{fda} techniques were then coupled together to decompose the variance of the output in the reduced space.
This was done through the implementation and application of \gls[hyper=false]{mc} methods to estimate the Sobol', main- and total-effect, indices.  
The sensitivity analysis reveals interesting behavior of the \gls[hyper=false]{trace} model of \gls[hyper=false]{feba} in terms of interactions

The implementations of the employed \gls[hyper=false]{gsa} methods were developed in-house as a python module to allow full internal control.
\marginpar{\texttt{gsa-module}}
The module \texttt{gsa-module} is documented and was tested against a suite of test functions and applied to obtain all the results presented in Chapter~\ref{ch:gsa}.

% Metamodeling and Bayesian Calibration
