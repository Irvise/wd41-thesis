%*************************************************************************************************************************
\section{Uncertainty Quantification in Nuclear Engineering Thermal-Hydraulics}\label{sec:intro_uncertainty_quantification}
%*************************************************************************************************************************

% Introductory Paragraph

%----------------------------------------------------------------------------------------------
\subsection{Statistical Uncertainty Analysis}\label{sub:intro_statistical_uncertainty_analysis}
%----------------------------------------------------------------------------------------------

% Best-estimate, limitation
As explained, best-estimate analysis uses more realistic modeling assumptions for analyzing transient behavior of \gls[hyper=false]{npp}.
It attempts as realistically as possible to describe the behavior of the relevant physical processes occur during the plant transient.
And yet, even the best available understanding of the physical process is still limited.
Understanding of complex phenomena might not yet adequate and data support for some processes can be very limited.
Simplifying assumptions, approximations, and expert judgments to some degree are unavoidable and still required to have a complete analysis.

% Best-estimate, plus uncertainty

% Source of possible uncertainties

% Statistical uncertainty analysis, Inputs as random variables 

% Source of uncertainty, initial and boundary condition

% Source of uncertainty, physical model parameters

% Inverse uncertainty

% Connection to PREMIUM Benchmark

%---------------------------------------------------------------
\subsection{OECD/NEA PREMIUM Benchmark}\label{sub:intro_premium}
%---------------------------------------------------------------

% Introductory paragraph
The \gls[hyper=false]{premium} benchmark was an activity launched by the \gls[hyper=false]{oecd}/\gls[hyper=false]{nea} in $2012$ and concluded in $2016$ with the aim to advance the methods for quantifying the uncertainties associated with the physical model parameters in \gls[hyper=false]{th} system codes.
It was the continuation of the previous project \gls[hyper=false]{bemuse}, which concetrated on the propagation and sensitivity analysis of the input uncertainties in large scale simulation (large break \gls[hyper=false]{loca}).
The main finding of \gls[hyper=false]{bemuse} can be found in \cite{Perez2011}.
The emphasis of the \gls[hyper=false]{premium} benchmark was placed on the derivation of the model parameters uncertainty and their validation.

% Scope of the Project

% Main Findings

%-------------------------------------------------------
\subsection{GRS Methodology}\label{sub:intro_grs_method}
%-------------------------------------------------------

%-----------------------------------------------------------
\subsection{FFTBM Methodology}\label{sub:intro_fftbm_method}
%-----------------------------------------------------------

%-------------------------------------------------------
\subsection{CIRC\'E}\label{sub:intro_circe_method}
%-------------------------------------------------------
