%*********************************************************************************
\section{Objectives and Scope of the Thesis}\label{sec:intro_objectives_and_scope}
%*********************************************************************************

%--------------------------------------------------------------------------
\subsection{Statement of the Problem}\label{sub:intro_statement_of_problem}
%--------------------------------------------------------------------------

% Introductory Paragraph
The development of closure laws for reflooding described in \cite{Nelson1992,USNRC2012} showed the difficulties and the amount of assumptions used.
In a nutshell, system code development is an effort to consolidate correlations and mechanistic models, to create a phenomenological-based simulator code that can provide best-estimate results.
This consolidated effort results in a code that can simulate wide range of transients foreseen in nuclear power plant operation in a best-estimate manner.
Alas, to come up with a consistent set of closure laws is a great challenge for code developers.

% Closure Laws Difficulty, Conceptual
The closure laws required to close the two-fluid model pose particularly difficult challenges \cite{Wulff2007}.
For instance, to have a correlation of heat transfer between the wall and the fluid, temperature data from each of the constituents are needed (i.e., the wall, the liquid phase, and the gas phase).
But measuring temperature of the individual phases in an arbitrary interfacial topology has its own technical difficulties to the extend that no such data exists or available to be implemented in the closure laws.
Additionally, the experiments to obtain hydrodynamic closure laws (e.g., interfacial friction factor, wall friction factor, etc.) were generally carried out in adiabatic conditions.
As a result, this excludes the coupling of any heat transfer phenomena between the phases and the wall in such correlation.

% Closure Laws Difficult, Practical
Furthermore, during the development of a simulation code, programming considerations also came into the picture.
For robustness, simplification is often required and continuity is enforced.
Transitionary flow regime between two known (observed) flow regimes for which experimental data is not available is modeled to be the average of the two bounding regimes.
Different code development, which used different assumptions and experimental database, comes up with different set of closure laws with their own parametrization (see for instance \cite{Nelson1992} for TRAC code and \cite{Bestion1990} for CATHARE code).
Several authors have expressed their concerns about the uncertainty stemming from the closure laws \cite{Wulff2007,Petruzzi2008a,DAuria2012}.

% an Illustration
As an example of the point given above, consider that in the \gls[hyper=false]{trace} code, after some derivations the interfacial drag coefficient closure law in the inverted slug flow regime $C_{i,\text{IS}}$ is given by,
\begin{equation*}
	C_{i,\text{IS}} = \hat{x}_{m,\text{SET}} \times \frac{1}{24} \frac{\rho_g}{\text{La}} \frac{(1-\alpha)}{\alpha^{1.8}} \,\,\,;\,\,\, \hat{x}_{m,\text{SET}} = 0.75 
\label{eq:intf_drag_isf}
\end{equation*}
where $\rho_g$ is the density of the gas phase;
$\text{La}$ is the Laplace number;
$\alpha$ is the void fraction;
and $\hat{x}_{m,\text{SET}}$ is a fitting parameter.

There are several remarks about the closure law given above.
First, the second term in the right-hand side was derived from experimental data but not directly.

% Closing
As illustrated above, it is clear that models in thermal-hydraulics system code, to a certain extent, flawed.
Various experimental programs were carried out to gain better understanding of important phenomena,
and to validate (and, as noted above, to calibrate) the models.
Series of the experiments, carried out in \glspl[hyper=false]{setf} were aimed to reproduce limited part of the transient in a selected component following a postulated scenario.
For example, in the case of reflooding, several facilities existed and data were available (FEBA, PERICLES, etc.).
But, there has not been an orchestrated effort to incorporate the accumulated data into the calibration process of the physical models, in a systematic way, while acknowledging multiple sources of the uncertainty in the process.

%--------------------------------------------------
\subsection{Objectives}\label{sub:intro_objectives}
%--------------------------------------------------

%----------------------------------------
\subsection{Scope}\label{sub:intro_scope}
%----------------------------------------
