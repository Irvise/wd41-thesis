\section{Parameters Screening}\label{sec:sa_parameters_screening}

Screening methods are used to rank the importance of the model parameters using a relatively small number of model evaluations \cite{Saltelli2004}.
However, they tend to simply give qualitative measures.
That is, meaningful information resides in the rank itself but not in the exact importance of the parameters with respect to the output. 
These methods are particularly valuable in the early phase of a SA to identify the non-influential parameters of a model, 
which then could be safely excluded from further detailed analysis. 
This exclusion step is important to reduce the size of the problem especially if a more expensive method is to be applied at the subsequent step. 
In this work, attention was paid to a particular screening method proposed by Morris \cite{Morris1991} with an extension proposed by Campolongo et al. cite.

\subsection{Elementary Effects}

Consider a model with $D$ parameters, where $\mathbf{x} = (x_1, x_2, \dots,x_D)$ is a vector of parameter value evaluated at point $\mathbf{x}$.
The elementary effect of the $d$-th parameter is defined as

\begin{equation}
EE_d = \frac{y(x_1, x_2, \dots, x_d+\Delta,\dots,x_D) - y(x_1, x_2, \dots, x_d,\dots,x_D)}{\Delta}
\end{equation}

where

The key idea behind the original Morris method is in initiating the mdeol evaluations from various \textit{nominal} points $\mathbf{x}$ randomly selected over the grid and then gradually advancing one grid jump at a time between each model evaluation (one at a time), along a different dimension of the parameter space selected randomly. This ensures 

To remove the necessity to specify the method-specific parameter $p$ (the number of levels), Campolongo et al. proposed to use a radial based design coupled with Sobol' quasirandom sequence (cite). In addition to removing the number of levels, radial design does not require nominal points to lie in a grid and at the same time varies the size of parameter perturbations from one dimension to another, from replica to replica, incorporating additional possible sources of variation in the method.
An illustration of a radial design in $2$-dimensional input space with $4$ base points is shown in Fig.~ as a comparison with the trajectory design.

 
\subsection{Statistics of Elementary Effects and Sensitivity Measures}

