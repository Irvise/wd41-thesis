%****************************************************
\section{Implementation}\label{sec:sa_implementation}
%****************************************************

In this work, 
an implementation of the Morris method and a Monte Carlo method to estimate the main- and total-effect sensitivity indices has been developed using the Python \cite{PCT2017} programming language, 
to allow for well-controlled parametric and convergence studies.
The implementation follows a black box approach of \gls[hyper=false]{sa}. 
It deals with the generation of design of experiment (a set of input values at which the model or code is evaluated) and the post-processing of output to obtain the selected measures of sensitivity presented in the previous sub-sections.
In the following, the basic procedures that underlie the implementation of both methods are laid out.
More details on the programming aspects of the implementation (the so-called \texttt{gsa-module}) can be found in Appendix~\ref{app:gsa_module}.

%--------------------------------------------------
\subsection{The Morris Method}\label{sub:sa_morris}
%--------------------------------------------------

The implementation of the Morris method (see Section~\ref{sub:sa_ee_oat}) follows four sequential steps:
\begin{enumerate}

	\item An OAT design matrix consisting of $N_R$ replications is created by randomly sampling the nominal points as well as the perturbed points for each parameter.
A replication in an OAT design consists of one nominal point with $D$ (number of dimensions/parameters) additional perturbed points.
In each of the perturbed points, only one parameter change its value relative to the base.
Note that in trajectory design, the nominal point only serves as the starting point and a perturbed point becomes the base point for the next perturbation.
Different replications yield different nominal points and the associated perturbed points.

  \item Each point in the design matrix, included in $[0,1]^D$, is scaled to the corresponding point in the $D$-dimensional parameter space of the model parameters through iso-probabilistic transformation (see Appendix~\ref{app:its}).

  \item The model is evaluated for each (rescaled) point of the design matrix. The total number of model evaluations for a given design matrix is $N_R \times (D + 1)$.
  
  \item Finally, for a selected \gls[hyper=false]{qoi} the $N_R$ elementary effects $EE_d$ are computed for each input parameter $x_d$. The statistical summaries $\mu_d, \mu^*_d,$ and $\sigma_d$ are computed, and the ranking of the input parameters for the selected \gls[hyper=false]{qoi} is established based on $\mu^*_d$.

\end{enumerate}

The different rankings based on $\mu_d^*$ obtained from various relevant \glspl[hyper=false]{qoi} can then be used and compared to consistently identify and screen out noninfluential parameters (low $\mu_d^*$) from the relatively influential ones (high $\mu_d^*$)~\cite{Campolongo2007}.

%-------------------------------------------------------------------
\subsection{The Sobol'-Saltelli Method}\label{sub:sa_sobol_saltelli}
%-------------------------------------------------------------------

In principle, the estimation of the Sobol' sensitivity indices defined by Eqs.~(\ref{eq:sa_main_effect_index}) and (\ref{eq:sa_total_effect_index}) can be directly carried out using \gls[hyper=false]{mc} simulation.
\marginpar{Brute force \\ Monte Carlo}
The most straightforward, though rather rudimentary, 
estimation method is to use two nested loops for the computation of the conditional variances and expectations appearing in both indices.

In the estimation of the main-effect index of parameter $x_d$, for instance, 
the outer loop samples values of $\mathcal{X}_d$ while the inner loop samples values of $\bm{\mathcal{X}}_{\sim d}$ (all parameters except $x_d$).
These samples, in turn, are used to evaluate the model and generate the output realizations.

Algorithm~\ref{alg:brute_force_mc} illustrates the procedure to compute the variance of conditional expectation used in main-effect indices estimation of a parameter $x_d$.
In the inner loop, the arithmetic mean of the model output is taken for a given value of $\mathcal{X}_d$ but over many values of $\bm{\mathcal{X}}_{\sim d}$. 
Afterward, in the outer loop, the variance of the model output is taken over many values of $\mathcal{X}_d$.

% Brute forcem Monte Carlo Pseudo-code
\begin{algorithm}
\caption{Brute Force \gls[hyper=false]{mc} for estimating $\mathbb{V}_d [\mathbb{E}_{\sim d} [\mathcal{Y}|\mathcal{X}_d]]$}
\label{alg:brute_force_mc}
\begin{algorithmic}
\STATE $\Sigma_i \leftarrow 0$
\STATE $\Sigma_{i^2} \leftarrow 0$
  \FOR{$i=1$ to $N$}
    \STATE sample $x_d^{(i)}$ from $\mathcal{X}_d$
    \STATE $\Sigma_j \leftarrow 0$
    \FOR{$j=1$ to $N$}
      \STATE sample $\bm{x}_{\sim d}^{(j)}$ from $\bm{\mathcal{X}}_{\sim d}$
      \STATE $\Sigma_j \mathrel{+}= f(\bm{x}_{\sim d}^{(j)}, x_d^{(i)})$ 
    \ENDFOR
    \STATE $\mathbb{E}_{\sim d} [\mathcal{Y}|\mathcal{X}_d]^{(i)} \leftarrow \frac{1}{N} \Sigma_j$
    \STATE $\Sigma_i \mathrel{+}= \mathbb{E}_{\sim d} [\mathcal{Y}|\mathcal{X}_d]^{(i)}$
    \STATE $\Sigma_{i^2} \mathrel{+}= \mathbb{E}_{\sim d} [\mathcal{Y}|\mathcal{X}_d]^{(i)} \times \mathbb{E}_{\sim d} [\mathcal{Y}|\mathcal{X}_d]^{(i)}$
  \ENDFOR
  \STATE $\mathbb{V}_d [\mathbb{E}_{\sim d} [\mathcal{Y}|\mathcal{X}_d]] \leftarrow \frac{1}{N} \left(\Sigma_{i^2} - \Sigma_i^2/N\right)$
\end{algorithmic}
\end{algorithm}

% The cost of brute formce MC
Algorithm~\ref{alg:brute_force_mc} can easily become prohibitively expensive as the nested structure requires $N^2$ model evaluations \emph{per input dimension} for one of the sensitivity indices (i.e., the main- or total-effect index), 
while $N$ (the size of \gls[hyper=false]{mc} samples) are typically in the range of $10^2 - 10^4$ for a reliable estimate. 

% Sobol'-Saltelli Approach
Sobol' \cite{Sobol2001} and Saltelli \cite{Saltelli2002} proposed an alternative approach that circumvent the nested structure of \gls[hyper=false]{mc} simulation to estimate the indices.
\marginpar{Sobol'-Saltelli method}
The formulation starts by expressing the expectation and variance operators in their integral form 
and ends with different possible \gls[hyper=false]{mc} estimators for both sensitivity indices.
A detailed derivation of the integral form and the origin of the estimator can be found in Appendix~\ref{app:sobol_saltelli}. 

An implementation of the Sobol'-Saltelli method is also part of \texttt{gsa-module} python3 package (see Appendix~\ref{app:gsa_module} for detail). 
For $N$ number of \gls[hyper=false]{mc} samples and $D$ number of model parameters, the \gls[hyper=false]{mc} simulation procedure to estimate the sensitivity indices follows the sampling and resampling approach adopted in~\cite{Sobol2001,Saltelli2002,Homma1996} described in the following.

\begin{enumerate}

  \item Generate two $N \times D$ independent random samples $A$ and $B$ from a uniform independent distribution in $D$-dimension, $[0,1]^D$:
    \begin{equation}
      A = 
      \begin{pmatrix}
        a_{11}  & \cdots  & a_{1D}\\
        \vdots	& \ddots & \vdots\\
        a_{N1}  & \cdots  & a_{ND}\\
      \end{pmatrix}
      ;\quad B = 
      \begin{pmatrix}
        b_{11}  & \cdots  & b_{1D}\\
        \vdots	& \ddots & \vdots\\
        b_{N1}  & \cdots  & b_{ND}\\
        \end{pmatrix}
    \label{eq:ss_two_samples}
    \end{equation}

  \item Construct $D$ additional design matrices from $A$ and $B$ where each matrix $A^d_B$ is matrix $A$ with the $d$-th column substituted by the $d$-th column of $B$:
  \begin{equation}
    \begin{split}
      & A_{B}^1 = 
      \begin{pmatrix}
        b_{11}  & \cdots  & a_{1D}\\
        \vdots	& \ddots & \vdots\\
        b_{N1}  & \cdots  & a_{ND}\\
      \end{pmatrix} \\
      & A_{B}^{d} = 
      \begin{pmatrix}
        a_{11}  & \cdots & b_{1d} & \cdots & a_{1D}\\
        \vdots	& \cdots & \vdots & \cdots & \vdots\\
        a_{N1}  & \cdots & b_{Nd} & \cdots & a_{ND}\\
      \end{pmatrix} \\
      & A_{B}^{D} = 
      \begin{pmatrix}
        a_{11}  & \cdots  & b_{1D}\\
        \vdots	& \ddots & \vdots\\
        a_{N1}  & \cdots  & b_{ND}\\
      \end{pmatrix}
    \end{split}
  \label{eq:ss_sampling_resampling}
  \end{equation}

  \item Rescale each element in the matrices of samples to the actual values of model parameters according to their actual range of variation through iso-probabilistic transformation (Appendix~\ref{app:its}).

  \item Evaluate the model multiple times using input parameters vectors that correspond to each row of $A$, $B$, and all $A_B^d$.

  \item Finally, extract the \gls[hyper=false]{qoi}s from all the outputs and recast them as vectors,
  which will be used to estimate the main- and total-effect indices according to a selected estimator described below.
  
\end{enumerate}

For the main-effect sensitivity index, two estimators are considered.
One is proposed by Saltelli~\cite{Saltelli2002}, and the other, as an alternative, is proposed by Janon et al~\cite{Janon2014}.
The latter proved to be more efficient, especially for a large variation around a parameter estimate~\cite{Iooss2015,Janon2014}.

The general form of main-effect sensitivity index estimator is
\marginpar{Main-effect sensitivity index estimator}
\begin{equation}
  \widehat{S}_d = \frac{\frac{1}{N}\sum_{n=1}^N f(B)_n \cdot f(A_B^d)_n - \mathbb{E}^2[\mathcal{Y}]}{\mathbb{V}[\mathcal{Y}]}
\label{eq:ss_main_effect_estimator}
\end{equation}
where the subscript $n$ corresponds to the row of the sampled model parameters 
such that $f(B)_n$ and $f(A^d_B)_n$ are the model outputs evaluated using inputs taken from the $n$-th row of matrix $B$ 
and matrix $A_B^D$, respectively.
The estimators for the term $\mathbb{E}^2[\mathcal{Y}]$ and $\mathbb{V}[\mathcal{Y}]$ differs for the two indices estimators and are given in Table~\ref{tab:ss_main_effect_estimator}.
%The first term in the numerator of Eq.~(\ref{eq:ss_main_effect_estimator}) is the same for both Saltelli and Janon et al. estimators and is given by
%\begin{equation}
%  \int \int f(\mathbf{x}^{\prime}_{\sim d}, x_d) f(\mathbf{x}_{\sim d}, x_d) d\mathbf{x}^{\prime}_{\sim d} d\mathbf{x}_{\sim d} \approx \frac{1}{N}\sum_{n=1}^N f(B)_n \cdot f(A_B^d)_n
%\label{eq:ss_first_term}
%\end{equation}

\begin{table}[h]
	\myfloatalign
	\caption[Monte Carlo estimators to estimate the main-effect indices]{Two \gls[hyper=false]{mc} estimators for the terms in Eq.~(\ref{eq:ss_main_effect_integral}) to estimate the main-effect indices (the sum is taken implicitly over all samples $N$)}
	\label{tab:ss_main_effect_estimator}
	\begin{tabularx}{\textwidth}{Xll} \toprule
		\tableheadline{Estimator}         & $\mathbb{E}^2[\mathcal{Y}] = \left( \int f d\mathbf{x}\right)^2$          & $\mathbb{V}[\mathcal{Y}] = \int f^2 d\mathbf{x} - \left( \int f d\mathbf{x}\right)^2$ \\ \midrule 
		Saltelli \cite{Saltelli2002}      & \footnotesize{$\frac{1}{N} \sum f(A)_n \cdot f(B)_n$}                     & \footnotesize{$\frac{1}{N}\sum f(A)_n^2-\left(\frac{1}{N}\sum f(A)_n\right)^2$}  \\[0.75cm]
		Janon et~al.~\cite{Janon2014}     & \footnotesize{$\left(\frac{1}{2N} \sum f(B)_n + f(A_B^d)_n\right)^2$}      & \footnotesize{$\frac{1}{2N} \sum f(B)_n^2 + f(A_B^d)_n^2$} \\
                                      &                                                                 & \footnotesize{$\quad -\left(\frac{1}{2N} \sum f(B)_n^2 + f(A_B^d)_n^2\right)^2$} \\
		\bottomrule
	\end{tabularx}
\end{table}

To estimate the total-effect sensitivity indices, the Jansen estimator~\cite{Jansen1999} is recommended in~\cite{Saltelli2010a}.
\marginpar{Total-effect sensitivity index estimator}
The estimator reads
\begin{equation}
  \widehat{ST}_d = \frac{\frac{1}{2N}\sum_{n=1}^{N}\left(f(A)_n - f(A_B^d)_n\right)^2}{\mathbb{V}[\mathcal{Y}]}
\label{eq:ss_jansen_estimator}
\end{equation}
where $\mathbb{V}[\mathcal{Y}]$ is estimated by the Saltelli et al. estimator in Table~\ref{tab:ss_main_effect_estimator}.

% Sobol-Saltelli Computational Cost
The computational cost associated with the estimation of all the main-effect and total-effect indices using the Sobol'-Saltelli method is $N \times (D + 2)$ code runs,
\marginpar{computational cost: \\ brute force Monte Carlo vs. Sobol'-Saltelli}
where $N$ is the number of \gls[hyper=false]{mc} samples and $D$ is the number of parameters.
Compare this to the cost of brute force Monte Carlo of $2 \times D \times N^2$ code runs to estimate all the main-effect and total-effect sensitivity indices. 

% Morris method computational cost
As an additional comparison, the cost for Morris method to compute the statistics of elementary effect is $N_R \times (D + 1)$ code runs,
\marginpar{computational cost: \\ Morris vs. Sobol'-Saltelli}
where $N_R$ is the number of OAT design replications.
In either methods, the number of samples $N$ (in the case of the Sobol'-Saltelli method) and replications $N_R$ (in the case of the Morris method)
determines the precision of the estimates.
A larger number of samples (and replications) increases the precision.
Note, however, that in practice the typical number of Morris replications is between $10^1 - 10^2$~\cite{Saltelli2010}, 
while the number of \gls[hyper=false]{mc} samples for the Sobol' indices estimation amounts to $10^2 - 10^4$~\cite{Sobol2001}.
