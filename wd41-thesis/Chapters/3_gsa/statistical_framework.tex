\section{Statistical Framework}\label{sec:sa_statistical_framework}

The methodology for SA presented in this work belongs to
the category of statistical framework, a term attributed
to Cacuci and Ionescu-Bujor \cite{Ionescu-Bujor2004} or simply the global
method, following the terminology from Saltelli et al. \cite{Saltelli2004}.
Within this framework, sensitivity measures of a parameterized
model are obtained by postprocessing the collection
of model outputs obtained from multiple model
evaluations at different points in the input parameter space
according to a certain experimental design. 
As such, the model itself can be considered as a black box, 
and the input parameters are modeled as random variables
equipped with a joint \gls{pdf}. 
The specification of a joint \gls{pdf} allows for experimental design generation.

Consider the following mathematical model used as a template for the rest of the present chapter:
\begin{equation}
y(t) = f(t;\mathbf{x}), \, t \in [t_a,t_b]
\end{equation}
where $y(t)$ is the scalar output at time $t$ from a deterministic
function $f$ and $\mathbf{x}$ is the input parameter vector in
$D$-dimension, i.e., $\mathbf{x} = (x_1, x_2, \dots , x_d, \dots , x_D)$.
It is customary to assume, for generality, that the input parameters are normalized between $[0,1]$ (in other words $\mathbf{x} \in [0,1]^D$).

Let $\mathbf{DM}$ be an experimental design matrix of size
$N \times D$, where $N$ is the number of samples. 
Each row in the matrix represents a point in the $D$-dimensional parameter space. 
The model is then evaluated at each of these $D$ points by using a simulation code that results in a matrix of discrete-time outputs of size $N\times \frac{t_b-t_a}{\Delta t}$ where $\Delta t$ is the time-step size,
\begin{equation}
y(t; \mathbf{DM}) = 
\begin{pmatrix}
y_1(t_a)  & \cdots & y_1(t_i) & \cdots & y_1(t_b)\\
\vdots	  &        & \vdots   &        & \vdots\\
y_n(t_a)  & \cdots & y_n(t_i) & \cdots & y_n(t_b)\\
\vdots	  &        & \vdots   &        & \vdots\\
y_N(t_a)  & \cdots & y_N(t_i) & \cdots & y_N(t_b) \\
\end{pmatrix}
\label{eq:discrete_time}
\end{equation}

Based on this general model time-dependent output description, 
the next three sections will outline the main components of the proposed SA methodology.
