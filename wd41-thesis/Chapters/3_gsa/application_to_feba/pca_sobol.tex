%*****************************************************************************************************
\subsection{Sobol' Indices for QoIs based on Principal Components}\label{sub:sa_application_pca_sobol}
%*****************************************************************************************************

% Introductory paragraph
The \gls[hyper=false]{fpc} score $\theta_{n,i}$ associated with each single realization $n$ and a principal component $i$ is used as the \gls[hyper=false]{qoi} in a \gls[hyper=false]{sa} similar to what was done for the conventional \gls[hyper=false]{qoi} is Section~\ref{sub:sa_application_conventional_sobol}.
In other words, the variance of the score is decomposed into the variance contribution associated with each model parameter.

% 1st fPC of Registered TC4
The estimated Sobol' indices for the first \gls[hyper=false]{fpc} of the (registered) mid-height clad temperature transient are given in Fig.~\ref{fig:ch3_plot_sobol_boxplot_tc4_pc1}.
As can be seen, the variation in the amplitude of the temperature reversal was mainly due to the spacer grid HT enhancement and the \gls[hyper=false]{dffb}-related model parameters.
The other parameters are proved to be almost noninfluential.
This result is consistent with the result obtained when using the maximum clad temperature as the \gls[hyper=false]{qoi} 
and confirms the maximum cladding temperature as a viable representative \gls[hyper=false]{qoi} during the temperature reversal period.
\bigfigure[pos=tbhp,
					 opt={width=0.85\textwidth},
           label={fig:ch3_plot_sobol_boxplot_tc4_pc1},
           shortcaption={Sobol' Indices estimates with the $1$\textsuperscript{st} \gls[hyper=false]{fpc} scores of the (registered) mid-height clad temperature transient as the \gls[hyper=false]{qoi}}]
{../figures/chapter3/figures/plotSobolBoxplotTC4RegPC1}
{The sensitivity indices with respect to the $1$\textsuperscript{st} \gls[hyper=false]{fpc} of the (registered) mid-height clad temperature transient. Each boxplot represents the bootstrap sample quartile statistics and the vertical line extends the $95$th sample percentile.}

% 2nd fPC of Registered TC4
Fig.~\ref{fig:ch3_plot_sobol_boxplot_tc4_pc2} shows the Sobol' sensitivity indices using the scores associated with the second \gls[hyper=false]{fpc} of the (registered) mid-height clad temperature transient as the \gls[hyper=false]{qoi}.
Contrary to the first component, most of the variation in the second \gls[hyper=false]{fpc} can only be explained through interactions between model parameters since the main-effect indices only summed up to $27\%$ of the total variance.
The difference between the main-effect and total-effect indices are large for all input parameters, but especially for the \gls[hyper=false]{dffb}-related parameters.
These parameter interactions, associated with a particular mode of variation, could not be captured from the conventional \glspl[hyper=false]{qoi} (e.g., the maximum cladding temperature),
or only in speculative manner from the time-dependent representation of the sensitivity indices showed in Fig.~\ref{fig:ch3_plot_si_evol}, which gave a less concise description of the parameter sensitivity.
\bigfigure[pos=!tbhp,
					 opt={width=0.85\textwidth},
           label={fig:ch3_plot_sobol_boxplot_tc4_pc2},
           shortcaption={Sobol' Indices estimates with the $2$\textsuperscript{nd} \gls[hyper=false]{fpc} scores of the (registered) mid-height clad temperature transient as the \gls[hyper=false]{qoi}}]
{../figures/chapter3/figures/plotSobolBoxplotTC4RegPC2}
{The sensitivity indices with respect to the $2$\textsuperscript{nd} \gls[hyper=false]{fpc} of the (registered) mid-height clad temperature transient. Each boxplot represents the bootstrap samples quartile statistics and the vertical line extends the $95$th sample percentile.}

% 1st fPC of Warping TC4
The sensitivity indices with respect to the first \gls[hyper=false]{fpc} of the warping functions for the clad temperature transient at the mid-height of the assembly are shown in Fig.~\ref{fig:ch3_plot_sobol_boxplot_tc4_warp_pc1}.
The spacer grid HT enhancement model is the main source of variation in the time-shift of the landmarks,
while two \gls[hyper=false]{dffb}-related parameters (\texttt{dffbWHT} and \texttt{dffbIntDr}) and the quenching temperature each contributes around $10\%$ to the total output variation.
In comparison, the rest of the parameters have a trivial effect to the shift.
Only a small portion of the output variation is due to parameters interaction from the fact that the main-effect indices summed up to a value close to $1.0$ ($94\%$).
\bigfigure[pos=!tbhp,
					 opt={width=0.85\textwidth},
           label={fig:ch3_plot_sobol_boxplot_tc4_warp_pc1},
           shortcaption={Sobol' Indices estimates with the $1$\textsuperscript{st} \gls[hyper=false]{fpc} scores of the warping function for the mid-height clad temperature transient as the \gls[hyper=false]{qoi}}]
{../figures/chapter3/figures/plotSobolBoxplotTC4WarpPC1}
{The sensitivity indices with respect to the $1$\textsuperscript{st} \gls[hyper=false]{fpc} of the warping function for the mid-height clad temperature transient. Each boxplot represents the bootstrap sample quartile statistics and the vertical line extends the $95$th sample percentile.}

% 1st fPC of DP Pressure Drop
Fig.~\ref{fig:ch3_plot_sobol_boxplot_dpmid_pc1} presents the sensitivity indices with respect to the scores associated with the first \gls[hyper=false]{fpc} of the pressure drop transient at the middle of the assembly.
The inlet mass velocity parameter (\texttt{fillV}) is the main contributor to the overall output variation ($\sim 30\%$), while the two interfacial drag parameters of the reflood model amount to the same combined contribution.  
\bigfigure[pos=!tbhp,
					 opt={width=0.85\textwidth},
           label={fig:ch3_plot_sobol_boxplot_dpmid_pc1},
           shortcaption={Sobol' Indices estimates with the $1$\textsuperscript{st} \gls[hyper=false]{fpc} of the pressure drop transient at the middle of the assembly as the \gls[hyper=false]{qoi}}]
{../figures/chapter3/figures/plotSobolBoxplotDPMidPC1}
{The sensitivity indices with respect to the $1$\textsuperscript{st} \gls[hyper=false]{fpc} of the pressure drop transient at the middle of the assembly. Each boxplot represents the bootstrap sample quartile statistics and the vertical line extends the $95$th sample percentile}

% 1st fPC of Liquid Carryover
Finally, Fig.~\ref{fig:ch3_plot_sobol_boxplot_co_pc1} shows the sensitivity indices with the first \gls[hyper=false]{fpc} of the liquid carryover transient as the \gls[hyper=false]{qoi}.
The inlet mass velocity parameter (\texttt{fillV}) is by far the main source of variation in the output variation ($\sim 90\%$),
followed by minor contributions ($\sim 9\%$) from two \gls[hyper=false]{dffb}-related parameters (\texttt{dffbVIHT} and \texttt{dffbIntDr}),
and the rest of the parameters have a negligible effect.
\bigfigure[pos=tbhp,
					 opt={width=0.85\textwidth},
           label={fig:ch3_plot_sobol_boxplot_co_pc1},
           shortcaption={Sobol' indices estimates with the $1$\textsuperscript{st} \gls[hyper=false]{fpc} of the liquid carryover transient as the \gls[hyper=false]{qoi}.}]
{../figures/chapter3/figures/plotSobolBoxplotCOPC1}
{The main-effect and total-effect sensitivity indices with respect to the $1$\textsuperscript{st} \gls[hyper=false]{fpc} of the liquid carryover transient. Each boxplot represents the bootstrap sample quartile statistics and the vertical line extends the $95$th sample percentile}

% Closing Paragraph
The numerical results of the estimated Sobol' indices presented above are tabulated in Table~\ref{tab:app_sobol_tc4_maxtemp} through \ref{tab:app_sobol_co_pc1} in Appendix~\ref{app:tbl_results_sobol}.
To give a measure of uncertainties on the estimates due to \gls[hyper=false]{mc} sampling, 
the results in the table are complemented by the $95$th bootstrap percentile confidence interval $CI_{pct}$ \cite{Efron1986}.