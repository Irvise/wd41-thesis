%*******************************************************************************************
\subsection{Principal Components of the Reflood Curve}\label{sub:sa_application_pca_reflood}
%*******************************************************************************************

% Introductory paragraph
To derive a more novel set of \glspl[hyper=false]{qoi},
variance decomposition was carried out on the \gls[hyper=false]{fpc} scores of the whole time-dependent curves of the different types of outputs (i.e., clad temperature, pressure drop, and liquid carryover).
\marginpar{explained variance}
Because each \gls[hyper=false]{fpc} is asscoiated with a particular mode of variation over the whole transient, it can parsimoniously describe the overall variation of the time-dependent curve in a smaller set of numbers.
The \gls[hyper=false]{fpc} analysis of all the time-dependent curves for each type of outputs showed that the first two respective \gls[hyper=false]{fpc} accounts for more than $85\%$ of the overall functional variations (see Fig.~\ref{fig:ch3_plot_explained_variance}).

\bigtriplefigure[pos=tbhp,
								 mainlabel={fig:ch3_plot_explained_variance},
			           maincaption={The proportion of explained variance for each \gls[hyper=false]{fpc} extracted from of each of the (time-dependent) outputs. The points with a connecting line are the cumulative explained variance, while the horizontal line is the cumulative of the first two \glspl[hyper=false]{fpc}.},
			           mainshortcaption={The proportion of explained variance for each \gls[hyper=false]{fpc} extracted from of each of the (time-dependent) outputs.},%
			           leftopt={width=0.30\textwidth},
			           leftlabel={fig:ch3_plot_explained_variance_1},
			           leftcaption={Mid-Assembly Temperature},
			           midopt={width=0.30\textwidth},
			           midlabel={fig:ch3_plot_explained_variance_dpmid},
			           midcaption={Mid. Pressure Drop},
			           rightopt={width=0.30\textwidth},
			           rightlabel={fig:ch3_plot_explained_variance_co},
			           rightcaption={Liquid Carryover},
			           spacing={},
			           spacingtwo={}]
{../figures/chapter3/figures/plotExplainedVarianceTC4}
{../figures/chapter3/figures/plotExplainedVarianceDPMid}
{../figures/chapter3/figures/plotExplainedVarianceCO}

% 1st fPC of the mid-assembly clad temperature
The first \gls[hyper=false]{fpc} of the (registered) clad temperature in the reflood transient and the effect of its perturbation around the mean function are shown in Fig.~\ref{fig:ch3_plot_pc1_tc4}.
\marginpar{the 1st \gls[hyper=false]{fpc} of the (registered) time-dependent clad temperature at the middle of the assembly}
The \gls[hyper=false]{fpc} shown in Fig.~\ref{fig:ch3_plot_pc1_tc4_fpc} was obtained by multiplying the eigenfunction with the square root of the respective eigenvalue, i.e., $fPC_j = \sqrt{\rho_j} \times \xi_j(t)$.
As the eigenfunction only represents the shape (mode) of function variation, this multiplication was done to give it a sense of scale with respect to the cladding temperature variation.
The perturbation around the mean function (Fig.~\ref{fig:ch3_plot_pc1_tc4_perturbation}) is done by adding to and subtracting from the mean function, the eigenfunction multiplied by twice the \gls[hyper=false]{fpc}, i.e., $\bar{y}(t) \pm 2 \times \sqrt{\rho_j} \times \xi_j(t)$.

\normdoublefigure[pos=tbhp,
                  mainlabel={fig:ch3_plot_pc1_tc4},
                  maincaption={The 1st \gls[hyper=false]{fpc} of the (registered) time-dependent clad temperature and the effect of its perturbation on the mean function.},%
									mainshortcaption={The 1st \gls[hyper=false]{fpc} of the (registered) time-dependent clad temperature and the effect of its perturbation on the mean function.},
                  leftopt={width=0.45\textwidth},
                  leftlabel={fig:ch3_plot_pc1_tc4_fpc},
                  leftcaption={1st \gls[hyper=false]{fpc} ($55\%$)},
                  %leftshortcaption={},%
                  rightopt={width=0.45\textwidth},
                  rightlabel={fig:ch3_plot_pc1_tc4_perturbation},
                  rightcaption={Perturbation on the mean},
                  %rightshortcaption={},
                  %spacing={\hfill}
                 ]
{../figures/chapter3/figures/plotPC1TC4.pdf}
{../figures/chapter3/figures/plotPC1PerturbationTC4.pdf}

This particular eigenfunction corresponds to a mode of variation that relates to the amplitude of the temperature reversal period.
This is the strongest mode of variation, accounting for $55\%$ of the overall clad temperature variation.

% 2nd fPC of the mid-assembly clad temperature
Fig.~\ref{fig:ch3_plot_pc2_tc4} shows the results for the second \gls[hyper=false]{fpc} of the (registered) reflood curve and the effect of its perturbation on the mean function.
\marginpar{the 2nd \gls[hyper=false]{fpc} of the (registered) time-dependent clad temperature at the middle of the assembly}
This particular mode relates to the variation in the temperature descent after reaching the maximum temperature, prior to quenching.
Visibly, some realizations tend to bring about more convexity in the temperature descent than the others.
This mode of variation constitutes $32\%$ of the overall temperature variation.
\normdoublefigure[pos=tbhp,
                  mainlabel={fig:ch3_plot_pc2_tc4},
                  maincaption={The 2nd \gls[hyper=false]{fpc} of the (registered) time-dependent clad temperature and the effect of its perturbation on the mean function.},%
									mainshortcaption={The 2nd \gls[hyper=false]{fpc} of the (registered) time-dependent clad temperature and the effect of its perturbation on the mean function.},
                  leftopt={width=0.45\textwidth},
                  leftlabel={fig:ch3_plot_pc1_tc4_fpc},
                  leftcaption={2nd \gls[hyper=false]{fpc} ($30\%$)},
                  %leftshortcaption={},%
                  rightopt={width=0.45\textwidth},
                  rightlabel={fig:ch3_plot_pc1_tc4_perturbation},
                  rightcaption={Perturbation on the mean},
                  %rightshortcaption={},
                  %spacing={\hfill}
                 ]
{../figures/chapter3/figures/plotPC2TC4.pdf}
{../figures/chapter3/figures/plotPC2PerturbationTC4.pdf}

% 1st fPC of the mid-assembly clad temperature warping function
The previous \glspl[hyper=false]{fpc} were carried on the registered clad temperature transient where the phase variations in the dataset have been removed.
However, it is also interesting to see the phase variations in the dataset separately.
This is done by carrying out the same procedure on resulting warping functions associated with each clad temperature realization.
Fig.~\ref{fig:ch3_plot_pc1_tc4_warp} shows the 1st \glspl[hyper=false]{fpc} and the effect of its perturbation on the mean function.
The mode corresponds to the overall shift in the timing of the two landmarks compared to the mean function. 
From the figure, delay in the maximum temperature tends also to result in a delay in the time of quenching, and vice versa.
However, the variation in the time of the maximum temperature tends to be much smaller than the variation in the time of quenching.
\normdoublefigure[pos=tbhp,
                  mainlabel={fig:ch3_plot_pc1_tc4_warp},
                  maincaption={The 1st \gls[hyper=false]{fpc} of the warping function of the time-dependent clad temperature and the effect of its perturbation on the mean function.},%
									mainshortcaption={The 2nd \gls[hyper=false]{fpc} of the time-dependent clad temperature and the effect of its perturbation on the mean function.},
                  leftopt={width=0.45\textwidth},%width=0.45\textwidth},
                  leftlabel={fig:ch3_plot_pc1_tc4_warp_fpc},
                  leftcaption={1st fPC (93\% Output variation)},
                  %leftshortcaption={},%
                  rightopt={width=0.45\textwidth},%width=0.45\textwidth},
                  rightlabel={fig:ch3_plot_pc1_tc4_warp_perturbation},
                  rightcaption={the effect of 1st fPC perturbation on the mean function},
                  %rightshortcaption={},
                  %spacing={\hfill}
                 ]
{../figures/chapter3/figures/plotPC1TC4Warp.pdf}
{../figures/chapter3/figures/plotPC1PerturbationTC4Warp.pdf}

% 1st fPC of the mid-assembly pressure drop
\normdoublefigure[pos=tbhp,
                  mainlabel={fig:ch3_plot_pc1_tc4},
                  maincaption={Mat\'ern kernels for $2$ different range parameters $\theta$ and, for each, $2$ different shape parameters $\nu$.},%
									mainshortcaption={Examples of Mat\'ern kernel functions},
                  leftopt={width=0.45\textwidth},%width=0.45\textwidth},
                  leftlabel={fig:ch3_plot_pc1_tc4},
                  leftcaption={},
                  %leftshortcaption={},%
                  rightopt={width=0.45\textwidth},%width=0.45\textwidth},
                  rightlabel={fig:ch3_plot_pc1_perturbation_tc4},
                  rightcaption={},
                  %rightshortcaption={},
                  %spacing={\hfill}
                 ]
{../figures/chapter3/figures/plotPC1DPMid.pdf}
{../figures/chapter3/figures/plotPC1PerturbationDPMid.pdf}

% 1st fPC of the liquid carryover
\normdoublefigure[pos=tbhp,
                  mainlabel={fig:ch3_plot_pc1_tc4},
                  maincaption={Mat\'ern kernels for $2$ different range parameters $\theta$ and, for each, $2$ different shape parameters $\nu$.},%
									mainshortcaption={Examples of Mat\'ern kernel functions},
                  leftopt={width=0.45\textwidth},%width=0.45\textwidth},
                  leftlabel={fig:ch3_plot_pc1_tc4},
                  leftcaption={1st fPC (93\% Output variation)},
                  %leftshortcaption={},%
                  rightopt={width=0.45\textwidth},%width=0.45\textwidth},
                  rightlabel={fig:ch3_plot_pc1_perturbation_tc4},
                  rightcaption={the effect of 1st fPC perturbation on the mean function},
                  %rightshortcaption={},
                  %spacing={\hfill}
                 ]
{../figures/chapter3/figures/plotPC1CO.pdf}
{../figures/chapter3/figures/plotPC1PerturbationCO.pdf}