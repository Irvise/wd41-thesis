%*******************************************************************************************
\subsection{Principal Components of the Reflood Curve}\label{sub:sa_application_pca_reflood}
%*******************************************************************************************

% Introductory paragraph
The time-dependent clad temperature, pressure drop, and liquid carryover were decomposed in their respective functional principal components (\glspl[hyper=false]{fpc}) to better quantify the mode of variations of the whole time dependent curves.
Therefore, the variance decomposition was also carried out on the \gls[hyper=false]{fpc} scores associated with the \glspl[hyper=false]{fpc}.
\marginpar{explained variance}
Because each \gls[hyper=false]{fpc} is associated with a particular mode of variation over the whole transient,
it parsimoniously describes the overall variation of the time-dependent curve in a smaller set of numbers.
The \gls[hyper=false]{fpc} analysis of all the time-dependent curves for each type of output showed that the first two respective \gls[hyper=false]{fpc} account for more than $85\%$ of the overall functional variations (see Fig.~\ref{fig:ch3_plot_explained_variance}).

\bigtriplefigure[pos=tbhp,
								 mainlabel={fig:ch3_plot_explained_variance},
			           maincaption={The proportion of explained variance for each \gls[hyper=false]{fpc} extracted from selected time-dependent outputs. The points with a connecting line are the cumulative explained variance, while the horizontal line is variance explained by first two \glspl[hyper=false]{fpc}.},
			           mainshortcaption={The proportion of explained variance for each \gls[hyper=false]{fpc} extracted from selected time-dependent outputs.},%
			           leftopt={width=0.30\textwidth},
			           leftlabel={fig:ch3_plot_explained_variance_1},
			           leftcaption={Mid-height assembly temperature},
			           midopt={width=0.30\textwidth},
			           midlabel={fig:ch3_plot_explained_variance_dpmid},
			           midcaption={Mid. of assembly pressure drop},
			           rightopt={width=0.30\textwidth},
			           rightlabel={fig:ch3_plot_explained_variance_co},
			           rightcaption={Liquid carryover},
			           spacing={},
			           spacingtwo={}]
{../figures/chapter3/figures/plotExplainedVarianceTC4}
{../figures/chapter3/figures/plotExplainedVarianceDPMid}
{../figures/chapter3/figures/plotExplainedVarianceCO}

% 1st fPC of the mid-assembly clad temperature
The first \gls[hyper=false]{fpc} of the (registered) clad temperature at the mid-height of the assembly (in the reflood transient) and the effect of its perturbation around the mean function are shown in Fig.~\ref{fig:ch3_plot_pc1_tc4}.
\marginpar{the 1\textsuperscript{st} \gls[hyper=false]{fpc} of the (registered) clad temperature transient at the mid-height of the assembly}
The \gls[hyper=false]{fpc} shown in Fig.~\ref{fig:ch3_plot_pc1_tc4_fpc} was obtained by multiplying the eigenfunction $\xi_j(t)$ with the square root of the respective eigenvalue, i.e., $fPC_j = \sqrt{\rho_j} \times \xi_j(t)$.
As the eigenfunction only represents the shape (mode) of function variation,
this multiplication was done to give it a sense of scale with respect to the cladding temperature variation (as $\sqrt{\rho_j}$ represents the standard deviation of the mode $j$).
The perturbation around the mean function (Fig.~\ref{fig:ch3_plot_pc1_tc4_perturbation}) is done by adding to and subtracting from the mean function,
the eigenfunction multiplied by twice the \gls[hyper=false]{fpc}, i.e., $\bar{y}(t) \pm 2 \times \sqrt{\rho_j} \times \xi_j(t)$.

\normdoublefigure[pos=tbhp,
                  mainlabel={fig:ch3_plot_pc1_tc4},
                  maincaption={The 1st \gls[hyper=false]{fpc} of the (registered) clad temperature transient at the mid-height of the assembly and the effect of its perturbation on the mean function.},%
									mainshortcaption={The 1st \gls[hyper=false]{fpc} of the (registered) clad temperature transient at the mid-height of the assembly and the effect of its perturbation on the mean function.},
                  leftopt={width=0.45\textwidth},
                  leftlabel={fig:ch3_plot_pc1_tc4_fpc},
                  leftcaption={1st \gls[hyper=false]{fpc} ($55\%$)},
                  %leftshortcaption={},%
                  rightopt={width=0.45\textwidth},
                  rightlabel={fig:ch3_plot_pc1_tc4_perturbation},
                  rightcaption={Perturbation on the mean},
                  %rightshortcaption={},
                  %spacing={\hfill}
                 ]
{../figures/chapter3/figures/plotPC1TC4.pdf}
{../figures/chapter3/figures/plotPC1PerturbationTC4.pdf}

This particular \gls[hyper=false]{fpc} corresponds to a mode of variation that relates to the amplitude of the temperature reversal period (see Chapter~2).
This is the strongest mode of variation, accounting for $55\%$ of the overall clad temperature variation.

% 2nd fPC of the mid-assembly clad temperature
Fig.~\ref{fig:ch3_plot_pc2_tc4} shows the results for the second \gls[hyper=false]{fpc} of the (registered) clad temperature transient at the mid-height of the assembly and the effect of its perturbation on the mean function.
\marginpar{the 2\textsuperscript{nd} \gls[hyper=false]{fpc} of the (registered) clad temperature transient at the middle of the assembly}
This mode relates to the variation in the temperature descent after reaching the maximum temperature, prior to quenching.
Visibly, some realizations tend to bring about more convexity in the temperature descent than the others.
This mode of variation constitutes $30\%$ of the overall variation.
\normdoublefigure[pos=tbhp,
                  mainlabel={fig:ch3_plot_pc2_tc4},
                  maincaption={The 2\textsuperscript{nd} \gls[hyper=false]{fpc} of the (registered) clad temperature transient at the mid-height of the assembly and the effect of its perturbation on the mean function.},%
									mainshortcaption={The 2\textsuperscript{nd} \gls[hyper=false]{fpc} of the (registered) clad temperature transient at the mid-height of the assembly and the effect of its perturbation on the mean function.},
                  leftopt={width=0.45\textwidth},
                  leftlabel={fig:ch3_plot_pc2_tc4_fpc},
                  leftcaption={2\textsuperscript{nd} \gls[hyper=false]{fpc} ($30\%$)},
                  %leftshortcaption={},%
                  rightopt={width=0.45\textwidth},
                  rightlabel={fig:ch3_plot_pc2_tc4_perturbation},
                  rightcaption={Perturbation on the mean},
                  %rightshortcaption={},
                  %spacing={\hfill}
                 ]
{../figures/chapter3/figures/plotPC2TC4.pdf}
{../figures/chapter3/figures/plotPC2PerturbationTC4.pdf}

% 1st fPC of the mid-assembly clad temperature warping function
The previous \glspl[hyper=false]{fpc} were carried on the registered clad temperature transient where the phase variations in the data set have been removed.
\marginpar{the 1\textsuperscript{st} \gls[hyper=false]{fpc} of the warping function of the clad temperature transient at the middle of the assembly}
It is also interesting to see the phase variations in the data set separately.
This can be done by carrying out the same procedure on the resulting warping functions associated with each clad temperature realization.
Fig.~\ref{fig:ch3_plot_pc1_tc4_warp} shows the 1st \gls[hyper=false]{fpc} and the effect of its perturbation on the mean function.
The mode corresponds to the overall shift in the timing of the two landmarks compared to the mean function. 
From the figure, a delay in the maximum temperature tends also to result in a delay in the time of quenching, and vice versa.
However, the variation in the time of the maximum temperature tends to be much smaller than the variation in the time of quenching.
\normdoublefigure[pos=tbhp,
                  mainlabel={fig:ch3_plot_pc1_tc4_warp},
                  maincaption={The 1\textsuperscript{st} \gls[hyper=false]{fpc} of the warping function of the clad temperature transient at the mid-height of the assembly and the effect of its perturbation on the mean function.},%
									mainshortcaption={The 1\textsuperscript{st} \gls[hyper=false]{fpc} of the warping function of the clad temperature transient at the mid-height of the assembly and the effect of its perturbation on the mean function.},
                  leftopt={width=0.45\textwidth},%width=0.45\textwidth},
                  leftlabel={fig:ch3_plot_pc1_tc4_warp_fpc},
                  leftcaption={1\textsuperscript{st} fPC (93\% Output variation)},
                  %leftshortcaption={},%
                  rightopt={width=0.45\textwidth},%width=0.45\textwidth},
                  rightlabel={fig:ch3_plot_pc1_tc4_warp_perturbation},
                  rightcaption={Perturbation on the mean},
                  %rightshortcaption={},
                  %spacing={\hfill}
                 ]
{../figures/chapter3/figures/plotPC1TC4Warp.pdf}
{../figures/chapter3/figures/plotPC1PerturbationTC4Warp.pdf}

% 1st fPC of the mid-assembly pressure drop
The first \gls[hyper=false]{fpc} of the pressure drop transient curves at the middle of the assembly is shown in Fig.~\ref{fig:ch3_plot_pc1_dpmid}.
\marginpar{the 1\textsuperscript{st} \gls[hyper=false]{fpc} of the pressure drop transient at the middle of the assembly}
The \gls[hyper=false]{fpc}, taking into account $77\%$ of the output variation, is mostly responsible for the variation during the pressure drop rise, where the channel segment is continuously quenched from the bottom.
That is, some realizations rise more quickly (or more slowly) in reaching the equilibrium pressure drop.
\normdoublefigure[pos=tbhp,
                  mainlabel={fig:ch3_plot_pc1_dpmid},
                  maincaption={The 1\textsuperscript{st} \gls[hyper=false]{fpc} of the pressure drop transient at the mid. height of the assembly and the effect of its perturbation on the mean function.},%
									mainshortcaption={The 1\textsuperscript{st} \gls[hyper=false]{fpc} of the pressure drop transient at the mid. height of the assembly and the effect of its perturbation on the mean function.},
                  leftopt={width=0.45\textwidth},%width=0.45\textwidth},
                  leftlabel={fig:ch3_plot_pc1_dpmid_fpc},
                  leftcaption={1\textsuperscript{st} fPC (77\% Output variation)},
                  %leftshortcaption={},%
                  rightopt={width=0.45\textwidth},%width=0.45\textwidth},
                  rightlabel={fig:ch3_plot_pc1_dpmid_perturbation},
                  rightcaption={Perturbation on the mean},
                  %rightshortcaption={},
                  %spacing={\hfill}
                 ]
{../figures/chapter3/figures/plotPC1DPMid.pdf}
{../figures/chapter3/figures/plotPC1PerturbationDPMid.pdf}

% 1st fPC of the liquid carryover
The first \gls[hyper=false]{fpc} of the liquid carryover transient curves, shown in Fig.~\ref{fig:ch3_plot_pc1_co}, are straightforward to interpret.
\marginpar{the 1\textsuperscript{st} \gls[hyper=false]{fpc} of the liquid carryover transient}
The \gls[hyper=false]{fpc}, taking into account $93\%$ of the output variation, is the linear change of the average liquid carryover during the transient.
In other words, the perturbation on the liquid carryover is accumulated linearly over time.
\normdoublefigure[pos=tbhp,
                  mainlabel={fig:ch3_plot_pc1_co},
                  maincaption={The 1\textsuperscript{st} \gls[hyper=false]{fpc} of the liquid carryover transient and the effect of its perturbation on the mean function.},%
									mainshortcaption={The 1\textsuperscript{st} \gls[hyper=false]{fpc} of the liquid carryover transient and the effect of its perturbation on the mean function.},
                  leftopt={width=0.45\textwidth},%width=0.45\textwidth},
                  leftlabel={fig:ch3_plot_pc1_co_fpc},
                  leftcaption={1\textsuperscript{st} fPC (93\% Output variation)},
                  %leftshortcaption={},%
                  rightopt={width=0.45\textwidth},%width=0.45\textwidth},
                  rightlabel={fig:ch3_plot_pc1_co_perturbation},
                  rightcaption={Perturbation on the mean},
                  %rightshortcaption={},
                  %spacing={\hfill}
                 ]
{../figures/chapter3/figures/plotPC1CO.pdf}
{../figures/chapter3/figures/plotPC1PerturbationCO.pdf}