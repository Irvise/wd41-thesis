\section{Variance Decomposition}\label{sec:sa_variance_decomposition}

Variance-based methods for global sensitivity analysis use variance as the basis to define a measure of input parameter influence on the overall output variation \cite{Cacuci2004}.
In a statistical framework of sensitivity and uncertainty analysis, 
this choice is natural because variance (or standard deviation, a related concept) is often used as a measure of dispersion or variability in the model prediction \cite{Saltelli2008}.
A measure of such dispersion, in turn, indicates the precision of the prediction due to variations in the input parameters.

This section, a method to decompose the output variance into the contributions from the input variances is presented.
Afterward, two sensitivity measures based on the decomposition and a method for their estimation are introduced.

\subsection{High-Dimensional Model Representation}\label{sub:sa_hdmr}

Consider once more a mathematical model $f: \mathbf{x} \in [0,1]^D \mapsto y = f(\mathbf{x}) \in \mathbb{R}$.
The high-dimensional model representation (HDMR) if $f(\mathbf{x})$ is a linear combination of functions with increasing dimensionality \cite{Li2001},
\begin{equation}
	\begin{split}
		f(\mathbf{x}) = f_o & + \sum_{d=1,2,...D} f_d(x_d) + \sum_{1\leq d < e \leq D} f_{d,e} (x_d, x_e) + \cdots  \\
	                      & + f_{1,2,\cdots,D} (x_1, x_2, \cdots, x_D)
	\end{split}
\label{eq:sa_hdmr}
\end{equation}
where $f_o$ is a constant. 
The representation in Eq.~(\ref{eq:sa_hdmr}) is unique given the following condition \cite{Sobol2001}:
\begin{equation}
  \begin{split}
    \int_{0}^{1} & f_{i1, i2, \cdots is}(x_{i1}, x_{i2}, \cdots, x_{is}) d_{x_{im}} = 0  \\
                 & \text{for}\quad m = 1,2,\cdots,s;\quad 1\leq i_1 < i_2 < \cdots < i_s \leq D; \\
                 & \text{and} \quad s \in {1,\cdots,D}
   \end{split}
\label{eq:sa_unicity}
\end{equation}

Assume now that $\mathbf{X}$ is a random vector of independent and uniform random variables over a unit hypercube
$\{\Omega = \mathbf{x} | 0 \leq x_i  \leq 1; i = 1,\cdots D\}$ such that
\begin{equation}
	Y = f(\mathbf{X})
\label{eq:sa_random_function}
\end{equation}
where $Y$ is a random variable, resulting from the transformation of random vector $\mathbf{X}$ by function $f$.
Using Eq.~(\ref{eq:sa_unicity}) to express each term in Eq.~(\ref{eq:sa_hdmr}), it follows that
\begin{equation}
	\begin{split}
		f_o & = \mathbb{E}[Y] \\
	  f_d(x_d) & = \mathbb{E}_{\sim d}[Y|X_d] \\
    f_{d,e}(x_d,x_e) & = \mathbb{E}_{\sim d,e} [Y|X_d, X_e] - \mathbb{E}_{\sim d}[Y|X_d] - \mathbb{E}_{\sim e}[Y|X_e] - \mathbb{E}[Y] 
	\end{split}
\label{eq:sa_conditional_expectation}
\end{equation}

The same follows for higher-order terms in the decomposition. 
In Eq.~(\ref{eq:sa_conditional_expectation}), $\mathbb{E}_{\sim e} [Y|X_e]$ corresponds to the conditional expectation operator,
and the $\sim\circ$ in the subscript means that the integration over the parameter space is carried out over all parameters except the specified parameter in the subscript.
For instance, $\mathbb{E}_{\sim 1} [Y|X_1]$ refers to the conditional mean of $Y$ given $X_1$, and the integration is carried out for all possible values of parameters in $\mathbf{x}$ except $x_1$.
Note that because $X_1$ is a random variable, the expectation conditioned on it is also a random variable.

Assuming that $f$ is a square integrable function, applying the variance operator on $Y$ results is
\begin{equation}
	\begin{split}
		\mathbb{V}[Y] = \sum_{d=1}^{D} \mathbb{V}[f_d (x_D)] & + \sum_{1 \leq d < e \leq D} \mathbb{V} [f_{d,e} (x_d, x_e)] + \cdots \\
	                                                       & + \mathbb{V} [f_{1,2,\cdots,D} (x_1, x_2, \cdots, x_D)]
		\end{split}
\label{eq:sa_variance_decomposition}
\end{equation}

\subsection{Sobol' Sensitivity Indices}\label{sub:sa_sobol_indices}

Division by $\mathbb{V}[Y]$ aptly normalizes Eq.~(\ref{eq:sa_variance_decomposition}):
\begin{equation}
  1 = \sum_{d = 1}^{D} S_d + \sum_{1 \leq d < e \leq D} S_{d,e} + \cdots + S_{1,2,\cdots,D}
\label{eq:sa_normalized_variance}
\end{equation}

The Sobol' main-effect sensitivity index $S_d$ is defined as,
\marginpar{the main-effect index}
\begin{equation}
  S_d = \frac{\mathbb{V}_d [\mathbb{E}_{\sim d} [Y|X_d]]}{\mathbb{V}[Y]}
\label{eq:sa_main_effect_index}
\end{equation}
The numerator is the variance of the conditional expectation,
and the index is a global sensitivity measure interpreted as the amount of variance reduction in the model output if the parameter $X_d$ is fixed (i.e., its variance is reduced to zero).
\pagebreak

A closely related sensitivity index proposed by Homma and Saltelli \cite{Homma1996} is the Sobol' total-effect index defined as,
\marginpar{the total-effect index}
\begin{equation}
  \begin{split}
    ST_{d} & = \frac{\mathbb{E}_{\sim d}[\mathbb{V}_{d}[Y|\mathbf{X}_{\sim d}]]}{\mathbb{V}[Y]}
%           & = \frac{\mathbb{V}[Y] - \mathbb{V}_{\sim d}\left[\mathbb{E}_{d}\left[Y|\mathbf{X}_{\sim d}\right]\right]}{\mathbb{V}[Y]} \\
%           & = 1 - \frac{\mathbb{V}_{\sim d}[\mathbb{E}_{d}[Y|\mathbf{X}_{\sim d}]}{\mathbb{V}[Y]}
  \end{split}
\label{eq:sa_total_effect_index}
\end{equation}
The index, also a global sensitivity measure, can be interpreted as the amount of variance left in the output if the values of all input parameters, 
\emph{except} $x_d$, can be fixed.

These two sensitivity measures can be related to the objectives of global SA for model assessment as proposed by Saltelli et al \cite{Saltelli2004,Saltelli2008}.
\marginpar{parameter prioritization objective}
The main-effect index is relevant to parameter prioritization in the context of identifying the most influential parameter 
since fixing a parameter with the highest index value would, \emph{on average}, lead to 
the greatest reduction in the output variation.

The total-effect index, on the other hand, is relevant to parameter fixing (or screening) in the context of identifying the least influential set of parameters since fixing any parameter that has a very small 
\marginpar{parameter screening objective}
total-effect index value would not lead to significant reduction in the output variation.
The use of total-effect index to identify which parameter can be fixed or excluded is similar to that of the elementary effect statistics of the Morris method, 
albeit more exact but also more computationally expensive to compute.
And finally, the difference between the two indices of a given parameter (Eqs.~(\ref{eq:sa_total_effect_index}) and (\ref{eq:sa_main_effect_index})) is used to quantify the amount of all interactions involving that parameters in the model output.
