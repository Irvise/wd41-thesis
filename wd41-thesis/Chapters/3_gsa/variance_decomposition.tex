\section{Variance Decomposition}\label{sec:sa_variance_decomposition}

Variance-based methods for global sensitivity analysis use variance as the basis to define a measure of input parameter influence on the overall output variation \cite{Cacuci2004}.
In a statistical framework of sensitivity and uncertainty analysis, 
this choice is natural because variance (or standard deviation, a related concept) is often used as a measure of dispersion or variability in the model prediction \cite{Saltelli2008}.
A measure of such dispersion, in turn, indicates the precision of the prediction due to variations in the input parameters.

This section, a method to decompose the output variance into the contributions from the input variances is presented.
Afterward, two sensitivity measures based on the decomposition and a method for their estimation are introduced.

\subsection{High-Dimensional Model Representation}\label{sec:sa_hdmr}

Consider once more a mathematical model $f: \mathbf{x} \in [0,1]^D \mapsto y = f(\mathbf{x}) \in \mathbb{R}$.
The high-dimensional model representation (HDMR) if $f(\mathbf{x})$ is a linear combination of functions with increasing dimensionality \cite{Li2001},
\begin{equation}
	\begin{split}
		f(\mathbf{x}) = f_o & + \sum_{d=1,2,...D} f_d(x_d) + \sum_{1\leq d < e \leq D} f_{d,e} (x_d, x_e) + \cdots  \\
	                      & + f_{1,2,\cdots,D} (x_1, x_2, \cdots, x_D)
	\end{split}
\label{eq:sa_hdmr}
\end{equation}
where $f_o$ is a constant. 
The representation in Eq.~\ref{eq:sa_hdmr} is unique given the following condition \cite{Sobol2001}:
\begin{equation}
	\int_{0}^{1} 
\label{eq:sa_unicity}
\end{equation}

Assume now that $\mathbf{X}$ is a random vector of independent and uniform random variables over a unit hypercube
$\{\Omega = \mathbf{x} | 0 \leq x_i  \leq 1; 1 = 1,\cdots D\}$ such that
\begin{equation}
	Y = f(\mathbf{X})
\label{eq:sa_random_function}
\end{equation}
where $Y$ is a random variable, resulting from the transformation of random vector $\mathbf{X}$ by function $f$.
Using Eq.~\ref{eq:sa_unicity} to express each term in Eq.~\ref{eq:sa_hdmr}, it follows that
\begin{equation}
	\begin{split}
		f_o & = \mathbb{E}[Y] \\
	  f_d(x_d) & = \mathbb{E}_{\sim d}[Y|X_d] \\
    f_{d,e}(x_d,x_e) & = \mathbb{E}_{\sim d,e} [Y|X_d, X_e] - \mathbb{E}_{\sim d}[Y|X_d] - \mathbb{E}_{\sim e}[Y|X_e] - \mathbb{E}[Y] 
	\end{split}
\label{eq:sa_conditional_expectation}
\end{equation}

The same follows for higher-order terms in the decomposition. 
In Eq.~\ref{eq:sa_conditional_expectation}, $\mathbb{E}_{\sim e} [Y|X_e]$ corresponds to the conditional expectation operator,
and the $\sim\circ$ in the subscript means that the integration over the parameter space is carried out over all parameters except the specified parameter in the subscript.
For instance, $\mathbb{E}_{\sim 1} [Y|X_1]$ refers to the conditional mean of $Y$ given $X_1$, and the integration is carried out for all possible values of parameters in $\mathbf{x}$ except $x_1$.
Note that because $X_1$ is a random variable, the expectation conditioned on it is also a random variable.

Assuming that $f$ is a square integrable function, applying the variance operator on $Y$ results is
\begin{equation}
	\begin{split}
		\mathbb{V}[Y] = \sum_{d=1}^{D} \mathbb{V}[f_d (x_D)] & + \sum_{1 \leq d < e \leq D} \mathbb{V} [f_{d,e} (x_d, x_e)] + \cdots \\
	                                                       & + \mathbb{V} [f_{1,2,\cdots,D} (x_1, x_2, \cdots, x_D)]
		\end{split}
\label{eq:sa_variance_decomposition}
\end{equation}

\subsection{Sobol' Sensitivity Indices}\label{sub:sa_sobol_indices}

Division by $\mathbb{V}[Y]$ aptly normalizes Eq.~:

