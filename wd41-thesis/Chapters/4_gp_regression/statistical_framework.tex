\section{Statistical Framework}\label{sec:gp_statistical_framework}

Consider a (possibly nonlinear) \emph{regression} problem:
Given a deterministic computer code (which again, in essence a function) $f:\mathbf{x} \in \boldsymbol{\chi} \subseteq \mathbb{R}^D \mapsto \mathbb{R}$ 
evaluated at $\mathbf{DM}$, an experimental design matrix $\{\mathbf{x}_i\}_{i=1}^N$, 
resulting in a finite set inputs/outputs pair of such code $\{\mathbf{x}_i, f(\mathbf{x}_i) = y_i\}_{i=1}^N$\footnote{often referred to as \emph{training data}},
the objective of regression is to compute (or \emph{predict}) the value of $\mathbf{x}^*$ with $\mathbf{x}^* \notin \mathbf{DM}$.
As before, the domain $\boldsymbol{\chi}$ is often rescaled such that $\mathbf{x} \in [0,1]^D$.

To evaluate $f$ at any given $\mathbf{x}^*$ not in the original design matrix (or training data), 
the function has to be reconstructed or estimated based on the given training data.
Furthermore, the true underlying function $f(\circ)$ that produces $y_i$ itself might be too complex and expensive to evaluate, 
such as in the case of a complex computer code.
This fact also often limits the size of training data.
As such, the estimated function is chosen to be a simpler function that can be evaluated much faster (such as polynomials).
Although simpler, such an approximation should capture the most, if not all, important aspects of the inputs/outputs relationship of the true underlying function.
This simpler, approximating function is often referred to as an \emph{emulator}, \emph{surrogate model}, or \emph{meta-model}.

In this thesis, 
the meta-model is represented using \gls[hyper=false]{gp}, following the seminal works of Sacks et al. (\cite{Sacks1989, Sacks1989a}) and interpreted through a Bayesian perspective.
The advantages of using \gls[hyper=false]{gp} to represent an unknown function are its ability to model a complicated multi-dimensional function with limited number of parameters (\cite{Jones2009})
as well as the provision of prediction error estimate (\cite{Santner2003,Currin1991}).
The Bayesian perspective, in turn, gives more intuitive interpretation of the prediction error in terms of (epistemic) uncertainty.

% Summary of interpretations
To summarize, the statistical process model above can be interpreted either in frequentistic sense or Bayesian sense.
In frequentistic sense, the stochastic process $Y(\circ)$ is a particular realization of stochastic process.
The prediction made at particular value of $\mathbf{x}$ is made based on the process as estimated based on the training data.
On the other had, in Bayesian sense, a Gaussian process is first set up as the prior for the stochastic process and the
prediction made at particular value of of $\mathbf{x}$ is made based on the posterior (or conditional) process as updated by the data.   
These two possible interpretations ultimately yield a subtle difference in the interpretation of the associated prediction uncertainty.

\paragraph{a Bayesian Perspective}

