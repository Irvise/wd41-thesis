%*************************************************************************************
\subsection{Dimension Reduction by \glsfirst[hyper=false]{pca}}\label{sub:gp_feba_pca}
%*************************************************************************************

% Introductory paragraph
As mentioned, the different outputs considered in this study (i.e., clad temperature, pressure drop, and liquid carryover) were of multivariate nature.
For instance, the clad temperature output was defined both in time instance and axial location.
Considering eight different axial locations for the thermocouples and $1'000 \, [s]$ transient (to ensure that all runs were quenched) with time step size of $0.1 \, [s]$, the dimensionality of this output amounted to $80'000$.
Fig.~\ref{fig:ch4_plot_image_tc_run} shows three different clad temperature outputs from three different \gls[hyper=false]{trace} runs.
It shows the contour plot of clad temperature as function of time and axial location in the $x$- and $y$- axes, respectively.
Though not shown here, the other two types of output were of similar nature: pressure drop was also defined in time instance and in four axial segments (with dimensionality of the output amounting to $40'000$), while liquid carryover output was defined only in time (with dimensionality amounting to $10'000$).
\bigtriplefigure[pos=tbhp,
                 mainlabel={fig:ch4_plot_image_tc_run},
			           maincaption={Examples of multivariate clad temperature output $[K]$ at eight different locations as function of time, presented as ``images'', taken from three different training runs.},
			           mainshortcaption={Examples of multivariate clad temperature output in at eight different locations as function of time, presented as ``images''},%
			           leftopt={width=0.305\textwidth},
			           leftlabel={fig:ch4_plot_image_tc_run_1},
			           leftcaption={Training run 996},
			           midopt={width=0.305\textwidth},
			           midlabel={fig:ch4_plot_image_tc_run_2},
			           midcaption={Training run 1480},
			           rightopt={width=0.305\textwidth},
			           rightlabel={fig:ch4_plot_image_tc_run_3},
			           rightcaption={Training run $1805$},
			           spacing={},
			           spacingtwo={}]
{../figures/chapter4/figures/plotImageTCRun1}
{../figures/chapter4/figures/plotImageTCRun2}
{../figures/chapter4/figures/plotImageTCRun3}

% Multivariate Clad Temperature
\gls[hyper=false]{pca} was used to reduce this significant number of output dimensions.
Fig.~\ref{fig:ch4_plot_pca_image} shows the sample mean surface and the first two \glspl[hyper=false]{pc} (loadings) estimated using $1'920$ training samples.
These two \glspl[hyper=false]{pc} explained about $83\%$ of the output variance in training samples.
This implied that any realization of the training samples could be reconstructed by using the mean surface 
added by linear combination of the \gls[hyper=false]{pc} multiplied by a unique set of scalars (the standardized \gls[hyper=false]{pc} scores) associated with that realization, such that on average (over many realizations) the reconstruction would be $83\%$ accurate with respect to the \gls[hyper=false]{rmse}.
In other words, ``images'' in Fig.~\ref{fig:ch4_plot_image_tc_run} can be reconstructed by overlapping the mean surface and the multiples of \glspl[hyper=false]{pc} ``images'' shown in Fig.~\ref{fig:ch4_plot_pca_image}.
\bigtriplefigure[pos=tbhp,
                 mainlabel={fig:ch4_plot_pca_image},
			           maincaption={PCA results for the clad temperature output},
			           mainshortcaption={PCA results for the clad temperature output},%
			           leftopt={width=0.29\textwidth},
			           leftlabel={fig:ch4_plot_pca_image_ave},
			           leftcaption={Mean surface},
			           midopt={width=0.29\textwidth},
			           midlabel={fig:ch4_plot_pca_image_pc1},
			           midcaption={$1$\textsuperscript{st} PC loading ($73\%$)},
			           rightopt={width=0.29\textwidth},
			           rightlabel={fig:ch4_plot_pca_image_pc2},
			           rightcaption={$2$\textsuperscript{nd} PC loading ($10\%$)},
			           spacing={},
			           spacingtwo={}]
{../figures/chapter4/figures/plotPCAImageAve}
{../figures/chapter4/figures/plotPCAImagePC1}
{../figures/chapter4/figures/plotPCAImagePC2}

% RMSE of PCA, different output
Fig.\ref{fig:ch4_plot_reconstruction_error} summarizes the convergence behavior of the reconstruction error for the three types of output with increasing number of \glspl[hyper=false]{pc} (up to the first $10$, out of $80'000$, $40'000$, and $10'000$ \glspl[hyper=false]{pc} obtained for the respective outputs) used for the reconstruction.
The plots were obtained from reconstructing the outputs in the validation data set (with sample size of $5'000$) using a set of \glspl[hyper=false]{pc} derived from the training data set and by computing the average of the squared error over all realizations in the validation data set.
\bigtriplefigure[pos=tbhp,
                 mainlabel={fig:ch4_plot_reconstruction_error},
			           maincaption={The reconstruction error, in terms of \gls[hyper=false]{rmse}, as a function of the number of \glspl[hyper=false]{pc} used in the reconstruction of the output space for three different output types.},
			           mainshortcaption={Convergence of the reconstruction error as function of the number of \gls[hyper=false]{pc} used in the reconstruction of the output space for three different output types.},%
			           leftopt={width=0.28\textwidth},
			           leftlabel={fig:ch4_plot_reconstruction_error_tc},
			           leftcaption={Clad temperature output},
			           midopt={width=0.28\textwidth},
			           midlabel={fig:ch4_plot_reconstruction_error_dp},
			           midcaption={Pressure drop output},
			           rightopt={width=0.28\textwidth},
			           rightlabel={fig:ch4_plot_reconstruction_error_co},
			           rightcaption={Liquid carryover output},
			           spacing={},
			           spacingtwo={}]
{../figures/chapter4/figures/plotReconstructionErrorTC}
{../figures/chapter4/figures/plotReconstructionErrorDP}
{../figures/chapter4/figures/plotReconstructionErrorCO}

Because the \gls[hyper=false]{pc} scores used in the reconstruction were exact, the plot shows the magnitude of error to be expected from the dimension reduction procedure for each type of outputs.
It also shows that the benefit of using larger number of \gls[hyper=false]{pc} is increasingly marginal.
