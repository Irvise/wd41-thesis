%**************************************************************************
\subsection{Simulation Experiment}\label{sub:gp_feba_simulation_experiment}
%**************************************************************************

% Introductory Paragraph
The construction of \gls[hyper=false]{gp} \gls[hyper=false]{pc} metamodel of the \gls[hyper=false]{feba} \gls[hyper=false]{trace} model was carried out in three steps following the recommendation in the statistical/machine learning literature (\cite{Rasmussen2006, Hastie2009}): training, validation, testing as summarized in Fig.~\ref{fig:ch4_simulation_experiment} below.
\bigfigure[pos=tbhp,
           opt={width=1.0\textwidth},
           label={fig:ch4_simulation_experiment},
           shortcaption={The flowchart of the simulation experiment for constructing a \gls[hyper=false]{gp} \gls[hyper=false]{pc} metamodel of the \gls[hyper=false]{trace} model of the \gls[hyper=false]{feba} facility}]
{../figures/chapter4/lpe/simulation_experiment}
{The flowchart of the simulation experiment for constructing a \gls[hyper=false]{gp} \gls[hyper=false]{pc} metamodel of the \gls[hyper=false]{trace} model of the \gls[hyper=false]{feba} facility}

% Training
In the training step, a metamodel was constructed (i.e. \emph{trained}) based on a training dataset. 
\marginpar{training, experimental design}
The training dataset consists of a set of training inputs and its code output counteparts (from actual code runs). 
The training inputs of certain size was generated using an experimental design algorithm, of which several of them were considered (simple random sampling, latin hypercube, optimized latin hypercube, and Sobol' sequence).
Following the recommendation in \cite{Loeppky2009} the starting sample size was 120 (or $10\times D$, with $D$ is the number of dimension) and increased in multiples of two ($240$, $480$, and $960$).
Furthermore, to take into account the effect of random variation of the training dataset of the same design and size,
$5$ replications of each were generated for training.
That is, for a given experimental design algorithm of a given size, $5$ different set of metamodels were trained.

% Dimension Reduction
Metamodels predicting different types of outputs (i.e., clad temperature, pressure drop, and liquid carryover) were constructed separately.
\marginpar{dimension reduction}
These different outputs were themselves of a multivariate nature:
liquid carryover was a time-dependent quantity,
while clad temperature and pressure drop were a time- and space-dependent quantities.
As such, \gls[hyper=false]{pca} was carried out on the raw outputs to reduce their dimensionality and 
the metamodel was trained with respect to a few retained \gls[hyper=false]{pc} scores.

Carrying out \gls[hyper=false]{pca} for each type of outputs results in pairs of standardized \gls[hyper=false]{pc} scores and \gls[hyper=false]{pc} loadings.
A \gls[hyper=false]{gp} metamodel was then trained with respect to the standardized \gls[hyper=false]{pc} scores for selected number of retained \gls[hyper=false]{pc} loadings.
As such, there would be multiple \gls[hyper=false]{gp} metamodels representing the standardized \gls[hyper=false]{pc} scores associated with each of the retained \gls[hyper=false]{pc} loadings for each of the output types.

% Model Fitting
Several covariance kernel functions were considered for constructing a metamodel: Gaussian, Mat\'ern $3/2$, Mat\'ern $5/2$, and power exponential kernels.
\marginpar{model fitting}
The hyper-parameters associated with each kernel were estimated (i.e., \emph{fitted}) by \gls[hyper=false]{mle} as implemented in the \texttt{R} package \texttt{DiceKriging} \cite{Roustant2012}.
Following the hyper-parameters estimation for each \gls[hyper=false]{gp} \gls[hyper=false]{pc} metamodel, a metamodel of the \gls[hyper=false]{trace} model was fully trained and ready for making prediction in arbitrary inputs.
To make prediction back in the original physical space, the full output space had to be first reconstructed using linear combinations of the predicted standardized \gls[hyper=false]{pc} scores and the \gls[hyper=false]{pc} loadings (See Eq.~(\ref{eq:p_variate_metamodel})).

% Validation
Validation step was conducted to assess and compare the predictive performance of different metamodels constructed with different experimental design and covariance functions,
taking into account the effects of randomness in the experimental design generation and of training sample size.
\marginpar{validation and selection}
The validation step was conducted by means of independent validation dataset, a separate set of \gls[hyper=false]{trace} outputs from actual runs (preferably large enough) for the metamodels to predict.
A validation metric measured the discrepancy between the output from the validation dataset and from the prediction by the metamodel, and thus the quality of the metamodel.
It was quantified by the predictivity coefficient defined in Eq.~(\ref{eq:predictivity_coefficient}) when comparing metamodels of the standardized \gls[hyper=false]{pc} scores.
However, to have a more intuitive measure of performance directly related to the output on the physical space (as opposed to the reduced space of the principal components),
the \gls[hyper=false]{rmse} of the reconstructed prediction was used which is defined below,
\begin{equation}
	\begin{split}
	& RMSE_{reconstructed} = \left[\frac{1}{N_{valid} P}  \sum_{n=1}^{N_{valid}} \sum_{p=1}^{P} \left(y_{np} - \hat{y}_{np}\right)^2\right]^{0.5} \\
  & \hat{\bm{y}}_n = \bar{\mathbf{Y}} + \boldsymbol{\Phi}_Q \mathbf{m}_{SK}(\bm{x}_n) \\
  & \hat{\bm{y}}_n = (\hat{y}_{n1}, \ldots, \hat{y}_{np}, \ldots, \hat{y}_{nP})
  \end{split}
\label{eq:pc_rmse}
\end{equation}
where $N_{valid}$ is the number of validation data;
$P$ is the number of dimension of the output space;
$y_{np}$ is the value of the output dimension $p$ at validation input $n$;
$\hat{y}_{np}$ is the predicted value of the output dimension $p$ at validation input $n$;
and $\hat{\bm{y}}_n$ is the mean of the reconstructed multivariate output at validation input $n$ predicted by the \gls[hyper=false]{gp} \gls[hyper=false]{pc} metamodel (See the explanation of Eq.~(\ref{eq:p_variate_metamodel}) for detail).
This error combined the error due to the misprediction of the standardized \gls[hyper=false]{pc} scores by the metamodel as well as the error due to the truncation of the \glspl[hyper=false]{pc}.
The best setting of the metamodel (the experimental design and the covariance function) was then selected based on the \gls[hyper=false]{rmse} and one additional metamodel was trained using an increased number of training samples.

% Testing
Finally, the ultimate performance of the metamodel were assessed in the testing step based on yet another large number of test dataset, separately generated.
\marginpar{testing}
The purpose of this step was to further confirm the previous results on another dataset 
and to give a more robust idea on the expected error of the metamodel in the application setting.

% Tables
The settings used in the simulation experiment for constructing and assessing the \gls[hyper=false]{gp} \gls[hyper=false]{pc} metamodel is summarized in Table~\ref{tab:ch4_simulation_experiment}.
\begin{table}[!htpb]
    \myfloatalign
    \caption{Simulation experiment settings for constructing and assessing the \gls[hyper=false]{gp} \gls[hyper=false]{pc} metamodel of the \gls[hyper=false]{trace} model of \gls[hyper=false]{feba} Test No. $216$}
    \label{tab:ch4_simulation_experiment}
    \begin{tabularx}{\textwidth}{Xl} \toprule
        \tableheadline{Description} & \tableheadline{Value} \\ \midrule
        \textbf{\small{Inputs/Outputs}}       & \\
        \footnotesize{Number of inputs} 	    & \footnotesize{$12$} \\
        \footnotesize{Number of outputs:}     & \\		
        \scriptsize{Clad temperature}	        & \footnotesize{$8$ (axial locations) $\times 10'000$ (time-steps) $= 80'000$} \\
        \scriptsize{Pressure drop}		        & \footnotesize{$4$ (axial segments) $\times 10'000$ (time-steps) $= 40'000$}   \\
        \scriptsize{Liquid carryover}	        & \footnotesize{$10'000$ (time-steps)}   \\
        \footnotesize{Dimension reduction}    & \footnotesize{\glsfirst[hyper=false]{pca}} \\
        \midrule
        \textbf{\small{Training}}             & \\
        \footnotesize{Experimental Designs}   & \footnotesize{Simple random (SRS), latin hypercube (LHS),} \\
                                              & \footnotesize{optimized latin hypercube (Opt. LHS), Sobol' sequence} \\
        \footnotesize{Sample sizes}  	        & \footnotesize{$120, 240, 480, 960, 1'920$ (only for testing)}\\
        \footnotesize{Replication}           	& \footnotesize{$5$}\\
        \footnotesize{Covariance kernels}     & \footnotesize{Gaussian, Mat\'ern $3/2$, Mat\'ern $5/2$, power exponential} \\
        \footnotesize{Model fitting}          & \footnotesize{\glsfirst[hyper=false]{mle}} \\
        \midrule
        \textbf{\small{Validation}}           & \\
        \footnotesize{Strategy}		            & \footnotesize{Independent dataset (holdout)} \\
        \footnotesize{Experimental design}    & \footnotesize{Latin hypercube} \\
        \footnotesize{Sample size}            & \footnotesize{$5'000$} \\
        \footnotesize{Validation metric} 		  & \footnotesize{$Q_2$ (Eq.~(\ref{eq:predictivity_coefficient})) and $RMSE$ (Eq.~(\ref{eq:pc_rmse}))} \\
        \midrule
        \textbf{\small{Testing}}            & \\
        \footnotesize{Strategy}		          & \footnotesize{Independent dataset (holdout)} \\
        \footnotesize{Experimental design}  & \footnotesize{Latin hypercube} \\
        \footnotesize{Sample size}          & \footnotesize{$5'000$} \\
        \bottomrule
    \end{tabularx}
\end{table}
