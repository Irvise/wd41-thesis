%*******************************************************************************************************************************************
\subsection[GP PC Metamodel Testing]{\gls[hyper=false]{gp} \gls[hyper=false]{pc} Metamodel Testing}\label{sub:gp_feba_pca_metamodel_testing}
%*******************************************************************************************************************************************

% Introductory paragraph
Based on the results presented above, a final set of \gls[hyper=false]{gp} metamodels was constructed with a larger training set of $1'920$ samples using a power-exponential covariance kernel function based on a Sobol' sequence (the best options found).
Furthermore, $7$, $10$, and $5$ \glspl[hyper=false]{pc} were used in the reconstruction of the clad temperature, pressure drop, and liquid carryover outputs, respectively.
As such there were $22$ separate \gls[hyper=false]{gp} \gls[hyper=false]{pc} metamodels. 
An additional (i.e., \emph{testing}) dataset of size $5'000$ was then independently generated (with actual \gls[hyper=false]{trace} code runs) 
and used as the basis for testing the predictive performance of the final model.
This additional step was done to avoid any possible bias due to the fact that the validation data set was already used to select the final model.

% Convergence of error measure, test dataset
The validation metric $Q_2$ of the metamodels with respect to the standardized \glspl[hyper=false]{pc} scores computed on the testing dataset converged for all types of output (Fig.~\ref{fig:ch4_plot_q2_pcs}).
In other words, the size of the testing dataset was found to be (or more than) sufficient to assess the predictive performance of the selected metamodel.
\bigfigure[pos=!tbhp,
					 opt={width=1.0\textwidth},
           label={fig:ch4_plot_q2_pcs},
           shortcaption={Convergence of the predictive performance of the metamodel with respect to the standardized \glspl[hyper=false]{pc} scores for each output type}]
{../figures/chapter4/figures/plotQ2PCs}
{Convergence of the predictive performance of the metamodel with respect to the standardized \glspl[hyper=false]{pc} scores for each output type. Shown above are the first $10$ \glspl[hyper=false]{pc} for the clad temperature and pressure drop outputs and the first $5$ \glspl[hyper=false]{pc} for the liquid carryover output. For the clad temperature output, the predictivity coefficient falls below $0.75$ after the first $7$ \glspl[hyper=false]{pc}.}

% The final reconstruction error
There were two main sources of error that dictated the predictive performance of a \gls[hyper=false]{gp} \gls[hyper=false]{pc} metamodel.
The first was due to the representation of the full output dimension only with a few selected \glspl[hyper=false]{pc} (i.e., the dimension reduction) 
and the second was due to the misprediction of the standardized \gls[hyper=false]{pc} scores by the \gls[hyper=false]{gp} metamodel (i.e., the functional approximation).
Fig.~\ref{fig:ch4_plot_rec_error_pred_obs} illustrates these errors by presenting the predicted and observed reconstruction error (in terms of \gls[hyper=false]{rmse}) for each realization in the testing data set.
Note that the \emph{observed} reconstruction error was obtained using the reference standardized \gls[hyper=false]{pc} scores of the testing data set,
while the \emph{predicted} reconstruction error was obtained using the standardized \gls[hyper=false]{pc} scores as predicted by the \gls[hyper=false]{gp} metamodels.
\bigtriplefigure[pos=tbhp,
                 mainlabel={fig:ch4_plot_rec_error_pred_obs},
			           maincaption={Errors, predicted and observed, due to the dimension reduction procedure (\gls[hyper=false]{pca}) and the functional approximation (\gls[hyper=false]{gp}) for the three types of output.},
			           mainshortcaption={Errors, predicted and observed, due to the dimension reduction procedure and the functional approximation for the three types of output.},%
			           leftopt={width=0.31\textwidth},
			           leftlabel={fig:ch4_plot_rec_error_pred_obs_tc},
			           leftcaption={Clad temperature ($TC$) output},
			           midopt={width=0.31\textwidth},
			           midlabel={fig:ch4_plot_rec_error_pred_obs_dp},
			           midcaption={Pressure drop ($DP$) output},
			           rightopt={width=0.31\textwidth},
			           rightlabel={fig:ch4_plot_rec_error_pred_obs_cp},
			           rightcaption={Liquid carryover ($CO$) output},
			           spacing={},
			           spacingtwo={}]
{../figures/chapter4/figures/plotRecErrorPredObsTC}
{../figures/chapter4/figures/plotRecErrorPredObsDP}
{../figures/chapter4/figures/plotRecErrorPredObsCO}

The extend of the $x$-axis signifies the range of error due to the \gls[hyper=false]{pc} truncation.
The farther a data point is from the left, the larger the error is due to the dimension reduction.
On the other hand, the extend of the $y$-axis, specifically the vertical distance between the data points and the line,
signifies the error due to the misprediction of the standardized \glspl[hyper=false]{pc} scores by the \gls[hyper=false]{gp} metamodel.
Data points which are located along the line implied a perfect prediction by the \gls[hyper=false]{gp} metamodel.
The farther a data point is from the line, the larger the error is due to the metamodel approximation.
As can be seen, no data point is located below the line as the truncation error sets the limit of the metamodel predictive performance.
Furthermore, though some data points (i.e., realizations) might be mispredicted and lie over a wide range of value,
the cloud of the data points is only concentrated around a particular range of value.
Table~\ref{tab:ch4_gp_testing} numerically summarizes the results of the testing step.
For comparison the standard deviation of the testing data set for each output is also given.  


\begin{table*}[!bhtp]\centering
\ra{0.9}
\caption{Predictive performance of the selected \gls[hyper=false]{gp} \gls[hyper=false]{pc} metamodel on the testing dataset of size $5'000$}
\label{tab:ch4_gp_testing}
\begin{tabular*}{\textwidth}{@{}rrrrrrrr@{}}\toprule
\multirow{2}{*}{\scriptsize{Output}}&\multirow{2}{*}{\scriptsize{\gls[hyper=false]{pc}\textsubscript{max}}}&\multicolumn{2}{c}{\scriptsize{Predictivity Coefficient}}&\phantom{a}&\multicolumn{2}{c}{\scriptsize{Reconstruction Error}}&\scriptsize{Test Data}\\             
                                                                              \cmidrule{3-4}                                       \cmidrule{6-7}
																			&                                 & \scriptsize{$Q_2$ \gls[hyper=false]{pc}$_1$}&\scriptsize{$Q_2$ \gls[hyper=false]{pc}\textsubscript{max}}&&\scriptsize{\gls[hyper=false]{rmse}\textsubscript{obs}}&\scriptsize{\gls[hyper=false]{rmse}\textsubscript{pred}}&\scriptsize{Std. Dev.}\\ \midrule
\footnotesize{$TC$} & \footnotesize{$7$}  & \footnotesize{$\approx 1.0$}  & \footnotesize{$0.77$} && \footnotesize{$20.17 \, [K] $} & \footnotesize{$22.43 \, [K] $} & \footnotesize{$ 254.0 \, [K] $} \\
\footnotesize{$DP$} & \footnotesize{$10$} & \footnotesize{$\approx 1.0$}  & \footnotesize{$0.74$} && \footnotesize{$55.57 \, [Pa]$} & \footnotesize{$77.95 \, [Pa]$} & \footnotesize{$9200.0 \, [Pa]$} \\
\footnotesize{$CO$} & \footnotesize{$5$}  & \footnotesize{$\approx 1.0$}  & \footnotesize{$0.77$} && \footnotesize{$ 0.16 \, [kg]$} & \footnotesize{$ 0.27 \, [kg]$} & \footnotesize{$  30.4 \, [kg]$} \\
\bottomrule
\end{tabular*}
\end{table*}
