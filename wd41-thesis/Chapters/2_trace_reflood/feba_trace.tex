\section{FEBA Model in TRACE}\label{sec:reflood_feba_trace}

The \textsc{FEBA} facility was modeled using the TH system code \textsc{TRACE}.
The model consists of a one-dimensional \textsc{VESSEL} component (to model the bundle test section),
a \textsc{PIPE} component (upper plenum of the test section), 
a \textsc{FILL} component (inlet flow and inlet temperature boundary conditions),
a \textsc{BREAK} component (outlet pressure boundary condition),
two \textsc{HTSTR} components (heater rods simulator and non-powered test section housing),
and a \textsc{POWER} components (imposed electrical boundary condition).

The \textsc{VESSEL} component was nodalized into $28$ hydraulic nodes of varying sizes between $60$ and $315$ $[mm]$.
Both \textsc{HTSTR} components were also nodalized into the sample number of coarse axial conduction nodes.
However, since a large axial temperature gradient is expected in a reflood transient,
the fine-mesh reflood flag in \textsc{TRACE} was enabled.
As a result, each of the course conduction nodes is divided uniformly by $5$,
yielding a total of $142$ axial conduction nodes.
The main geometrical parameters and experimental conditions used to develop the \textsc{FEBA} input model
are summarized in Table~\ref{tab:feba_trace}.

\begin{table}[ht]
    \myfloatalign
    \caption[]{Geometrical parameters}  \label{tab:feba_trace}
    \begin{tabularx}{\textwidth}{Xcc} \toprule
        \tableheadline{Parameter}		& \tableheadline{Unit} & \tableheadline{Value} \\ \midrule
        Test section total length 		& $[m]$		& $4.114$ \\
        Total heated length 			& $[m]$		& $3.9$ \\
        Flow area						& $[m^2]$	& $3.901 \times 10^{-3}$\\
        Hydraulic diameter				& $[mm]$	& $3.901 \times 10^{-3}$\\
        Rectangular housing width   	& $[mm]$	& $3.901 \times 10^{-3}$\\
        Rectangular housing thickness	& $[mm]$	& $3.901 \times 10^{-3}$\\
        Number of rods					& $[-]$		& $3.901 \times 10^{-3}$\\
        Rod outer diameter				& $[mm]$	& $3.901 \times 10^{-3}$\\
        Pitch-to-Diameter ratio			& $[-]$		& $3.901 \times 10^{-3}$\\
        Number of spacer grids			& $[-]$		& $3.901 \times 10^{-3}$\\
        \bottomrule
    \end{tabularx}
\end{table}

\subsection{Base Case Simulation Results}

\subsection{Prior Uncertainty Propagation}
