%******************************************************************************************
\section{Initial Selection of Input Parameters}\label{sec:reflood_initial_inputs_selection}
%******************************************************************************************

% Introductory paragraph
This section presents the selection process of the initial set of uncertain input parameters of the \gls[hyper=false]{feba} simulation in \gls[hyper=false]{trace}.
Afterward, the assignment of the initial (prior) uncertainties of these parameters are presented. 
This part is closely related to PSI participation in the \gls[hyper=false]{premium} benchmark thus several reference are made to activities related to that benchmark \cite{Zerkak2016}.

%---------------------------------------------------------------------------------
\subsection{Selection of Input Parameters}\label{sub:reflood_parameters_selection}
%---------------------------------------------------------------------------------

% Introductory paragraph
The selection process for the uncertain input parameters to consider differs depending on the type of parameter.
Each of the selected parameters can broadly fall into one of the two following categories:
\begin{itemize}
	\item Input parameters that are not specific to the \gls[hyper=false]{trace} code (e.g., initial and boundary conditions, material thermo-physical properties).
		This category of parameters is often referred to as the \emph{controllable inputs} of the simulation.
	\item Input parameters that are specific to \gls[hyper=false]{trace} code (e.g., implementation of the two-phase momentum and heat transfer package for reflood condition).
		This category is often referred to as the \emph{model parameters} of the simulation.
\end{itemize}

% Selection on the first category
The selection of parameters belonging to the controllable inputs 
\marginpar{Controllable inputs selection}
category simply corresponds to the parameters recommended by the benchmark organizers and employed by most participants \cite{Kovtonyuk2015}.
The $13$ selected parameters of this category are listed in Table~\ref{tab:trace_model_parameter_1}.

% Selection on the second category
On the other hand, the selection of the model parameters specific to the \gls[hyper=false]{trace} code is challenging due to the fact that \gls[hyper=false]{trace} is a relatively recent code (in comparison with codes like RELAP5, ATHLET, or CATHARE).
In essence, the code has been developed from different variants of the TRAC codes for different reactor types (TRAC-BF1, TRAC-P) to result in a single consolidated code applicable to both \gls[hyper=false]{pwr} and \gls[hyper=false]{bwr}.
Contributing to that difficulty is the fact that \gls[hyper=false]{trace} is currently undergoing significant developments and improvements, including modifications to the two-phase closure models for momentum and heat transfers.
Consequently, the tasks of selecting the model parameters and later their prior uncertainties are more difficult than for a more established codes.

% Criteria for choosing the parameters
To overcome this issue,
\marginpar{Model parameters selection}
the following principles have been followed to select the model parameters:
\begin{enumerate}
	\item The selection has been focused on the physical models in the post-\glsentryshort{chf} package of the \gls[hyper=false]{trace} code (including the reflood models).
		    Specifically, these are models for the \gls[hyper=false]{iafb} and \gls[hyper=false]{dffb} flow regimes \cite{USNRC2012}.
	\item Models related to spacer grid are also included as they are known to have a significant impact on reflooding \cite{Miller2013}.
	\item Parameters related to the minimum film boiling temperature and transition boiling should be selected, since they have (by model construction) an impact on the time of quenching.
\end{enumerate}

\clearpage
\begin{sidewaystable}

\caption{Selected \gls[hyper=false]{trace} input parameters (controllable inputs), their perturbation factors and their range of variations.}
\label{tab:trace_model_parameter_1}
\centering
\newcolumntype{Y}{>{\RaggedRight\arraybackslash}X}
%\begin{tabularx}{\textwidth}{@{}c|>{$}c<{$}|>{$}c<{$}|Y|Y|Y|Y|Y@{}}
\begin{tabularx}{0.985\textwidth}{@{}cccc>{$}c<{$}>{$}c<{$}c@{}}
\toprule
No.	& Parameter 				& Description 										& Distribution & \text{Range of}  & \text{Nominal} & Mode of \\
		& ID        				&                             		&              & \text{Variation} & \text{Value}   & Perturbation \\
\midrule
1  	& \texttt{breakP} 	& Outlet pressure 								& Uniform 	& [0.90, 1.10]   	& 1.0 			& Multiplicative \\ 
2  	& \texttt{fillT} 		& Inlet water temperature 				& Uniform 	& [-5.00, +5.00] 	& 0.0\,[K] 	& Additive \\ 
3  	& \texttt{fillV}		& Inlet water velocity          	& Uniform 	& [0.90, 1.10]   	& 1.0 			& Multiplicative \\ 
4  	& \texttt{pwr} 			& Heater rod power             	 	& Uniform 	& [0.90, 1.05]   	& 1.0 			& Multiplicative \\ 
\midrule
5  	& \texttt{nicK} 		& Conductivity (Nichrome) 				& Uniform 	& [0.95, 1.05] 		& 1.0 			& Multiplicative \\ 
6  	& \texttt{nicCP} 		& Specific heat (Nichrome)				& Uniform 	& [0.95, 1.05] 		& 1.0 			& Multiplicative \\ 
7  	& \texttt{nicEM} 		& Emissivity (Nichrome) 					& Uniform 	& [0.90, 1.00] 		& 0.95			& Substitutive \\ 
8  	& \texttt{mgoK} 		& Conductivity (MgO)							& Uniform 	& [0.80, 1.20] 		& 1.0 			& Multiplicative \\
9  	& \texttt{mgoCP}		& Specific heat (MgO) 						& Uniform 	& [0.80, 1.20] 		& 1.0 			& Multiplicative \\ 
10 	& \texttt{vesEps}		& Wall roughness 									& Uniform 	& [6.10\times 10^{-7}, 2.44\times 10^{-6}] & 1.5 \times 10^{-6} \, [m] & Substitutive \\ 
11 	& \texttt{ssK} 			& Conductivity (stainless steel)	& Uniform 	& [0.95, 1.05] 		& 1.0 			& Multiplicative \\ 
12 	& \texttt{ssCP} 		& Specific heat (stainless steel)	& Uniform		& [0.95, 1.05] 		& 1.0 			& Multiplicative \\ 
13 	& \texttt{ssEM} 		& Emissivity (stainless steel)		& Uniform 	& [0.56, 0.94] 		& 0.84 			& Substitutive \\ 
\bottomrule
\end{tabularx}
\end{sidewaystable}
\clearpage

% Perturbation factor, high-level perturbation
Additionally, as a common principle, the selected models and their parameters are perturbed by means of perturbation factors (detailed below) at the highest-possible level of the structure of these models.
Different codes share similarity in representing major flow regimes (high-level) but might differ in the constituents models (i.e., sub-models) for each flow regime (lower-level).
Focusing on the higher-level implementation of the models allows, to some extent, to use reference uncertainty information obtained from codes other than \gls[hyper=false]{trace}.

% Resulting List of Model Parameters 1
In accordance with the first selection principle above, a set of $10$ high-level parameters has been selected (five for each flow regime).
Specifically, for each flow regime: the wall-fluid \gls[hyper=false]{htc}, the liquid-inter\-face \gls[hyper=false]{htc}, the vapor-interface \gls[hyper=false]{htc}, the wall-fluid drag coefficient, and the interfacial drag coefficient.

% Resulting List of Model Parameters 2
Following the second principle, two additional parameters have been selected:
the spacer grid pressure loss coefficient model from Yao, Loftus, and Hochreiter as well as the grid convective heat transfer enhancement model from Yao, Hochreiter, and Leech (see \cite{USNRC2012} pp. $425$--$429$ and \cite{Yao1982}).
These perturbations on the parameters are applied to all seven spacer grids at once.

% Resulting List of Model Parameters 3
Lastly, from the third principle, the quench temperature parameter in \gls[hyper=false]{trace} and wall-fluid \gls[hyper=false]{htc} for transition boiling (see \cite{USNRC2012} pp. $293$--$299$) have been added to the list of uncertain input parameters.

In the end, $14$ parameters are selected and are summarized in Table~\ref{tab:trace_model_parameter_2}, yielding a total number of $27$ uncertain input parameters.

% Table of Model Parameters
\begin{sidewaystable}

\caption{Selected \gls[hyper=false]{trace} input parameters (model parameters), their perturbation factors and their range of variations}
\label{tab:trace_model_parameter_2}

\centering
\newcolumntype{Y}{>{\RaggedRight\arraybackslash}X}
%\begin{tabularx}{\textwidth}{@{}c|>{$}c<{$}|>{$}c<{$}|Y|Y|Y|Y|Y@{}}
\begin{tabularx}{0.90\textwidth}{@{}cccc>{$}c<{$}>{$}c<{$}c@{}}
\toprule
No. & Parameter 							& Description & Distribution & \text{Range of}  & \text{Nominal} & Mode of \\
    & ID        							&             &              & \text{Variation} & \text{Value}   & Perturbation \\
\midrule
14 	& \texttt{gridK} 					& Spacer grid $\Delta p$ coefficient															& Uniform 				& [0.25, 1.75] 		& 1.0 				& Multiplicative \\ 
15  & \texttt{gridHT} 				& Spacer grid \gls[hyper=false]{htc} enhancement									& Log-Uniform 		& [0.50, 2.00] 		& 1.0				 	& Multiplicative \\ 
16  & \texttt{iafbWHT}  			& Wall \gls[hyper=false]{htc} (\glsentryshort{iafb})							& Log-Uniform 		& [0.50, 2.00]   	& 1.0 				& Multiplicative \\ 
17  & \texttt{dffbWHT}  			& Wall \gls[hyper=false]{htc} (\glsentryshort{dffb})  						& Log-Uniform 		& [0.50, 2.00] 	  & 0.0 				& Multiplicative \\ 
18  & \texttt{iafbLIHT}				& Liquid-interface \gls[hyper=false]{htc} (\glsentryshort{iafb})  & Log-Uniform 		& [0.25, 4.00]   	& 1.0 				& Multiplicative \\ 
19  & \texttt{iafbVIHT} 			& Vapor-interface \gls[hyper=false]{htc}  (\glsentryshort{iafb})  & Log-Uniform 		& [0.25, 4.00]   	& 1.0 				& Multiplicative \\ 
20  & \texttt{dffbLIHT} 			& Liquid-interface \gls[hyper=false]{htc} (\glsentryshort{dffb}) 	& Log-Uniform 		& [0.25, 4.00] 	 	& 1.0 				& Multiplicative \\ 
21  & \texttt{dffbVIHT} 			& Vapor-interface \gls[hyper=false]{htc} (\glsentryshort{dffb})	 	& Log-Uniform 		& [0.25, 4.00] 	 	& 1.0 				& Multiplicative \\ 
22  & \texttt{iafbIntD} 			& Interfacial drag (\glsentryshort{iafb}) 												& Log-Uniform 		& [0.25, 4.00] 	 	& 1.0 				& Multiplicative \\ 
23  & \texttt{dffbIntDr} 			& Interfacial drag (\glsentryshort{dffb})													& Log-Uniform 		& [0.25, 4.00]		& 1.0 				& Multiplicative \\
24  & \texttt{iafbWallDr} 		& Wall drag (\glsentryshort{iafb}) 																& Log-Uniform 		& [0.50, 2.00] 		& 1.0 				& Multiplicative \\ 
25	& \texttt{dffbWallDr} 		& Wall drag (\glsentryshort{dffb})			  												& Log-Uniform 		& [0.50, 2.00] 		& 1.0 				& Multiplicative \\ 
26 	& \texttt{transHTCWallSV}	& Wall \gls[hyper=false]{htc} (Transition boiling)								& Log-uniform			& [0.50, 2.00] 		& 1.0 				& Multiplicative \\ 
27 	& \texttt{tQuench} 				& Quenching temperature $[K]$																			& Uniform 				& [-50.0, +50.0] 	& 0.0 \, [K]	& Additive \\ 
\bottomrule

\end{tabularx}
\end{sidewaystable}

%----------------------------------------------------------------------------------
\subsection{Perturbation Factors}\label{sub:reflood_parameters_perturbation_factor}
%----------------------------------------------------------------------------------

% Opening paragraph
The nominal values of the selected input parameters of the \gls[hyper=false]{trace} \gls[hyper=false]{feba} model are varied by means of perturbation factors.
\marginpar{Perturbation factor}
These perturbation factors are modeled as random variables following a predefined \gls[hyper=false]{pdf} detailed in the next section, from which a set of samples of input parameters values can be generated.

% Mode of perturbations
For a given sampled perturbation factor, one of three modes of perturbation is possible: \emph{additive}, \emph{multiplicative,} and \emph{substitutive}.
\marginpar{Modes of perturbation}
In the additive mode, the sampled perturbation factor is added to the nominal parameter value of the \gls[hyper=false]{trace} model.
In the multiplicative mode, the sampled perturbation factor is multiplied by the nominal parameter value.
Finally, in the substitutive mode, the sampled perturbation factor directly substitutes for the nominal parameter value.
The mode of perturbation for each selected input parameter are listed in last column of Table~\ref{tab:trace_model_parameter_1} and Table~\ref{tab:trace_model_parameter_2}.

% trace-simexp
A tool is developed in the Python programming language to assist in automatically pre-processing, executing, and post-processing 
\marginpar{\texttt{trace-simexp}}
numerous \gls[hyper=false]{trace} simulations of the \gls[hyper=false]{feba} model based on a set of sampled input parameters values.
The tool, \texttt{trace-simexp}, is detailed in Appendix~\ref{app:trace_simexp}.

%-----------------------------------------------------------------------------------
\subsection{Prior Uncertainty Quantification}\label{sub:reflood_parameters_prior_uq}
%-----------------------------------------------------------------------------------

% Opening Paragraph
The uncertainties associated with the controllable inputs were taken directly from the recommended value of the \gls[hyper=false]{premium} benchmark and the list can be found in Table~\ref{tab:trace_model_parameter_1}.
As for the model parameters, the uncertainty ranges has been determined following the available literature on uncertainties in physical models for \gls[hyper=false]{lbloca}.

% Main Source of Information
The main sources of information consisted of Ref.\cite{Wickett1991} and Ref.~\cite{Glaeser2008a}, which included uncertainty information for the closure models of the system codes TRAC-PF1/MOD1 and ATHLET-Mod2.1 (Cycle B), respectively. 
Furthermore, prior experience and knowledge of the closure model uncertainties of the CATHARE2 code (\texttt{V1.3L\_1}, Rev.5) have been used \cite{Zerkak2016,IPSN2001}.
Ref.~\cite{IPSN2001} accounts for an analysis by IPSN\footnote{Institut de protection et de s\^{u}ret\'e nucl\'eaire (IPSN)} of the uncertainty quantification method ``M\'ethode D\'eterministe R\'ealiste'' for the \gls[hyper=false]{pwr} \gls[hyper=false]{lbloca} which was proposed by EDF\footnote{Electricit\'e de France} and was evaluated in $2000$.
The high-level implementation of the perturbation factors for the uncertainty analysis in the post-\gls[hyper=false]{chf} closure models, allowed information from different codes to be extracted for the initial estimate.
This approach was deemed adequate in the context of the determination of the prior \glspl[hyper=false]{pdf}.

% Uniform and Log-Uniform
To simplify the quantification of the prior uncertainties further, the \glspl[hyper=false]{pdf} of the multiplication factors were assumed to follow symmetric bounded uniform and log-uniform distributions with the nominal parameter value equals to the median value.
For the log-uniform distribution the form $[2^{-n}, 2^{n}]$ was assumed, where $n$ is an integer.
All model parameters that were a priori deemed to be important were assumed to follow log-uniform distributions.

The ranges of the parameters (i.e., their minimum and maximum), were chosen to cover range of similar parameters available in Refs.~\cite{Wickett1991,Glaeser2008a}.
Though this at times resulted in the selection of large bounds, they were deemed acceptable following a verification study against the nominal predictions.
The verification heavily relied on engineering judgment via visual inspection of the width of the prediction uncertainty bands to decide if such bands were indeed reasonable\footnote{The loosely defined notion of ``reasonable'' in this case is similar to the notion of ``'behavioral vs. non-behavioral'' prediction in hydrology modeling \cite{Beven2009}.}.
The approach is admittedly imprecise, but is intentional as it avoids underestimating the prior uncertainty range of influential model parameters. 

%
Table~\ref{tab:trace_model_parameter_2} lists the results of the prior uncertainty quantification of the selected model parameters.
Note that all $27$ input parameters considered are a priori independent.
