\section{Thermal-Hydraulics System Code \glsentryshort{trace}}\label{sec:reflood_trace}

\gls[hyper=false]{trace} is the best-estimate system \gls[hyper=false]{th} code developed by the \gls[hyper=false]{usnrc} 
as a tool for light water reactor transient analysis during normal and accident scenarios.
Its development is an on-going effort 
to modernize into a single software package all previous \gls[hyper=false]{usnrc} \gls[hyper=false]{th} codes
that were developed separately for specific reactor types and/or applications.
This ultimately would make the code more versatile for end users and more efficient to maintain for the developer.

\subsection{Governing Equations}\label{sub:governing_equations}

The hydraulic module of \gls[hyper=false]{trace} is based on a two-fluid six-equation model, 
solving the conservation equations of mass, momentum, and energy for the liquid and vapor phases in the coolant.
The formulations are both averaged in time and volume to filter out short-term and short scale variations in the flow.
Furthermore, the formulations are given in volumetric term (i.e., per unit volume basis) with respect to a defined control volume (or \emph{node}).
A symbol is defined the first time it appears in an equation 
and the complete list of symbols can be found in the \hyperref[app:notations]{Glossary of Notations} at back of the thesis.

\paragraph{Mass balance equations, liquid and gas phases:}
\begin{equation}
	\frac{\partial [(1-\alpha)\rho_l]}{\partial t} + \nabla \cdot [(1-\alpha) \rho_l \mathbf{v_l}] = - \Gamma
\label{eq:mass_balance_liquid}
\end{equation}
\begin{equation}
	\frac{\partial [\alpha \rho_g]}{\partial t} + \nabla \cdot [\alpha \rho_g \mathbf{v_g}] = \Gamma
\label{eq:mass_balance_gas}
\end{equation}
where the subscripts indicate the phase, $l$ for the liquid phase and $g$ for the gas phase (vapor); 
\gls[hyper=false]{alpha} is the void fraction; 
\gls[hyper=false]{rhol} (\gls[hyper=false]{rhog}) is the mass density of the liquid (gas) phase;
and \gls[hyper=false]{vl} (\gls[hyper=false]{vg}) is the flow velocity of the liquid (gas) phase.
The terms in either sides of the two mass balance equations are explained in Table~\ref{tab:mass_balance}.

\begin{table}[ht]
	\myfloatalign
	\caption[The terms in \glsentryshort{trace} two-fluid model mass balance equations]{The terms in \glsentryshort{trace} two-fluid model mass balance equations (all are given in volumetric term)}
	\label{tab:mass_balance}
	\begin{tabularx}{\textwidth}{Xcc} 
		\toprule
		\tableheadline{Terms} & \tableheadline{Liquid Phase} & \tableheadline{Gas Phase} \\ 
		\midrule
		\footnotesize{mass rate of change}  					 & $\frac{\partial [(1-\alpha)\rho_l]}{\partial t}$  & $\frac{\partial [\alpha \rho_g]}{\partial t}$ \\
		\footnotesize{mass convection rate} 					 & $\nabla \cdot [(1-\alpha) \rho_l \mathbf{v_l}]$   & $\nabla \cdot [\alpha \rho_g \mathbf{v_g}]$ \\
		\footnotesize{interfacial mass-transfer rate}  & $- \Gamma$                                        & $\Gamma$ \\
		\bottomrule
	\end{tabularx}
\end{table}

Note that the term \gls[hyper=false]{Gamma}, the volumetric interfacial mass-transfer rate,
is given with a convention that it is positive for the transfer from liquid phase to gas phase.
This term is defined in Eq.~(\ref{eq:Gamma}) below.

\paragraph{Momentum balance equations, liquid and gas phases:}
\begin{equation}
	\begin{split}
		& \frac{\partial [(1-\alpha)\rho_l \mathbf{v}_l]}{\partial t} + \nabla \cdot [(1-\alpha) \rho_l \mathbf{v_l} \otimes \mathbf{v_l}] + (1 - \alpha) \nabla p \\
		& \quad = \mathbf{f}_i + \mathbf{f}_{wl} + (1 - \alpha) \rho_l \mathbf{g} - \Gamma \mathbf{v}_i
	\end{split}
\label{eq:momentum_balance_liquid}
\end{equation}
\begin{equation}
	\begin{split}
		& \frac{\partial [\alpha \rho_g \mathbf{v}_g]}{\partial t} + \nabla \cdot [\alpha \rho_g \mathbf{v_g} \otimes \mathbf{v_g}] + \alpha \nabla p \\
		& \quad = - \mathbf{f}_i + \mathbf{f}_{wg} + \alpha \rho_g \mathbf{g} + \Gamma \mathbf{v}_i
	\end{split}
\label{eq:momentum_balance_gas}
\end{equation}
where $\nabla p$ is the pressure gradient;
\gls[hyper=false]{fi} is the volumetric force due to shear at the phase interface;
\gls[hyper=false]{fwl} is the volumetric force acting on the liquid phase due to shear at the wall (i.e., fluid-structure contact);
\gls[hyper=false]{fwg} is the volumetric force acting on the gas phase due to shear at the wall;
\gls[hyper=false]{gravity} is the gravitational acceleration;
and \gls[hyper=false]{vinterface} is the flow velocity at the phase interface.
Table~\ref{tab:momentum_balance} lists the terms in either sides of the two momentum balance equations.

\begin{table}[ht]
	\myfloatalign
	\caption[The terms in \glsentryshort{trace} two-fluid model momentum balance equations]{The terms in \glsentryshort{trace} two-fluid model momentum balance equations (all are given in volumetric term)}
	\label{tab:momentum_balance}
	\begin{tabularx}{\textwidth}{>{\raggedright}Xcc} 
		\toprule
		\tableheadline{Terms} & \tableheadline{Liquid Phase} & \tableheadline{Gas Phase} \\ 
		\midrule
		\footnotesize{momentum rate of change}  					 	& $\frac{\partial [(1-\alpha)\rho_l \mathbf{v}_l]}{\partial t}$  				& $\frac{\partial [\alpha \rho_g \mathbf{v}_g]}{\partial t}$ \\
		\footnotesize{momentum convection rate} 					 	& $\nabla \cdot [(1-\alpha) \rho_l \mathbf{v_l} \otimes \mathbf{v_l}]$ 	& $\nabla \cdot [\alpha \rho_g \mathbf{v_g} \otimes \mathbf{v_g}]$ \\
		\footnotesize{pressure gradient}                  	& $(1 - \alpha) \nabla p$                                        				& $\alpha \nabla p$ \\
		\midrule
		\footnotesize{momentum change due to:}            	&                                                                 			& \\
		\footnotesize{interfacial friction} 								& $\mathbf{f}_i$ 																												&  $- \mathbf{f}_i$ \\
		\footnotesize{wall friction} 						  					& $\mathbf{f}_{wl}$ 																										& $\mathbf{f}_{wg}$ \\
		\footnotesize{body force} 													& $(1 - \alpha) \rho_l \mathbf{g}$ 																			& $\alpha \rho_g \mathbf{g}$ \\
		\footnotesize{interfacial mass-transfer} 	  				& $- \Gamma \mathbf{v}_i$ 																							& $\Gamma \mathbf{v}_i$ \\
		\bottomrule
	\end{tabularx}
\end{table}

Note that the formulation in \gls[hyper=false]{trace} uses the simplifying assumption of $P_i = P_g = P_l$.
That is, the pressure in a given control volume is the same in either phases as well as at the interface \cite{USNRC2012}.

For the friction (shear) terms in right hand side, TRACE uses the following formulations,
\begin{equation}
	\mathbf{f}_i = C_i (\mathbf{v}_g - \mathbf{v}_l) |\mathbf{v}_g - \mathbf{v}_l| 
\label{eq:fi}
\end{equation}
\begin{equation}
	\mathbf{f}_{wl} = - C_{wl} \mathbf{v}_l |\mathbf{v}_l|
\label{eq:fwl}
\end{equation}
\begin{equation}
	\mathbf{f}_{wg} = - C_{wg} \mathbf{v}_g |\mathbf{v}_g|
\label{eq:fwg}
\end{equation}
where the friction coefficients \gls[hyper=false]{cint}, \gls[hyper=false]{cwl}, \gls[hyper=false]{cwg} for interfacial shear, wall-liquid shear, and wall-gas shear, respectively 
are obtained from flow regime-dependent empirical correlations.

\paragraph{Energy balance equations, liquid and gas phases:}
\begin{equation}
	\begin{split}
		& \frac{\partial [(1-\alpha)\rho_l(e_l + |\mathbf{v_l}|^2/2]}{\partial t} + \nabla \cdot \left[(1-\alpha) \rho_l \left(e_l+\frac{P}{\rho_l}+\frac{|\mathbf{v_l}|^2}{2}\right) \mathbf{v_l} \right] \\
		&	\quad = q_{il} + q_{wl} + q_{wsat} + q_{dl} + (1 - \alpha) \rho_l \mathbf{g} \cdot \mathbf{v}_l \\
		& \qquad - \Gamma h^\prime_l + (\mathbf{f}_{i} + \mathbf{f}_{wl}) \cdot \mathbf{v}_l
	\end{split}
\label{eq:energy_balance_liquid}
\end{equation}
\begin{equation}
	\begin{split}
		 & \frac{\partial [\alpha \rho_g (e_g + |\mathbf{v_g}|^2/2]}{\partial t} + \nabla \cdot \left[\alpha \rho_g \left(e_g+\frac{P}{\rho_g}+\frac{|\mathbf{v_g}|^2}{2}\right) \mathbf{v_g} \right] \\
		 & \quad  = q_{ig} + q_{wg} + q_{dg} + \alpha \rho_g \mathbf{g} \cdot \mathbf{v}_g + \Gamma h^\prime_g + (-\mathbf{f}_{i} + \mathbf{f}_{wg}) \cdot \mathbf{v}_g
	\end{split}
\label{eq:energy_balance_gas}
\end{equation}
where \gls[hyper=false]{el} (\gls[hyper=false]{eg}) is the liquid (gas) phase internal energy;
\gls[hyper=false]{qil} (\gls[hyper=false]{qig}) is the volumetric interfacial heat transfer on the liquid (gas) phase;
\gls[hyper=false]{qwl} (\gls[hyper=false]{qwg}) is the volumetric wall (sensible) heat transfer on the liquid (gas) phase;
\gls[hyper=false]{qwsat} is the volumetric wall (latent) heat transfer on the liquid phase;
\gls[hyper=false]{qdl} (\gls[hyper=false]{qdg}) is the volumetric direct power deposition on the liquid (gas) phase;
\gls[hyper=false]{hlp} is the bulk liquid enthalpy;
and \gls[hyper=false]{hgp} is the gas phase saturation enthalpy.
Table~\ref{tab:energy_balance} lists all of the terms in either sides of the two energy balance equations.

\begin{table}[ht]
	\myfloatalign
	\caption[The terms in \glsentryshort{trace} two-fluid model energy balance equations]{The terms in \glsentryshort{trace} two-fluid model momentum energy equations (all are given in volumetric term)}
	\label{tab:energy_balance}
	\begin{tabularx}{\textwidth}{>{\raggedright}Xcc} 
		\toprule
		\tableheadline{Terms} & \tableheadline{Liquid Phase} & \tableheadline{Gas Phase} \\ 
		\midrule
		\footnotesize{energy rate of change}  						& $\frac{\partial [(1-\alpha)\rho_l(e_l + |\mathbf{v_l}|^2/2]}{\partial t}$  																											& $\frac{\partial [\alpha \rho_g (e_g + |\mathbf{v_g}|^2/2]}{\partial t}$ \\
		\footnotesize{energy convection rate} 						& \tiny{$\nabla \cdot \left[(1-\alpha) \rho_l \left(e_l+\frac{P}{\rho_l}+\frac{|\mathbf{v_l}|^2}{2}\right) \mathbf{v_l} \right]$} & \tiny{$\nabla \cdot \left[\alpha \rho_g \left(e_g+\frac{P}{\rho_g}+\frac{|\mathbf{v_g}|^2}{2}\right) \mathbf{v_g} \right]$} \\
		\midrule
		\footnotesize{(sensible) interfacial heat transfer} 	& \footnotesize{$q_{il}$}																									& \footnotesize{$- \mathbf{f}_i$} \\
		\footnotesize{(sensible) wall heat transfer} 					& \footnotesize{$q_{wl}$}																									& \footnotesize{$\mathbf{f}_{wg}$} \\
		\footnotesize{(latent) wall heat transfer} 						& \footnotesize{$q_{w\text{sat}}$} 																				& \footnotesize{$\alpha \rho_g \mathbf{g}$} \\
		\footnotesize{direct heat deposition} 								& \footnotesize{$q_{dl}$}																									& \footnotesize{$\Gamma \mathbf{v}_i$} \\
		\footnotesize{energy loss (gain) due to:} 						&                                                                         & \\
		\footnotesize{gravity}      													& \footnotesize{$(1 - \alpha) \rho_l \mathbf{g} \cdot \mathbf{v}_l$} 			& \footnotesize{$\alpha \rho_l \mathbf{g} \cdot \mathbf{v}_l$} \\
		\footnotesize{phase change} 													& \footnotesize{$- \Gamma h^\prime_l$} 																		& \footnotesize{$\Gamma h^\prime_g$} \\
		\footnotesize{wall and interfacial friction} 					& \footnotesize{$(\mathbf{f}_{i} + \mathbf{f}_{wl}) \cdot \mathbf{v}_l$} 	& \footnotesize{$(-\mathbf{f}_{i} + \mathbf{f}_{wg}) \cdot \mathbf{v}_g$}\\
		\bottomrule
	\end{tabularx}
\end{table}

The heat transfer terms between the wall and the phases follow Newton's law of cooling,
\begin{equation}
	q_{wl} = h_{wl} \, a_{wl} \, (T_w - T_l)
\label{eq:qwl}
\end{equation}
\begin{equation}
	q_{wg} = h_{wg} \, a_{wg} \, (T_w - T_g)
\label{eq:qwg}
\end{equation}
\begin{equation}
	q_{w\text{sat}} = h_{w\text{sat}} \, a_{wl} \, (T_w - T_\text{sat})
\label{eq:qwsat}
\end{equation}
where \gls[hyper=false]{tw}, \gls[hyper=false]{tl}, \gls[hyper=false]{tg}, \gls[hyper=false]{tsat} are the wall, liquid phase, liquid phase, and liquid saturation temperatures, respectively;
\gls[hyper=false]{awl} (\gls[hyper=false]{awg}) is the volumetric surface contact area between the wall and liquid (gas) phase;
and \gls[hyper=false]{hwl}, \gls[hyper=false]{hwg}, and \gls[hyper=false]{hwsat} are the \glspl[hyper=false]{htc} between wall and liquid, wall and gas, and wall-saturated liquid, respectively.
The volumetric surface contact area as well as the heat transfer coefficients are obtained from a set of flow regime-dependent empirical correlations.

Additionally, the heat transfer terms at the interface between the two phases are also modeled using the same law,
\begin{equation}
	q_{il} = h_{il} \, a_{i} \, (T_{sg} - T_l)
\label{eq:qil}
\end{equation}
\begin{equation}
	q_{ig} = \frac{p_g}{p} \, h_{ig} \, a_{i} \, (T_{sg} - T_g)
\label{eq:qig}
\end{equation}
where \gls[hyper=false]{hil} (\gls[hyper=false]{hig}) is the \gls[hyper=false]{htc} for liquid (gas) phase at the interface;
\gls[hyper=false]{ai} is the volumetric interfacial surface area;
\gls[hyper=false]{pg} is the partial pressure of the gas phase;
and \gls[hyper=false]{tsg} is the saturation temperature corresponding to partial pressure of the gas phase.

Finally, the mass-transfer rate at the interface is defined using a thermal-energy jump condition that results in
\begin{equation}
	\Gamma = \frac{-(q_{ig} + q_{il}) + q_{w\text{sat}}}{(h^\prime_g - h^\prime_l)}
\label{eq:Gamma}
\end{equation}
In other words, the net heat transfer rate given to the saturated liquid phase, is used entirely for phase change.

\paragraph{}

In each of the balance equations given above, 
the right hand side represents the source and sink terms mainly due to fluid interaction with solid structure (\emph{wall})
and the interaction between the phases, among others.
The set of balance equations characterizes the two-phase flow inside a control volume in a time- and volume-averaged manner exactly provided that the terms in the right hand side of the equation (such as, Eqs.~(\ref{eq:fi})-(\ref{eq:fwg}) and Eqs.~(\ref{eq:qwl})-(\ref{eq:qig})) are correct.

Besides the set balance equations that govern the two-phase fluid flow,
\gls[hyper=false]{trace} also includes a heat conduction module (known as \emph{heat structure} component) 
to model correctly the heat transfer process in solid structures (e.g., active fuel, internal passive structures, etc.) 
and between the surface of such structures and the contacting fluid.
\paragraph{Heat conduction equation, solid structures:}
\begin{equation}
	\rho_s \, C_{ps} \frac{\partial T}{\partial t} - \nabla \cdot (k_s \nabla T) = q_s 
\label{eq:conduction}
\end{equation}
where \gls[hyper=false]{rhos} is the solid structure mass density;
\gls[hyper=false]{cps} is the solid structure thermal capacity;
\gls[hyper=false]{ks} is the solid structure thermal conductivity;
and $q_s$ is the volumetric heat source term in the solid.

At the contact between fluid and solid material, the total heat flux is given as,
\begin{equation}
	q'' = h_{wl} \, (T_{w} - T_l) + h_{w\text{sat}} \, (T_w - T_\text{sat}) + h_{wg} \, (T_w - T_g)
\label{eq:conduction_surface}
\end{equation}
where the heat flux at the surface of the structure, $q''$ is partitioned to different phases of the fluid, either as sensible or latent heat.
As can be seen, 
Eq.~(\ref{eq:conduction_surface}) couples the heat conduction equation with energy balance equations of the fluid through the terms defined in Eqs.~(\ref{eq:qwl}), (\ref{eq:qwg}), and (\ref{eq:qwsat}).

\subsection{Closure}\label{sub:closure}

