\section{Thermal-Hydraulics System Code \glsentryshort{trace}}\label{sec:reflood_trace}

\gls{trace} is the best-estimate system \gls{th} code developed by the \gls{usnrc} 
as a tool for light water reactor transient analysis during normal and accident scenarios.
Its development is an on-going effort 
to modernize into a single software package all previous \gls{usnrc} \gls{th} codes
that were developed separately for specific reactor types and/or applications.
This ultimately would make the code more versatile for end users and more efficient to maintain for the developer.

The hydraulic model of \gls{trace} is based on a two-fluid six-equation model, 
solving the conservation equations of mass, momentum, and energy for the liquid and vapor phases in the coolant.
These equations are described in the following.

\paragraph{Mass balance equations, liquid and gas phases:}
\begin{equation}
	\frac{\partial [(1-\alpha)\rho_l]}{\partial t} + \nabla \cdot [(1-\alpha) \rho_l \mathbf{v_l}] = - \Gamma
\label{eq:mass_balance_liquid}
\end{equation}
\begin{equation}
	\frac{\partial [\alpha \rho_g]}{\partial t} + \nabla \cdot [\alpha \rho_g \mathbf{v_g}] = \Gamma
\label{eq:mass_balance_gas}
\end{equation}
where the subscripts indicate the phase, $l$ for the liquid phase and $g$ for the gas phase (vapor); 
\gls{alpha} is the void fraction; 
$\rho_\circ$ is the mass density of the respective phase;
and $\mathbf{v}_\circ$ is the velocity of the respective phase.

The first term on the left hand side is the volumetric rate of change of the mass of the corresponding phase.
The second term is the the volumetric mass convection of the corresponding phase.
The term \gls{Gamma} on the right hand side is the volumetric interfacial mass-transfer rate,
with a convention that it is positive for the transfer from liquid phase to gas phase.

\paragraph{Momentum balance equations, liquid and gas phases:}
\begin{equation}
	\begin{split}
		& \frac{\partial [(1-\alpha)\rho_l \mathbf{v}_l]}{\partial t} + \nabla \cdot [(1-\alpha) \rho_l \mathbf{v_l} \otimes \mathbf{v_l}] + (1 - \alpha) \nabla P \\
		& \quad = \mathbf{f}_i + \mathbf{f}_{wl} + (1 - \alpha) \rho_l \mathbf{g} - \Gamma \mathbf{v}_i
	\end{split}
\label{eq:momentum_balance_liquid}
\end{equation}
\begin{equation}
	\begin{split}
		& \frac{\partial [\alpha \rho_g \mathbf{v}_g]}{\partial t} + \nabla \cdot [\alpha \rho_g \mathbf{v_g} \otimes \mathbf{v_g}] + \alpha \nabla P \\
		& \quad = - \mathbf{f}_i + \mathbf{f}_{wg} + \alpha \rho_g \mathbf{g} + \Gamma \mathbf{v}_i
	\end{split}
\label{eq:momentum_balance_gas}
\end{equation}
where $\nabla P$ is the pressure gradient;
\gls{fi} is the volumetric force due to shear at the phase interface;
\gls{fwl} is the volumetric force acting on the liquid phase due to shear at the wall (i.e., fluid-structure contact);
\gls{fwg} is the volumetric force acting on the gas phase due to shear at the wall;
\gls{gravity} is the gravitational acceleration;
and \gls{vinterface} is the flow velocity at the phase interface.

The first term on the left hand side is the volumetric rate of change of the momentum acting on the corresponding phase.
The second term is the volumetric momentum convection acting on the corresponding phase.
The third term is momentum exchange due to pressure gradient.
Here, the formulation in \gls{trace} uses the simplifying assumption of $P_i = P_g = P_l$.
That is, the local pressure is the same in either phases as well as at the interface \cite{USNRC2012}.

All the terms in the right hand side constitute the momentum sources (and sinks) due to shear at the interface and at the wall, 
to body force (i.e, gravity), 
and to mass exchange at the interface, respectively.
For the shear terms, TRACE uses the following formulations
\begin{equation}
	\mathbf{f}_i = C_i (\mathbf{v}_g - \mathbf{v}_l) |\mathbf{v}_g - \mathbf{v}_l| 
\label{eq:fi}
\end{equation}
\begin{equation}
	\mathbf{f}_{wl} = - C_{wl} \mathbf{v}_l |\mathbf{v}_l|
\label{eq:fwl}
\end{equation}
\begin{equation}
	\mathbf{f}_{wg} = - C_{wg} \mathbf{v}_g |\mathbf{v}_g|
\label{eq:fwg}
\end{equation}
where the friction coefficients $C_i$, $C_{wl}$, $C_{wg}$ for interfacial shear, wall-liquid shear, and wall-gas shear, respectively 
are obtained from flow regime-dependent empirical correlations.

\paragraph{Energy balance equations, liquid and gas phases:}
\begin{equation}
	\begin{split}
		& \frac{\partial [(1-\alpha)\rho_l(e_l + |\mathbf{v_l}|^2/2]}{\partial t} + \nabla \cdot \left[(1-\alpha) \rho_l \left(e_l+\frac{P}{\rho_l}+\frac{|\mathbf{v_l}|^2}{2}\right)\right] \\
		&	\quad = q_{il} + q_{wl} + q_{wsat} + q_{dl} + (1 - \alpha) \rho_l \mathbf{g} \cdot \mathbf{v}_l \\
		& \qquad - \Gamma h^\prime_l + (\mathbf{f}_{i} + \mathbf{f}_{wl}) \cdot \mathbf{v}_l
	\end{split}
\label{eq:energy_balance_liquid}
\end{equation}
\begin{equation}
	\begin{split}
		 & \frac{\partial [\alpha \rho_g (e_g + |\mathbf{v_g}|^2/2]}{\partial t} + \nabla \cdot \left[\alpha \rho_g \left(e_g+\frac{P}{\rho_g}+\frac{|\mathbf{v_g}|^2}{2}\right)\right] \\
		 & \quad  = q_{ig} + q_{wg} + q_{dg} + \alpha \rho_g \mathbf{g} \cdot \mathbf{v}_g - \Gamma h^\prime_g + (-\mathbf{f}_{i} + \mathbf{f}_{wg}) \cdot \mathbf{v}_g
	\end{split}
\label{eq:energy_balance_gas}
\end{equation}
where