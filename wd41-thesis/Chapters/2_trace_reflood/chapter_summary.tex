%***********************************************************
\section{Chapter Summary}\label{sec:reflood_chapter_summary}
%***********************************************************

% Opening Paragraph. Reflood
The physical model of the \gls[hyper=false]{th} system code of interest in the present doctoral research has been presented in this chapter.
The reflood phenonema and its modeling with the \gls[hyper=false]{trace} code were presented.

% TRACE Modeling of FEBA
The \gls[hyper=false]{feba} \gls[hyper=false]{setf} for reflood experiment was described and modeled using the \gls[hyper=false]{trace} code.
The simulation of the selected reflood experiment using the \gls[hyper=false]{trace} model with nominal parameters values gave no indication of major deficiency with respect to important outputs.

% Selection and Prior Propagation
A set of $27$ initial input parameters, each of which either belongs to the controllable input or model parameter category, have been selected.
The justification for the selection was given along with the specification of the prior uncertainties associated with the parameters.
The specification of the uncertainties was admittedly imprecise, but deemed adequate for the prior uncertainties.
These priors were then propagated through the \gls[hyper=false]{trace} model of \gls[hyper=false]{feba}.
As expected, the prediction uncertainty bands for all types of output were found to be very wide but at the same time covering all the experimental data points.

% What's Next
The set of methods presented in the next three chapters builds upon the results of this chapter.
The experimental data and the \gls[hyper=false]{trace} model become the basis for the applications of the methods proposed in this thesis.
In Chapter~\ref{ch:gsa}, the importance of each selected input parameter is verified in a quantitative manner via \gls[hyper=false]{sa}.
The assessment will serve as the basis for parameter screening to reduce the size of the problem.
In Chapter~\ref{ch:gp_metamodel}, a fast approximation of the \gls[hyper=false]{trace} model of \gls[hyper=false]{feba} is developed to alleviate the computational burden of evaluating the \gls[hyper=false]{trace} model numerous times.
Finally, in Chapter~\ref{ch:bayesian_calibration}, the selected model parameters are calibrated against the experimental data of \gls[hyper=false]{feba} test No. $216$, which results in an a posteriori quantification of the parameters uncertainties.