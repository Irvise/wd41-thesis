%**************************************************************************
\section{Bayesian Formulation of Calibration Problem}\label{sec:bc_modular}
%**************************************************************************

% Introductory paragraph
The Bayesian framework for model calibration begins by constructing a probabilistic model of $y_E$ given in an additive formulation of Eq.~(\ref{eq:bc_observation_true}). 
That is, it aims at formulating the data generating process $\mathcal{Y}_E(\bm{x}_c; \bm{\lambda})$.
This model implies that the experimental data $y_E$ taken at particular $\bm{x}_c$ observed at $\bm{\lambda}$ is a realization of a stochastic process.
Furthermore, this probabilistic modeling entails casting any \emph{uncertain} element in Eq.~(\ref{eq:bc_observation_true}) either as random variable or stochastic process.

%-----------------------------------------------------------------------------
\subsection{Probabilistic Model for the Model Bias}\label{sub:bc_modular_bias}
%-----------------------------------------------------------------------------

% Introductory Paragraph
Recall the relationship between the true system response and its prediction by a simulator (Eq.~(\ref{eq:bc_true_simulation})) rearranged below:
\begin{equation*}
    \delta (\bm{x}_c; \boldsymbol{\lambda}) = y_T(\bm{x}_c; \boldsymbol{\lambda}) - y_M (\bm{x}_c, \hat{\bm{x}}_m; \boldsymbol{\lambda})
\end{equation*}
where the prediction $y_M$ is made using the best but unknown value of the model parameters.
As such, the model bias function $\delta$ represents a possible systematic difference between the true system response and the simulator prediction that still remains, even from using a simulator with the \emph{best} set of model parameter values.
Note that, strictly speaking, there is a dependence of $\hat{\bm{x}}_m$ on $\delta$, but this dependence is suppress from the notation; $\hat{\bm{x}}_m$, though unknown, should in principle be a unique set of values valid for all $\bm{x}_c$ \cite{Bayarri2007,Arendt2012}.

% Why bias, unbiased model
Incorporating bias term in the calibration procedure is important to avoid overfitting in the model parameters estimates.
Overfitting can be seen in two different ways
Fig.~\ref{fig:ch5_plot_illustrate_bias_1} illustrates a calibration of a computer simulator without the presence of bias, but with a single uncertain model parameter.
As can be seen, a range of the model parameter values can be principle be constrained 
\normdoublefigure[pos=tbhp,
                  mainlabel={fig:ch5_plot_illustrate_bias},
                  maincaption={Illustration of predictions made by computer simulator with and without bias, both with a single uncertain model parameter. Crosses are the observed data along with the associated uncertainty taken at different controllable inputs $x$. Bold lines are the simulator prediction using the maximum and minimum of the uncertain model parameter, while thin lines are the prediction with varying values of the model parameter.},%
									mainshortcaption={.},
                  leftopt={width=0.45\textwidth},
                  leftlabel={fig:ch5_plot_illustrate_bias_1},
                  leftcaption={Model without bias and with uncertain model parameter.},
                  %leftshortcaption={},%
                  rightopt={width=0.45\textwidth},
                  rightlabel={fig:ch5_plot_illustrate_bias_2},
                  rightcaption={Model with biasa and with uncertain model parameter.},
                  %rightshortcaption={},
                  %spacing={\hfill}
                 ]
{../figures/chapter5/figures/plotIllustrateBias_1.pdf}
{../figures/chapter5/figures/plotIllustrateBias_2.pdf}

% Why bias
On the other hand

% Physical Parameter vs. Tuning Parameter, some philosophical issue
The distinction between physical and tuning parameters for the model parameter might have some impact philosophical impact on the two problem before.
In the calibration of a physical parameter based on experimental data,
it is often understood that the calibrated value will in principle be taken as the ``true'' value of that parameter.
Such ``true'' value of the physical parameter can then be used even outside the context of the model where it resides.

% Constructive Path
One might argue that if a model is known to be biased it simply requires more effort to correct the bias by putting more physics.
But this is often impractical for a large complex simulator.

% However, as argued in to insist the model has to be perfect is not very constructive.


% Probability Model for Model Discrepancy
The unknown model bias function $\delta$ can be represented as a random function $\mathcal{D} (\circ)$,
\begin{equation}
        (\mathcal{Y}_T - \mathcal{Y}_M) \equiv \mathcal{D}(\bm{x}_c; \bm{\lambda}) \thicksim p(\delta | \bm{\psi}_{\delta}, \bm{x}_c; \bm{\lambda})
\label{eq:bc_data_generating_bias}
\end{equation}
where $\bm{\psi}_{\delta}$ is the parametrization of the probability density describing the bias at $\bm{x}_c$ and $\bm{\lambda}$.

Casting the unknown model bias term as a stochastic process is the salient feature of Bayesian calibration framework proposed by Kennedy and O'Hagan \cite{Kennedy2001} and it is widely adapted in.

% Gaussian Process for Model Discrepancy
$\mathcal{D} (\circ)$ for $\bm{x}_c \in \mathbf{X}_C \subseteq \mathbb{R}^{D_c}$
\begin{equation}
        \mathcal{D}(\bm{x}_c; \bm{\lambda}) \thicksim \mathcal{GP}(\delta | \bm{\psi}_{\delta}, \bm{x}_c; \bm{\lambda})
\label{eq:bc_data_generating_bias_gp}
\end{equation}


%------------------------------------------------------------------------------------
\subsection{Probabilistic Model for the Experimental Data}\label{sub:bc_modular_data}
%------------------------------------------------------------------------------------

% Introductory paragraph
Now recall the relationship between the true system response and its observation through a measurement (Eq.~(\ref{eq:bc_observation_true})),
\begin{equation*}
    y_E(\bm{x}_c; \boldsymbol{\lambda}) = y_T (\bm{x}_c; \boldsymbol{\lambda}) + \epsilon
\end{equation*}
The observation error term $\epsilon$ represents any possible error during the measurement process, either from the imprecision of the instrument or any other residual variability of the experiment.
\marginpar{Observation error, possible origins}
This variability, in turn, might be due to the inherently stochastic nature of the physical process (irreducible) or unrecognized and uncontrolled variables (reducible) \cite{Kennedy2001}.

% Generic Formulation
Because this term is considered unknown, a stochastic process is defined on the observation layout,
\begin{equation}
        \mathcal{E}(\bm{\lambda}) \thicksim p(\epsilon | \psi_{\epsilon}; \bm{\lambda})
\label{eq:bc_data_generating_exp}
\end{equation}
where $\bm{\psi}_{\epsilon}$ is the parametrization of the \gls[hyper=false]{pdf} describing the observation error $\bm{\lambda}$.
That is, it depends on which response is observed, as well as where and when it is observed.

% Independence assumption and its justification
An important assumption made on the distribution of the observation error is that it is independent conditional on the true value of the system response.
\marginpar{Conditional independence}
One can argue that the measurement data points taken from a spatio-temporal physical process would have (perhaps complicated) correlation structure among them.
But intuitively, as argued in \cite{Wikle2001}, this structure becomes much simplified once the true value is known;
it can mainly be attributed to the residual variability and instrument precision with a simpler description.
The true system response itself is already separately formulated in terms of the simulator prediction and a model bias term (Eq.~(\ref{eq:bc_true_simulation})).
As such, any possible complicated structure of the error (either bias or correlation) is already assigned to the model bias formulation and assuming a simpler measurement error model (i.e., independent) is sufficient \cite{Wikle2001}.
At the same time, as noted in \cite{Kennedy2001,Bayarri2007}, it will be difficult to distinguish two correlation structures separately for the model bias and observation error based on the data alone.

% Gaussian Assumption, some comments
The particular distribution of the observation error is often assumed to be a Gaussian distribution in the
\marginpar{Gaussian observation error}
applied literature \cite{Wikle1998,Wikle2001,Kennedy2001,Bayarri2007,Arhonditsis2008},
\begin{equation}
        \mathcal{E} \thicksim \mathcal{N}(0, \sigma_{obs}^2(\bm{\lambda}))
\label{eq:bc_data_generating_exp_gaussian_1}
\end{equation}
or equivalently following the conditional independence assumption explained above,
\begin{equation}
  \mathcal{Y}_E | \mathcal{Y}_T = y_T(\bm{x}_c; \boldsymbol{\lambda}) \thicksim \mathcal{N}(y_T(\bm{x}_c; \boldsymbol{\lambda}), \sigma_{obs}^2(\bm{\lambda}))
\label{eq:bc_data_generating_exp_gaussian_2}
\end{equation}
where $\sigma_{obs}^2$ is the variance of the Gaussian distribution and the only hyper-parameter of this observation error specification.
The value of the variance depends on the element of the observation layout $\bm{\lambda}$.
Eq.~(\ref{eq:bc_data_generating_exp_gaussian_1}) implies that the observation is taken without bias and the error is independent (but need not be identically distributed) Gaussian random variable.

%---------------------------------------------------------------------------------
\subsection{Probabilistic Model for the Simulator}\label{sub:bc_modular_simulator}
%---------------------------------------------------------------------------------

% Introductory paragraph
For a deterministic simulator $y_M$,
the probabilistic modeling of the bias term $\delta$ and the observation error term $\epsilon$ are enough to formulate a probabilistic model for the experimental observation $\mathcal{Y}_E$.
However, following the development taken in Chapter~\ref{ch:gp_metamodel},
a \glsfirst[hyper=false]{gp} can also be used to represent a deterministic simulator using an explicit formulation of a stochastic process.
The prediction made by the simulator at particular values of $\bm{x}_c$, $\hat{\bm{x}}_m$, and $\bm{\lambda}$ is then given by,
\begin{equation}
          \mathcal{Y}_M (\bm{x}_c, \hat{\bm{x}}_m; \bm{\lambda}) \thicksim  \mathcal{N}(m(\bm{x}_c, \hat{\bm{x}}_m; \bm{\psi}_{m}, \bm{\lambda}), s^2(\bm{x}_c, \hat{\bm{x}}_m; \bm{\psi}_{m}, \bm{\lambda}))
\label{eq:bc_data_generating_simulator_gp}
\end{equation}
where $m$ and $s^2$ is the kriging mean and the kriging variance, respectively (see Section~\ref{sec:gp_metamodeling});
and $\bm{\psi}_{m}$ is the hyper-parameters associated with the specification of the \glsfirst[hyper=false]{gp}.

This step is taken especially if the simulator is computationally expensive to evaluate and only a limited number of simulator runs can be afforded \cite{Kennedy2001,Bayarri2007,Arendt2012}.
The probability model in Eq.~\ref{eq:bc_data_generating_simulator_gp} then becomes an approximation to the actual simulator (i.e., \gls[hyper=false]{gp} metamodel).
Furthermore, as explained in the Chapter~\ref{ch:gp_metamodel}, the uncertainty associated with a prediction by the metamodel at an arbitrary input point stems from the fact that the simulator itself was not run at that input.
This prediction is based on the outputs of which the simulator was run (i.e., the training data)\footnote{the statement conditional on the training data in Eq.~\ref{eq:bc_data_generating_simulator_gp}, i.e., $\mathcal{Y}_M (\bm{x}_c, \hat{\bm{x}}_m; \bm{\lambda}) | \mathcal{Y}(\mathbf{DM})$ has been implicitly assumed.}.
As such, in this case, the uncertainty has an epistemic interpretation.

%-----------------------------------------------------------------------------
\subsection{Posterior of the Model Paramaters}\label{sub:bc_modular_posterior}
%-----------------------------------------------------------------------------

% Introductory Paragraph
Summarizing the above discussions for a deterministic simulator $y_M$,
\begin{equation}
    \begin{split}
				\mathcal{Y}_M & \equiv \mathcal{Y}_M \thicksim p(y_M | \hat{\bm{x}}_m, \bm{x}_c; \bm{\lambda}) = \delta_d (y_M(\hat{\bm{x}}_m, \bm{x}_c; \bm{\lambda})) \\
        (\mathcal{Y}_T - \mathcal{Y}_M) & \equiv \mathcal{D}(\bm{x}_c; \bm{\lambda}) \thicksim p(\delta | \bm{\psi}_{\delta}, \bm{x}_c; \bm{\lambda}) \\
        (\mathcal{Y}_E - \mathcal{Y}_T) & \equiv \mathcal{E}(\bm{\lambda}) \thicksim p(\epsilon | \bm{\psi}_{\epsilon}; \bm{\lambda}) \\
    \end{split}
\label{eq:bc_data_generating_models}
\end{equation}
where $\delta_d$ is the Dirac delta function indicating that the simulator prediction is exact.

Suppose that the form of the densities in Eq.~\ref{eq:bc_data_generating_models} are already given,
then the stochastic process $\mathcal{Y}_E$ is obtained by adding the terms on the right hand side of Eq.~\ref{eq:bc_true_simulation}.
Assuming that they are independent, the \gls[hyper=false]{pdf} of $\mathcal{Y}_E$ is defined as the convolution of the terms,
\begin{equation}
  \begin{split}
  p(y_E | & \bm{\psi}_{\delta}, \bm{\psi}_{\epsilon}, \hat{\bm{x}}_m, \bm{x}_c ; \bm{\lambda}) = \ldots \\
	& (p(y_M(\hat{\bm{x}}_m, \bm{x}_c; \bm{\lambda})) * p(\delta | \bm{\psi}_{\delta}, \bm{x}_c; \bm{\lambda}) * p(\epsilon | \bm{\psi}_{\epsilon}; \bm{\lambda}))(y_E)
  \end{split}
\label{eq:bc_additive_convolution}
\end{equation}
where $*$ is the symbol for the convolution operation.

Given a set of experimental data $\mathbf{y}$ taken at $\mathbf{x}_c$ and observed on an observation layout $\bm{\Lambda}$,
the likelihood function is then defined as follows
\begin{equation}
  \mathcal{L}(\hat{\bm{x}}_m, \bm{\psi}_\delta, \bm{\psi}_\epsilon; \mathbf{y}, \mathbf{x}_c, \bm{\Lambda}) \equiv p(y_E = \mathbf{y} | \bm{x}_c = \mathbf{x}_c, \hat{\bm{x}}_m, \bm{\psi}_\delta, \bm{\psi}_{\epsilon} ; \bm{\Lambda})
\label{eq:bc_likelihood}
\end{equation}

% Full probability model
Following the Bayes' theorem, the joint posterior probability of the model parameters $\bm{x}_m$ and the hyper-parameters associated with the model bias and the observation error is given as, 
\begin{equation}
	\begin{split}
  p(\hat{\bm{x}}_m, & \bm{\psi}_\delta, \bm{\psi}_{\epsilon_y} | \mathbf{y}, \mathbf{x}_c; \bm{\Lambda}) = \ldots \\
	& \frac{\mathcal{L}(\hat{\bm{x}}_m, \bm{\psi}_\delta, \bm{\psi}_{\epsilon_y} ; \mathbf{y}, \mathbf{x}_c, \bm{\Lambda}) \cdot p(\hat{\bm{x}}_m) \cdot p(\bm{\psi}_{\epsilon_y}; \bm{\Lambda}) \cdot p(\bm{\psi}_{\delta}; \bm{\Lambda})}{p(y_E = \mathbf{y} | \bm{x}_c = \mathbf{x}_c ; \bm{\Lambda})}
	\end{split}
\label{eq:bc_joint_posterior}
\end{equation}
where $p(\hat{\bm{x}}_m)$, $p(\psi_{\bm{\epsilon}_y}; \bm{\Lambda})$, and $p(\bm{\psi}_{\delta}; \bm{\Lambda})$ are the prior probabilities for the model parameters, the model bias hyper-parameters, and the observation error hyper-parameters, respectively.
The denominator of the Eq.~\ref{eq:bc_joint_posterior} is a normalizing constant with respect to the model parameters and the hyper-parameters such that Eq.~\ref{eq:bc_joint_posterior} is a valid probability density (i.e., integration over the domain yields the value $1.0$).
As such, it is defined as,
\begin{equation}
	\begin{split}
	p(y_E = \mathbf{y} | \bm{x}_c = \mathbf{x}_c ; \bm{\Lambda}) = & \int \mathcal{L}(\hat{\bm{x}}_m,\bm{\psi}_\delta, \bm{\psi}_\epsilon ;  \mathbf{y}, \mathbf{x}_c, \bm{\Lambda}) \cdot \ldots \\
	& p(\hat{\bm{x}}_m) \cdot p(\bm{\psi}_\epsilon; \bm{\Lambda}) \cdot p(\bm{\psi}_\delta; \bm{\Lambda}) \, d\hat{\bm{x}}_{c} d\bm{\psi}_\epsilon d\bm{\psi}_\delta
	\end{split}
\label{eq:bc_normalizing_constant}
\end{equation}
The specifications of the likelihood and the associated priors completely specify the Bayesian calibration problem of the model parameters.

% Normal Approximation

%---------------------------------------------------------------------------------
\subsection{Modularization of the Bayesian Framework}\label{sub:bc_modularization}
%---------------------------------------------------------------------------------

%The actual forms of the densities in 
%then, under the additive formulation, the data generating process for $\mathcal{Y}_E$ can be obtained by adding all the three terms above.
