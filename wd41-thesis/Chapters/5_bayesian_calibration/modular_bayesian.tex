%**************************************************************************
\section{Bayesian Formulation of Calibration Problem}\label{sec:bc_modular}
%**************************************************************************

% Introductory paragraph
Consider once more the additive formulation of the calibration problem given in Eq.~(\ref{eq:bc_observation_true}).
The Bayesian framework for model calibration begins by constructing a probabilistic model of $y_E$. 
That is, it aims at formulating the data generating process $\mathcal{Y}_E(\bm{x}_c; \bm{\lambda})$.
This model implies that the experimental data $y_E$ taken at particular $\bm{x}_c$ observed at $\bm{\lambda}$ is a realization of a stochastic process.
Furthermore, this probabilistic modeling entails casting any \emph{uncertain} element in Eq.~(\ref{eq:bc_observation_true}) either as random variable or stochastic process.

%---------------------------------------------------------------------------------
\subsection{Probabilistic Model for the Simulator}\label{sub:bc_modular_simulator}
%---------------------------------------------------------------------------------

% Probability Model for Simulator
First, considering a priori that the best model parameters $\hat{\bm{x}}_m$ is uncertain,
the simulator $y_M$ can be casted as a random function (i.e., stochastic process) $\mathcal{Y}_M (\circ)$ as follows
\begin{equation}
          \mathcal{Y}_M \equiv \mathcal{Y}_M (\bm{x}_c, \hat{\bm{x}}_m; \bm{\lambda}) \thicksim  p(y_m | \bm{\psi}_{m}, \bm{x}_c, \hat{\bm{x}}_m; \bm{\lambda})
\label{eq:bc_data_generating_simulator}
\end{equation}
The specification of the probability density for the simulator at particular values of $\bm{x}_c$, $\hat{\bm{x}}_m$, and $\bm{\lambda}$ might require additional parametrization by a set of hyper-parameters $\bm{\psi}_{m}$.
In that case, the specification will be conditional on $\bm{\psi}_{m}$.

% Gaussian Process

% Modular Approach

%-----------------------------------------------------------------------------
\subsection{Probabilistic Model for the Model Bias}\label{sub:bc_modular_bias}
%-----------------------------------------------------------------------------

% Probability Model for Model Discrepancy
Second, the unknown model bias function $\delta$ can be represented as a random function $\mathcal{D} (\circ)$,
\begin{equation}
        (\mathcal{Y}_T - \mathcal{Y}_M) \equiv \mathcal{D}(\bm{x}_c; \bm{\lambda}) \thicksim p(\delta | \bm{\psi}_{\delta}, \bm{x}_c; \bm{\lambda})
\label{eq:bc_data_generating_bias}
\end{equation}
where $\bm{\psi}_{\delta}$ is the (hyper-) parametrization of the probability density describing the bias at $\bm{x}_c$ and $\bm{\lambda}$.

Casting the unknown model bias term as a stochastic process is the salient feature of Bayesian calibration framework proposed by Kennedy and O'Hagan \cite{Kennedy2001}.

% Gaussian Process

% Modular Approach

%------------------------------------------------------------------------------------
\subsection{Probabilistic Model for the Experimental Data}\label{sub:bc_modular_data}
%------------------------------------------------------------------------------------

% Probability Model for Experimental Data
Finally, 
\begin{equation}
        (\mathcal{Y}_E - \mathcal{Y}_T) \equiv \mathcal{E}(\bm{\lambda}) \thicksim p(\epsilon_y | \psi_{\epsilon_y}; \lambda)
\label{eq:bc_data_generating_exp}
\end{equation}


% The Likelihood
In the following,
the three constituents of the above formulation are elaborated further before the posterior of the model parameters are finally formulated.

%-----------------------------------------------------------------------------
\subsection{Posterior of the Model Paramaters}\label{sub:bc_modular_posterior}
%-----------------------------------------------------------------------------

Summarizing the above discussions,
\begin{equation}
    \begin{split}
        \mathcal{Y}_M & \equiv \mathcal{Y}_M (\bm{x}_c, \hat{\bm{x}}_m; \bm{\lambda}) \thicksim  p(y_m | \bm{\psi}_{m}, \bm{x}_c, \hat{\bm{x}}_m; \bm{\lambda}) \\
        (\mathcal{Y}_T - \mathcal{Y}_M) & \equiv \mathcal{D}(\bm{x}_c; \bm{\lambda}) \thicksim p(\delta | \bm{\psi}_{\delta}, \bm{x}_c; \bm{\lambda}) \\
        (\mathcal{Y}_E - \mathcal{Y}_T) & \equiv \mathcal{E}(\bm{\lambda}) \thicksim p(\epsilon | \bm{\psi}_{\epsilon}; \bm{\lambda}) \\
    \end{split}
\label{eq:bc_data_generating_models}
\end{equation}
The actual forms of the densities in Eqs.~(\ref{eq:bc_data_generating_simulator}~--~\ref{eq:bc_data_generating_exp}) above are at this point unimportant and suppose that they are already given,
then, under the additive formulation, the data generating process for $\mathcal{Y}_E$ can be obtained by adding all the three terms above.
Assuming those three terms are independent, the \gls[hyper=false]{pdf} of $\mathcal{Y}_E$ is defined as the convolution of the terms,
\begin{equation}
  \begin{split}
  p(y_E | \bm{\psi}_{m}, \bm{\psi}_{\delta}, \bm{\psi}_{\epsilon}, \hat{\bm{x}}_m, \bm{x}_c ; \bm{\lambda}) = & (p(y_M | \bm{\psi}_{m}, \hat{\bm{x}}_m, \bm{x}_c; \lambda) * \ldots \\
           & p(\delta | \bm{\psi}_{\delta}, \bm{x}_c, ; \bm{\lambda}) * p(\epsilon | \psi_{\epsilon}; \bm{\lambda}))(y_E)
  \end{split}
\label{eq:bc_additive_convolution}
\end{equation}
where $*$ is the symbol for the convolution operation.

Given the experimental data $\mathbf{y}$ taken at $\mathbf{x}_c$ and observed at $\bm{\lambda}$,
the likelihood function is then defined as follows
\begin{equation}
  \mathcal{L}(\hat{\bm{x}}_m, \bm{\psi}_m, \bm{\psi}_\delta, \bm{\psi}_\epsilon; \mathbf{y}, \mathbf{x}_c, \bm{\lambda}) \equiv p(y_E = \mathbf{y} | \bm{x}_c = \mathbf{x}_c, \hat{\bm{x}}_m, \bm{\psi}_m, \bm{\psi}_\delta, \bm{\psi}_{\epsilon} ; \bm{\lambda})
\label{eq:bc_likelihood}
\end{equation}

% Full probability model
Following Bayes' theorem, the probability of the model parameters $\bm{x}_m$
\begin{equation}
  p(\hat{\bm{x}}_m, \psi_\delta, \psi_{\epsilon_y} | \mathbf{y}, \mathbf{x}_c; \lambda) = \frac{\mathcal{L}(\hat{\bm{x}}_m, \psi_\delta, \psi_{\epsilon_y} ; \mathbf{y}, \mathbf{x}_c, \lambda) \cdot p(\hat{\bm{x}}_m) \cdot p(\psi_{\epsilon_y}; \lambda) \cdot p(\psi_{\delta}; \lambda)}{p(y_E = \mathbf{y} | \bm{x}_c = \mathbf{x}_c ; \lambda)}
\label{eq:bc_}
\end{equation}
where the denominator is defined as,
\begin{equation}
	p(y_E = \mathbf{y} | \bm{x}_c = \mathbf{x}_c ; \lambda) = \int \mathcal{L}(\hat{\bm{x}}_m, \psi_\delta, \psi_{\epsilon_y}; \mathbf{y}, \mathbf{x}_c, \lambda) \cdot p(\hat{\bm{x}}_m) \cdot p(\psi_{\epsilon_y}) \cdot p(\psi_{\delta}) d\psi_{\epsilon_y} d\psi_\delta
\label{eq:}
\end{equation}
