%******************************************************************
\section{Statistical Framework}\label{sec:bc_statistical_framework}
%******************************************************************

% Introductory Paragraph, Traditional Model Calibration and Its Implication
Consider a computer simulation of 
\begin{equation}
    y^{Exp.}(\bm{x}_c) = y^T (\bm{x}_c) + \epsilon
\label{eq:bc_computer_model}
\end{equation}
where $y^{Exp.}(\bm{x}_c)$


% Model calibration, definition and its classical sense
Model calibration refers to the process of inferring model parameters values based on the difference between observed/measured data and model output (8).
\marginpar{Model calibration}
Inference implies that the parameter values are not necessarily observable as in the case of reflood model parameters presented in this study.
Calibration, in traditional sense, then proceed to identify a set of model parameters values that best fits the available data (9).


% Complication in model calibration

% Statistical Calibration

% Statistical Calibration of KOH

% Bayes' theorem revisited

% Citation 8: van Oijen
% Citation 9: Campbell
% KOH Approach, the chapter's notational convention
\begin{equation}
    y^T(\bm{x}_c) = y^M (\bm{x}_c, \bm{\hat{x}}_m) + \delta(\bm{x}_c)
\label{eq:bc_koh_true_model}
\end{equation}

% KOH Approach, the chapter's notational convention
\begin{equation}
    y^{Exp.}(\bm{x}_c) = y^T (\bm{x}_c) + \epsilon
\label{eq:bc_koh_true_measured}
\end{equation}

% Control and Model Parameters
% Simulator and Model discrepancy

% The meaning of true parameters

% The essence of Bayesian Data Analysis:
% 1. Set up full probabilistic model
% 2. Carry out Bayesian Computation