%******************************************************************
\section{Statistical Framework}\label{sec:bc_statistical_framework}
%******************************************************************

% Model calibration, definition and its classical sense
Model calibration refers to the process of inferring model parameters values based on the difference between observed/measured data and model output (8).
\marginpar{Model calibration}
Inference implies that the parameter values are not necessarily observable as in the case of reflood model parameters presented in this study.
Calibration, in traditional sense, then proceed to identify a set of model parameters values that best fits the available data (9).

% Complication in model calibration

% Bayes' theorem revisited

% Citation 8: van Oijen
% Citation 9: Campbell
% KOH Approach, the chapter's notational convention

% Simulator and Model discrepancy

% The meaning of true parameters
