%******************************************************************
\section{Statistical Framework}\label{sec:bc_statistical_framework}
%******************************************************************

% Introductory Paragraph
The calibration framework in this thesis is in line with the seminal work of Kennedy and O'Hagan \cite{Kennedy2001} which is widely adapted in the applied literature \cite{Bayarri2007,Higdon2008,Arendt2012,Reichert2012}.
\marginpar{Calibration framework}
Meanwhile its explicit formulation uses a notation adapted from different sources \cite{Kennedy2001,Santner2003,Reichert2012}.
Let $y_E$ be the experimental observation of a variable,
then its relationship to the true value $y_T$ is given by
\begin{equation}
    y_E(\bm{x}_c; \boldsymbol{\Lambda}) = y_T (\bm{x}_c; \boldsymbol{\Lambda}) + \epsilon
\label{eq:bc_observation_true}
\end{equation}
where $\epsilon$ is an observation error.
The true value, in turn, is linked to the prediction made by a computer simulator of a physical model $y_M$,
\begin{equation}
    y_T(\bm{x}_c; \boldsymbol{\Lambda}) = y_M (\bm{x}_c, \bm{x}_m; \boldsymbol{\Lambda}) + \delta (\bm{x}_c; \boldsymbol{\Lambda})
\label{eq:bc_true_simulation}
\end{equation}
That is, apart from the \emph{model bias} $\delta$, the simulator prediction is potentially a valid representation of the true value.
Finally, combining the two relationships yields,
\begin{equation}
    y_E(\bm{x}_c; \boldsymbol{\Lambda}) = y_M (\bm{x}_c, \bm{x}_m; \boldsymbol{\Lambda}) + \delta (\bm{x}_c; \boldsymbol{\Lambda}) + \epsilon
\label{eq:bc_observation_true}
\end{equation}
These representations are parametrized by \emph{control} parameters $\bm{x}_c$,
\emph{model} parameters $\bm{x}_m$, and an \emph{observation (experimental) layout} $\boldsymbol{\Lambda}$.

% Parameterization, Model parameters vs Control variables
Departing from the previous chapters, this chapter categorically distinguishes two types of input parameters: the control and model parameters.
The control 

% Observation layout
and $\boldsymbol{\Lambda}$ is the experimental layout, an index set.
Following Reichert, the notion of layout defines the different type of outputs as well as their location and instance, such that a long vectors for model prediction and experimental data can be constructed.
For instance, the observation layout $\boldsymbol{\Lambda} = \{(A,t_1), (B,t_1), (A, t_2)\}$ might be used to signify model output/experimental data of variable $A$ at time $t_1$, variable $B$ at time $t_1$, and variable $A$ at time $t_2$.
The vector $\bm{y}^{(\lambda)}_M$ and $\bm{y}^{(\lambda)}_M$ for $\lambda \in \boldsymbol{\Lambda}$ will refer to the model prediction and experimental data using the index set, respectively.
The distinction between $\bm{x}_c$ and $\bm{x}_c$ is conceptual follow.

% Goal of Calibration
The goal of calibration, in classical sense (citation needed), is to obtain the best value of $\bm{x}_c$ base on the available experimental data $y$.
This is done through an optimization procedure with respect to some error measures between predicted value and experimental data, such that the difference is consistent with the measurement error.

% Statistical Calibration of Kennedy and O'Hagan

% Model calibration, definition and its classical sense
Model calibration refers to the process of inferring model parameters values based on the difference between observed/measured data and model output (8).
\marginpar{Model calibration}
Inference implies that the parameter values are not necessarily observable as in the case of reflood model parameters presented in this study.
Calibration, in traditional sense, then proceed to identify a set of model parameters values that best fits the available data (9).


% Complication in model calibration

% Statistical Calibration

% Statistical Calibration of KOH

% Bayes' theorem revisited

% Citation 8: van Oijen
% Citation 9: Campbell
% KOH Approach, the chapter's notational convention
\begin{equation}
    y^T(\bm{x}_c) = y^M (\bm{x}_c, \bm{\hat{x}}_m) + \delta(\bm{x}_c)
\label{eq:bc_koh_true_model}
\end{equation}

% KOH Approach, the chapter's notational convention
\begin{equation}
    y^{Exp.}(\bm{x}_c) = y^T (\bm{x}_c) + \epsilon
\label{eq:bc_koh_true_measured}
\end{equation}

% Control and Model Parameters
% Simulator and Model discrepancy

% The meaning of true parameters

% The essence of Bayesian Data Analysis:
% 1. Set up full probabilistic model
% 2. Carry out Bayesian Computation