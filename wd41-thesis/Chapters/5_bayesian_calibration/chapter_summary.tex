%******************************************************
\section{Chapter Summary}\label{sec:bc_chapter_summary}
%******************************************************

% Opening Paragraph, this chapter
The model calibration part of the proposed statistical framework has been presented in this chapter.
The goal of the chapter was to quantify the uncertainty associated with model parameters a posteriori based on observed/experimental data.
The quantification followed a Bayesian calibration framework.

% Reiterate first part (theory)
The Bayesian calibration framework has been detailed in this chapter, consisting of two parts:
the formulation of a probabilistic model for the calibration;
and the computation of the formulated model to obtain the posterior uncertainty of the model parameters.
A generic calibration formulation was presented along with an account on each of its element.
Afterward, the computational aspects of a posterior distribution were presented.
\gls[hyper=false]{mcmc} simulation was used in this thesis to directly obtain samples from the posterior,
useful for characterization of the posterior uncertainty or for uncertainty propagation.
Finally, some practical aspects of analyzing the resulting samples were provided.

% Connection with Previous Chapters
The results of this chapter have been built upon the results of the three previous chapters.
The development of the \gls[hyper=false]{trace} model of interest, the selection of initial input parameters, as well as the assignment of the parameters prior uncertainty were conducted in Chapter~\ref{ch:trace_reflood}.
The sensitivity analysis of the initial parameters selection were carried out in Chapter~\ref{ch:gsa}, in which the parameters importance were quantified and used as a basis for screening, reducing the size of the problem beforehand.
In Chapter~\ref{ch:gp_metamodel}, a fast surrogate of the \gls[hyper=false]{trace} model was developed -- and validated -- and used in this chapter as a substitute of directly running the \gls[hyper=false]{trace} code.

% Reiterate the application part, calibration schemes.
The calibration framework was applied to the running case study of the simulation of a reflood experiment using \gls[hyper=false]{trace} conducted at the \gls[hyper=false]{feba} facility.
Five calibration schemes that result in five likelihood (thus posterior) formulations were considered.
The schemes varies with respect to which type (or types) of experimental data is considered and whether model bias term is included.
An additional scheme was introduced to investigate the effect of removing a strongly correlated parameter from the calibration process.
The formulated posterior \glspl[hyper=false]{pdf} were then simulated using an implementation of \gls[hyper=false]{aies} \gls[hyper=false]{mcmc} ensemble sampler to obtain different sets of posterior samples.
Finally, these sets of samples were propagated through all the \gls[hyper=false]{trace} models of the \gls[hyper=false]{feba} tests.

% Result 1: MCMC Convergence

% Result 2: Identifiability Issues

% Result 4: Correlation in the propagation

% Result 3: Regarding Nominal Parameter Values (w/ w/o bias)

% Result 5: Across FEBA Test
