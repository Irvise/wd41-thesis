% ****************************************************************************************************
% classicthesis-config.tex 
% formerly known as loadpackages.sty, classicthesis-ldpkg.sty, and classicthesis-preamble.sty 
% Use it at the beginning of your ClassicThesis.tex, or as a LaTeX Preamble 
% in your ClassicThesis.{tex,lyx} with % ****************************************************************************************************
% classicthesis-config.tex 
% formerly known as loadpackages.sty, classicthesis-ldpkg.sty, and classicthesis-preamble.sty 
% Use it at the beginning of your ClassicThesis.tex, or as a LaTeX Preamble 
% in your ClassicThesis.{tex,lyx} with % ****************************************************************************************************
% classicthesis-config.tex 
% formerly known as loadpackages.sty, classicthesis-ldpkg.sty, and classicthesis-preamble.sty 
% Use it at the beginning of your ClassicThesis.tex, or as a LaTeX Preamble 
% in your ClassicThesis.{tex,lyx} with % ****************************************************************************************************
% classicthesis-config.tex 
% formerly known as loadpackages.sty, classicthesis-ldpkg.sty, and classicthesis-preamble.sty 
% Use it at the beginning of your ClassicThesis.tex, or as a LaTeX Preamble 
% in your ClassicThesis.{tex,lyx} with \input{classicthesis-config}
% ****************************************************************************************************  
% If you like the classicthesis, then I would appreciate a postcard. 
% My address can be found in the file ClassicThesis.pdf. A collection 
% of the postcards I received so far is available online at 
% http://postcards.miede.de
% ****************************************************************************************************


% ****************************************************************************************************
% 0. Set the encoding of your files. UTF-8 is the only sensible encoding nowadays. If you can't read
% äöüßáéçèê∂åëæƒÏ€ then change the encoding setting in your editor, not the line below. If your editor
% does not support utf8 use another editor!
% ****************************************************************************************************
\PassOptionsToPackage{utf8}{inputenc}
	\usepackage{inputenc}

% ****************************************************************************************************
% 1. Configure classicthesis for your needs here, e.g., remove "drafting" below 
% in order to deactivate the time-stamp on the pages
% ****************************************************************************************************
\PassOptionsToPackage{eulerchapternumbers,listings,drafting,%
					 pdfspacing,%floatperchapter,%linedheaders,%
					 subfig,beramono,eulermath,parts}{classicthesis}                                        
% ********************************************************************
% Available options for classicthesis.sty 
% (see ClassicThesis.pdf for more information):
% drafting
% parts nochapters linedheaders
% eulerchapternumbers beramono eulermath pdfspacing minionprospacing
% tocaligned dottedtoc manychapters
% listings floatperchapter subfig
% ********************************************************************


% ****************************************************************************************************
% 2. Personal data and user ad-hoc commands
% ****************************************************************************************************
\newcommand{\myTitle}{Bayesian Uncertainty Quantification of Physical Models in Thermal-Hydraulics System Codes\xspace}
\newcommand{\mySubtitle}{\xspace}
\newcommand{\myDegree}{ing. nucl. dipl. EPF\xspace}
\newcommand{\myName}{Damar Canggih Wicaksono\xspace}
\newcommand{\myProf}{Prof. Andreas Pautz\xspace}
\newcommand{\myOtherProf}{Put name here\xspace}
\newcommand{\mySupervisor}{Omar Zerkak\xspace}
\newcommand{\myFaculty}{Put data here\xspace}
\newcommand{\myDepartment}{Laboratory for Reactor Physics and Systems Behaviour\xspace}
\newcommand{\myUni}{École polytechnique fédérale de Lausanne\xspace}
\newcommand{\myLocation}{Lausanne\xspace}
\newcommand{\myTime}{April 2017\xspace}
\newcommand{\myVersion}{version 4.2\xspace}

% ********************************************************************
% Setup, finetuning, and useful commands
% ********************************************************************
\newcounter{dummy} % necessary for correct hyperlinks (to index, bib, etc.)
\newlength{\abcd} % for ab..z string length calculation
\providecommand{\mLyX}{L\kern-.1667em\lower.25em\hbox{Y}\kern-.125emX\@}
\newcommand{\ie}{i.\,e.}
\newcommand{\Ie}{I.\,e.}
\newcommand{\eg}{e.\,g.}
\newcommand{\Eg}{E.\,g.} 
% ****************************************************************************************************


% ****************************************************************************************************
% 3. Loading some handy packages
% ****************************************************************************************************
% ******************************************************************** 
% Packages with options that might require adjustments
% ******************************************************************** 
%\PassOptionsToPackage{ngerman,american}{babel}   % change this to your language(s)
% Spanish languages need extra options in order to work with this template
%\PassOptionsToPackage{spanish,es-lcroman}{babel}
	\usepackage{babel}                  

\usepackage{csquotes}
\PassOptionsToPackage{%
    %backend=biber, %instead of bibtex
	backend=bibtex8,bibencoding=ascii,%
	language=auto,%
	style=numeric-comp,%
    %style=authoryear-comp, % Author 1999, 2010
    %bibstyle=authoryear,dashed=false, % dashed: substitute rep. author with ---
    sorting=none, %nyts name, year, title
    maxbibnames=10, % default: 3, et al.
    %backref=true,%
    natbib=true % natbib compatibility mode (\citep and \citet still work)
}{biblatex}
    \usepackage{biblatex}

\PassOptionsToPackage{fleqn}{amsmath}       % math environments and more by the AMS 
    \usepackage{amsmath}

% ******************************************************************** 
% General useful packages
% ******************************************************************** 
\PassOptionsToPackage{T1}{fontenc} % T2A for cyrillics
    \usepackage{fontenc}     
\usepackage{textcomp} % fix warning with missing font shapes
\usepackage{scrhack} % fix warnings when using KOMA with listings package          
\usepackage{xspace} % to get the spacing after macros right  
\usepackage{mparhack} % get marginpar right
\usepackage{fixltx2e} % fixes some LaTeX stuff --> since 2015 in the LaTeX kernel (see below)
%\usepackage[latest]{latexrelease} % will be used once available in more distributions (ISSUE #107)
\PassOptionsToPackage{printonlyused,smaller}{acronym} 
    \usepackage{acronym} % nice macros for handling all acronyms in the thesis
    %\renewcommand{\bflabel}[1]{{#1}\hfill} % fix the list of acronyms --> no longer working
    %\renewcommand*{\acsfont}[1]{\textsc{#1}} 
    \renewcommand*{\aclabelfont}[1]{\acsfont{#1}}
% ****************************************************************************************************

% ****************************************************************************************************
% 4. Setup floats: tables, (sub)figures, and captions
% ****************************************************************************************************
\usepackage{tabularx} % better tables
    \setlength{\extrarowheight}{3pt} % increase table row height
\newcommand{\tableheadline}[1]{\multicolumn{1}{c}{\spacedlowsmallcaps{#1}}}
\newcommand{\myfloatalign}{\centering} % to be used with each float for alignment
\usepackage{caption}
% Thanks to cgnieder and Claus Lahiri
% http://tex.stackexchange.com/questions/69349/spacedlowsmallcaps-in-caption-label
% [REMOVED DUE TO OTHER PROBLEMS, SEE ISSUE #82]    
%\DeclareCaptionLabelFormat{smallcaps}{\bothIfFirst{#1}{~}\MakeTextLowercase{\textsc{#2}}}
%\captionsetup{font=small,labelformat=smallcaps} % format=hang,
\captionsetup{font=small} % format=hang,
\usepackage{subfig}  
% ****************************************************************************************************


% ****************************************************************************************************
% 5. Setup code listings
% ****************************************************************************************************
\usepackage{listings} 
%\lstset{emph={trueIndex,root},emphstyle=\color{BlueViolet}}%\underbar} % for special keywords
\lstset{language=[LaTeX]Tex,%C++,
    morekeywords={PassOptionsToPackage,selectlanguage},
    keywordstyle=\color{RoyalBlue},%\bfseries,
    basicstyle=\small\ttfamily,
    %identifierstyle=\color{NavyBlue},
    commentstyle=\color{Green}\ttfamily,
    stringstyle=\rmfamily,
    numbers=none,%left,%
    numberstyle=\scriptsize,%\tiny
    stepnumber=5,
    numbersep=8pt,
    showstringspaces=false,
    breaklines=true,
    %frameround=ftff,
    %frame=single,
    belowcaptionskip=.75\baselineskip
    %frame=L
} 
% ****************************************************************************************************             


% ****************************************************************************************************
% 6. PDFLaTeX, hyperreferences and citation backreferences
% ****************************************************************************************************
% ********************************************************************
% Using PDFLaTeX
% ********************************************************************
\PassOptionsToPackage{pdftex,hyperfootnotes=false,pdfpagelabels}{hyperref}
    \usepackage{hyperref}  % backref linktocpage pagebackref
\pdfcompresslevel=9
\pdfadjustspacing=1 
\PassOptionsToPackage{pdftex}{graphicx}
    \usepackage{graphicx} 
 

% ********************************************************************
% Hyperreferences
% ********************************************************************
\hypersetup{%
    %draft, % = no hyperlinking at all (useful in b/w printouts)
    colorlinks=true, linktocpage=true, pdfstartpage=3, pdfstartview=FitV,%
    % uncomment the following line if you want to have black links (e.g., for printing)
    %colorlinks=false, linktocpage=false, pdfstartpage=3, pdfstartview=FitV, pdfborder={0 0 0},%
    breaklinks=true, pdfpagemode=UseNone, pageanchor=true, pdfpagemode=UseOutlines,%
    plainpages=false, bookmarksnumbered, bookmarksopen=true, bookmarksopenlevel=1,%
    hypertexnames=true, pdfhighlight=/O,%nesting=true,%frenchlinks,%
    urlcolor=webbrown, linkcolor=RoyalBlue, citecolor=webgreen, %pagecolor=RoyalBlue,%
    %urlcolor=Black, linkcolor=Black, citecolor=Black, %pagecolor=Black,%
    pdftitle={\myTitle},%
    pdfauthor={\textcopyright\ \myName, \myUni, \myFaculty},%
    pdfsubject={},%
    pdfkeywords={},%
    pdfcreator={pdfLaTeX},%
    pdfproducer={LaTeX with hyperref and classicthesis}%
}   

% ********************************************************************
% Setup autoreferences
% ********************************************************************
% There are some issues regarding autorefnames
% http://www.ureader.de/msg/136221647.aspx
% http://www.tex.ac.uk/cgi-bin/texfaq2html?label=latexwords
% you have to redefine the makros for the 
% language you use, e.g., american, ngerman
% (as chosen when loading babel/AtBeginDocument)
% ********************************************************************
\makeatletter
\@ifpackageloaded{babel}%
    {%
       \addto\extrasamerican{%
			\renewcommand*{\figureautorefname}{Figure}%
			\renewcommand*{\tableautorefname}{Table}%
			\renewcommand*{\partautorefname}{Part}%
			\renewcommand*{\chapterautorefname}{Chapter}%
			\renewcommand*{\sectionautorefname}{Section}%
			\renewcommand*{\subsectionautorefname}{Section}%
			\renewcommand*{\subsubsectionautorefname}{Section}%     
                }%
       \addto\extrasngerman{% 
			\renewcommand*{\paragraphautorefname}{Absatz}%
			\renewcommand*{\subparagraphautorefname}{Unterabsatz}%
			\renewcommand*{\footnoteautorefname}{Fu\"snote}%
			\renewcommand*{\FancyVerbLineautorefname}{Zeile}%
			\renewcommand*{\theoremautorefname}{Theorem}%
			\renewcommand*{\appendixautorefname}{Anhang}%
			\renewcommand*{\equationautorefname}{Gleichung}%        
			\renewcommand*{\itemautorefname}{Punkt}%
                }%  
            % Fix to getting autorefs for subfigures right (thanks to Belinda Vogt for changing the definition)
            \providecommand{\subfigureautorefname}{\figureautorefname}%             
    }{\relax}
\makeatother


% ****************************************************************************************************
% 7. Last calls before the bar closes
% ****************************************************************************************************
% ********************************************************************
% Development Stuff
% ********************************************************************
\listfiles
%\PassOptionsToPackage{l2tabu,orthodox,abort}{nag}
%   \usepackage{nag}
%\PassOptionsToPackage{warning, all}{onlyamsmath}
%   \usepackage{onlyamsmath}

% ********************************************************************
% Last, but not least...
% ********************************************************************
\usepackage{classicthesis} 
% ****************************************************************************************************


% ****************************************************************************************************
% 8. Further adjustments (experimental)
% ****************************************************************************************************
% ********************************************************************
% Changing the text area
% ********************************************************************
%\linespread{1.05} % a bit more for Palatino
%\areaset[current]{312pt}{761pt} % 686 (factor 2.2) + 33 head + 42 head \the\footskip
%\setlength{\marginparwidth}{7em}%
%\setlength{\marginparsep}{2em}%

% ********************************************************************
% Using different fonts
% ********************************************************************
%\usepackage[oldstylenums]{kpfonts} % oldstyle notextcomp
%\usepackage[osf]{libertine}
%\usepackage[light,condensed,math]{iwona}
%\renewcommand{\sfdefault}{iwona}
%\usepackage{lmodern} % <-- no osf support :-(
%\usepackage{cfr-lm} % 
%\usepackage[urw-garamond]{mathdesign} <-- no osf support :-(
%\usepackage[default,osfigures]{opensans} % scale=0.95 
%\usepackage[sfdefault]{FiraSans}
% ****************************************************************************************************

% ***********************************************************************
% Additional WD41 packages
% ***********************************************************************
\usepackage{lipsum}
\usepackage{array}
\usepackage{ragged2e}
\usepackage{multirow}
\usepackage{chngcntr}
\usepackage{rotating}
\usepackage{algorithm}
\usepackage{algorithmic}
\usepackage[style=long,nolist,nonumberlist,toc,acronym]{glossaries}
\usepackage[roman]{parnotes}
\renewcommand{\thefootnote}{\fnsymbol{footnote}}
%%%%%%%%%%%%%%%%%%%%%%%%%%%%%%%%%%%%%%%%%%%%%%%%%%%%%%%%%%%%%%%%%%%%%%%%%%%%%%
%
% FIGURE TOOLS 
% (From Bejamin Hopfer's Thesis)
% http://benjaminhopfer.com/2014/04/16/typesetting-my-masters-thesis-in-latex/
%
%%%%%%%%%%%%%%%%%%%%%%%%%%%%%%%%%%%%%%%%%%%%%%%%%%%%%%%%%%%%%%%%%%%%%%%%%%%%%%
%%helper functions, no direct use
\usepackage{ifoddpage}

\newlength\fullmarginwidth
\fullmarginwidth=\marginparwidth
\advance\fullmarginwidth by \marginparsep

\newlength\fullwidth
\fullwidth=\textwidth
\advance\fullwidth by \fullmarginwidth


%                   #1   #2       #3     #4   #5
%\subFloatCorrCaps{Cap}{ShortCap}{Label}{Opt}{path}
\newcommand{\subFloatCorrCaps}[5]{
	\ifthenelse{\equal{#1}{}}%
  {
    %subfloat without captions
    \subfloat{%
      \expandafter\includegraphics\expandafter[#4]{#5}
      \label{#3}
    }
  }
  {
    \ifthenelse{\equal{#2}{}}%
    {
      %subfloat with same captions
      \subfloat[#1][#1]{%
        \expandafter\includegraphics\expandafter[#4]{#5}
        \label{#3}
      }
    }
    {
      %subfloat with different captions
      \subfloat[#2][#1]{%
        \expandafter\includegraphics\expandafter[#4]{#5}
        \label{#3}
      }
    }
  }
}

%Parameters: [keyvals (see below)]{file}{caption}
\makeatletter
  \define@key[FigTools]{OneFig}{pos}{\def\FtPos{#1}}
  \define@key[FigTools]{OneFig}{opt}{\def\FtOpt{#1}}
  \define@key[FigTools]{OneFig}{label}{\def\FtLabel{#1}}
  \define@key[FigTools]{OneFig}{shortcaption}{\def\FtShortCaption{#1}}
  
  \define@key[FigTools]{TwoFig}{pos}{\def\FtPos{#1}}
  \define@key[FigTools]{TwoFig}{mainlabel}{\def\FtMainLabel{#1}}
  \define@key[FigTools]{TwoFig}{maincaption}{\def\FtMainCaption{#1}}
  \define@key[FigTools]{TwoFig}{mainshortcaption}{\def\FtMainShortCaption{#1}}
  \define@key[FigTools]{TwoFig}{leftopt}{\def\FtLeftOpt{#1}}
  \define@key[FigTools]{TwoFig}{leftlabel}{\def\FtLeftLabel{#1}}
  \define@key[FigTools]{TwoFig}{leftcaption}{\def\FtLeftCaption{#1}}
  \define@key[FigTools]{TwoFig}{leftshortcaption}{\def\FtLeftShortCaption{#1}}
  \define@key[FigTools]{TwoFig}{rightopt}{\def\FtRightOpt{#1}}
  \define@key[FigTools]{TwoFig}{rightlabel}{\def\FtRightLabel{#1}}
  \define@key[FigTools]{TwoFig}{rightcaption}{\def\FtRightCaption{#1}}  
  \define@key[FigTools]{TwoFig}{rightshortcaption}{\def\FtRightShortCaption{#1}}    
  \define@key[FigTools]{TwoFig}{spacing}{\def\FtSpacing{#1}}    
  \define@key[FigTools]{TwoFig}{topspace}{\def\FtTopSpace{#1}}    
    

  \define@key[FigTools]{ThreeFig}{pos}{\def\FtPos{#1}}
  \define@key[FigTools]{ThreeFig}{mainlabel}{\def\FtMainLabel{#1}}
  \define@key[FigTools]{ThreeFig}{maincaption}{\def\FtMainCaption{#1}}
  \define@key[FigTools]{ThreeFig}{mainshortcaption}{\def\FtMainShortCaption{#1}}
  \define@key[FigTools]{ThreeFig}{leftopt}{\def\FtLeftOpt{#1}}
  \define@key[FigTools]{ThreeFig}{leftlabel}{\def\FtLeftLabel{#1}}
  \define@key[FigTools]{ThreeFig}{leftcaption}{\def\FtLeftCaption{#1}}
  \define@key[FigTools]{ThreeFig}{leftshortcaption}{\def\FtLeftShortCaption{#1}}
  \define@key[FigTools]{ThreeFig}{midopt}{\def\FtMidOpt{#1}}
  \define@key[FigTools]{ThreeFig}{midlabel}{\def\FtMidLabel{#1}}
  \define@key[FigTools]{ThreeFig}{midcaption}{\def\FtMidCaption{#1}}
  \define@key[FigTools]{ThreeFig}{midshortcaption}{\def\FtMidShortCaption{#1}}
  \define@key[FigTools]{ThreeFig}{rightopt}{\def\FtRightOpt{#1}}
  \define@key[FigTools]{ThreeFig}{rightlabel}{\def\FtRightLabel{#1}}
  \define@key[FigTools]{ThreeFig}{rightcaption}{\def\FtRightCaption{#1}}  
  \define@key[FigTools]{ThreeFig}{rightshortcaption}{\def\FtRightShortCaption{#1}}
  \define@key[FigTools]{ThreeFig}{spacing}{\def\FtSpacing{#1}}
  \define@key[FigTools]{ThreeFig}{spacingtwo}{\def\FtSpacingTwo{#1}}
\makeatother

\newcommand{\bigplot}[3][]{%
  \bigfloatinternal[#1]{#2}{#3}{plot}
}

\newcommand{\bigfigure}[3][]{%
  \bigfloatinternal[#1]{#2}{#3}{graph}
}

\newcommand{\bigfloatinternal}[4][]{%
  \begingroup
    \setkeys[FigTools]{OneFig}{ pos=!htp,
                                label={fig:#2},
                                shortcaption={#3},
                                opt={}%width=0.95\textwidth}
                              }
    \setkeys[FigTools]{OneFig}{#1}
    \def\efigure{\begin{figure}}%
    \expandafter\efigure\expandafter[\FtPos]
      \checkoddpage
      \edef\side{\ifoddpage l\else r\fi}
      \makebox[\textwidth][\side]{% 
        \begin{minipage}[t]{\fullwidth}
          \centering
          \ifthenelse{\equal{#4}{plot}}
      		{
          	\input{\evalDir{#2}}
      		}{
          	\expandafter\includegraphics\expandafter[\FtOpt]{#2}
          }
          \caption[\FtShortCaption]{#3}
          \label{\FtLabel}
        \end{minipage}
      }%
    \end{figure}
  \endgroup
}

\newcommand{\bigdoublefigure}[3][]{%
  \begingroup
    %Default values:
    \setkeys[FigTools]{TwoFig}{ pos=!htp,
                                mainlabel={fig:#2-#3},
                                maincaption={#2-#3},
                                mainshortcaption={},%
                                leftopt={},%0.45\textwidth,
                                leftlabel={fig:#2-left},
                                leftcaption={},
                                leftshortcaption={},%
                                rightopt={},%0.45\textwidth,
                                rightlabel={fig:#3-right},
                                rightcaption={},
                                rightshortcaption={},
                                spacing={\hfill},
                                topspace={}
                              }
    %User provided values:
    \setkeys[FigTools]{TwoFig}{#1}
    \def\efigure{\begin{figure}}%
    \expandafter\efigure\expandafter[\FtPos]
      \FtTopSpace
      %Check on which side whe are (right or left)
      \checkoddpage
      \edef\side{\ifoddpage l\else r\fi}
      %Ensure there will be no overfull box message
      \makebox[\textwidth][\side]{% 
        \begin{minipage}{\fullwidth}
          \centering
					\subFloatCorrCaps{\FtLeftCaption}
			                     {\FtLeftShortCaption}
			                     {\FtLeftLabel}
			                     {\FtLeftOpt}
			                     {#2}
			    \FtSpacing
					\subFloatCorrCaps{\FtRightCaption}
			                     {\FtRightShortCaption}
			                     {\FtRightLabel}
			                     {\FtRightOpt}
			                     {#3}
          \caption[\FtMainShortCaption]{\FtMainCaption}
          \label{\FtMainLabel}
        \end{minipage}
      }
    \end{figure}
  \endgroup
}

\newcommand{\bigtriplefigure}[4][]{%
  \begingroup
    %Default values:
    \setkeys[FigTools]{ThreeFig}{ pos=!htp,
	                                mainlabel={fig:#2-#3},
	                                maincaption={#2-#3},
	                                mainshortcaption={},%
	                                leftopt={},%0.45\textwidth,
	                                leftlabel={fig:#2-left},
	                                leftcaption={},
	                                leftshortcaption={},%
	                                midopt={},%0.45\textwidth,
	                                midlabel={fig:#3-mid},
	                                midcaption={},
	                                midshortcaption={},%
	                                rightopt={},%0.45\textwidth,
	                                rightlabel={fig:#4-right},
	                                rightcaption={},
	                                rightshortcaption={},
	                                spacing={\hfill},
	                                spacingtwo={}
	                              }
    %User provided values:
    \setkeys[FigTools]{ThreeFig}{#1}
    \def\efigure{\begin{figure}}%
    \expandafter\efigure\expandafter[\FtPos]
      %Check on which side whe are (right or left)
      \checkoddpage
      \edef\side{\ifoddpage l\else r\fi}
      %Ensure there will be no overfull box message
      \makebox[\textwidth][\side]{% 
        \begin{minipage}{\fullwidth}
          \centering
					\subFloatCorrCaps{\FtLeftCaption}
			                     {\FtLeftShortCaption}
			                     {\FtLeftLabel}
			                     {\FtLeftOpt}
			                     {#2}
			    \FtSpacing
					\subFloatCorrCaps{\FtMidCaption}
			                     {\FtMidShortCaption}
			                     {\FtMidLabel}
			                     {\FtMidOpt}
			                     {#3}
			    \FtSpacing
			    \FtSpacingTwo
					\subFloatCorrCaps{\FtRightCaption}
			                     {\FtRightShortCaption}
			                     {\FtRightLabel}
			                     {\FtRightOpt}
			                     {#4}
          \caption[\FtMainShortCaption]{\FtMainCaption}
          \label{\FtMainLabel}
        \end{minipage}
      }
    \end{figure}
  \endgroup
}

\newcommand{\normplot}[3][]{%
  \normfloatinternal[#1]{#2}{#3}{plot}
}

\newcommand{\normfigure}[3][]{%
  \normfloatinternal[#1]{#2}{#3}{graph}
}

\newcommand{\normfloatinternal}[4][]{%
  \begingroup
    \setkeys[FigTools]{OneFig}{ pos=!htp,
                                label={fig:#2},
                                shortcaption={#3},
                                opt={}%{width=0.95\textwidth}
                              }
    \setkeys[FigTools]{OneFig}{#1}
    \def\efigure{\begin{figure}}%
    \expandafter\efigure\expandafter[\FtPos]
      \centering
      \ifthenelse{\equal{#4}{plot}}
      {
        \input{\evalDir{#2}}
      }{
      	\expandafter\includegraphics\expandafter[\FtOpt]{#2}
      }
      \caption[\FtShortCaption]{#3}
      \label{\FtLabel}
    \end{figure}
  \endgroup
}

\newcommand{\normdoublefigure}[3][]{%
  \begingroup
    %Default values:
    \setkeys[FigTools]{TwoFig}{ pos=!htp,
                                mainlabel={fig:#2-#3},
                                maincaption={},
                                mainshortcaption={},%
                                leftopt={},%width=0.45\textwidth},
                                leftlabel={fig:#2-left},
                                leftcaption={},
                                leftshortcaption={},%
                                rightopt={},%width=0.45\textwidth},
                                rightlabel={fig:#3-right},
                                rightcaption={},
                                rightshortcaption={},
                                spacing={\hfill}
                              }
    %User provided values:
    \setkeys[FigTools]{TwoFig}{#1}
    \def\efigure{\begin{figure}}%
    \expandafter\efigure\expandafter[\FtPos]
      \centering
			\subFloatCorrCaps{\FtLeftCaption}
	                     {\FtLeftShortCaption}
	                     {\FtLeftLabel}
	                     {\FtLeftOpt}
	                     {#2}
			\FtSpacing
			\subFloatCorrCaps{\FtRightCaption}
	                     {\FtRightShortCaption}
	                     {\FtRightLabel}
	                     {\FtRightOpt}
	                     {#3}
      \caption[\FtMainShortCaption]{\FtMainCaption}
      \label{\FtMainLabel}
    \end{figure}
  \endgroup
}

\newcommand{\normtriplefigure}[4][]{%
  \begingroup
    %Default values:
    \setkeys[FigTools]{ThreeFig}{ pos=!htp,
	                                mainlabel={fig:#2-#3},
	                                maincaption={},
	                                mainshortcaption={},%
	                                leftopt={},%width=0.45\textwidth},
	                                leftlabel={fig:#2-left},
	                                leftcaption={},
	                                leftshortcaption={},%
	                                midopt={},%width=0.45\textwidth},
	                                midlabel={fig:#3-mid},
	                                midcaption={},
	                                midshortcaption={},%
	                                rightopt={},%width=0.45\textwidth},
	                                rightlabel={fig:#4-right},
	                                rightcaption={},
	                                rightshortcaption={},
	                                spacing={\hfill},
	                                spacingtwo={}
	                              }
    %User provided values:
    \setkeys[FigTools]{ThreeFig}{#1}
    \def\efigure{\begin{figure}}%
    \expandafter\efigure\expandafter[\FtPos]
      \centering
			\subFloatCorrCaps{\FtLeftCaption}
	                     {\FtLeftShortCaption}
	                     {\FtLeftLabel}
	                     {\FtLeftOpt}
	                     {#2}
			\FtSpacing
			\subFloatCorrCaps{\FtMidCaption}
	                     {\FtMidShortCaption}
	                     {\FtMidLabel}
	                     {\FtMidOpt}
	                     {#3}
			\FtSpacing
			\FtSpacingTwo
			\subFloatCorrCaps{\FtRightCaption}
	                     {\FtRightShortCaption}
	                     {\FtRightLabel}
	                     {\FtRightOpt}
	                     {#4}
      \caption[\FtMainShortCaption]{\FtMainCaption}
      \label{\FtMainLabel}
    \end{figure}
  \endgroup
}
% ****************************************************************************************************  
% If you like the classicthesis, then I would appreciate a postcard. 
% My address can be found in the file ClassicThesis.pdf. A collection 
% of the postcards I received so far is available online at 
% http://postcards.miede.de
% ****************************************************************************************************


% ****************************************************************************************************
% 0. Set the encoding of your files. UTF-8 is the only sensible encoding nowadays. If you can't read
% äöüßáéçèê∂åëæƒÏ€ then change the encoding setting in your editor, not the line below. If your editor
% does not support utf8 use another editor!
% ****************************************************************************************************
\PassOptionsToPackage{utf8}{inputenc}
	\usepackage{inputenc}

% ****************************************************************************************************
% 1. Configure classicthesis for your needs here, e.g., remove "drafting" below 
% in order to deactivate the time-stamp on the pages
% ****************************************************************************************************
\PassOptionsToPackage{eulerchapternumbers,listings,drafting,%
					 pdfspacing,%floatperchapter,%linedheaders,%
					 subfig,beramono,eulermath,parts}{classicthesis}                                        
% ********************************************************************
% Available options for classicthesis.sty 
% (see ClassicThesis.pdf for more information):
% drafting
% parts nochapters linedheaders
% eulerchapternumbers beramono eulermath pdfspacing minionprospacing
% tocaligned dottedtoc manychapters
% listings floatperchapter subfig
% ********************************************************************


% ****************************************************************************************************
% 2. Personal data and user ad-hoc commands
% ****************************************************************************************************
\newcommand{\myTitle}{Bayesian Uncertainty Quantification of Physical Models in Thermal-Hydraulics System Codes\xspace}
\newcommand{\mySubtitle}{\xspace}
\newcommand{\myDegree}{ing. nucl. dipl. EPF\xspace}
\newcommand{\myName}{Damar Canggih Wicaksono\xspace}
\newcommand{\myProf}{Prof. Andreas Pautz\xspace}
\newcommand{\myOtherProf}{Put name here\xspace}
\newcommand{\mySupervisor}{Omar Zerkak\xspace}
\newcommand{\myFaculty}{Put data here\xspace}
\newcommand{\myDepartment}{Laboratory for Reactor Physics and Systems Behaviour\xspace}
\newcommand{\myUni}{École polytechnique fédérale de Lausanne\xspace}
\newcommand{\myLocation}{Lausanne\xspace}
\newcommand{\myTime}{April 2017\xspace}
\newcommand{\myVersion}{version 4.2\xspace}

% ********************************************************************
% Setup, finetuning, and useful commands
% ********************************************************************
\newcounter{dummy} % necessary for correct hyperlinks (to index, bib, etc.)
\newlength{\abcd} % for ab..z string length calculation
\providecommand{\mLyX}{L\kern-.1667em\lower.25em\hbox{Y}\kern-.125emX\@}
\newcommand{\ie}{i.\,e.}
\newcommand{\Ie}{I.\,e.}
\newcommand{\eg}{e.\,g.}
\newcommand{\Eg}{E.\,g.} 
% ****************************************************************************************************


% ****************************************************************************************************
% 3. Loading some handy packages
% ****************************************************************************************************
% ******************************************************************** 
% Packages with options that might require adjustments
% ******************************************************************** 
%\PassOptionsToPackage{ngerman,american}{babel}   % change this to your language(s)
% Spanish languages need extra options in order to work with this template
%\PassOptionsToPackage{spanish,es-lcroman}{babel}
	\usepackage{babel}                  

\usepackage{csquotes}
\PassOptionsToPackage{%
    %backend=biber, %instead of bibtex
	backend=bibtex8,bibencoding=ascii,%
	language=auto,%
	style=numeric-comp,%
    %style=authoryear-comp, % Author 1999, 2010
    %bibstyle=authoryear,dashed=false, % dashed: substitute rep. author with ---
    sorting=none, %nyts name, year, title
    maxbibnames=10, % default: 3, et al.
    %backref=true,%
    natbib=true % natbib compatibility mode (\citep and \citet still work)
}{biblatex}
    \usepackage{biblatex}

\PassOptionsToPackage{fleqn}{amsmath}       % math environments and more by the AMS 
    \usepackage{amsmath}

% ******************************************************************** 
% General useful packages
% ******************************************************************** 
\PassOptionsToPackage{T1}{fontenc} % T2A for cyrillics
    \usepackage{fontenc}     
\usepackage{textcomp} % fix warning with missing font shapes
\usepackage{scrhack} % fix warnings when using KOMA with listings package          
\usepackage{xspace} % to get the spacing after macros right  
\usepackage{mparhack} % get marginpar right
\usepackage{fixltx2e} % fixes some LaTeX stuff --> since 2015 in the LaTeX kernel (see below)
%\usepackage[latest]{latexrelease} % will be used once available in more distributions (ISSUE #107)
\PassOptionsToPackage{printonlyused,smaller}{acronym} 
    \usepackage{acronym} % nice macros for handling all acronyms in the thesis
    %\renewcommand{\bflabel}[1]{{#1}\hfill} % fix the list of acronyms --> no longer working
    %\renewcommand*{\acsfont}[1]{\textsc{#1}} 
    \renewcommand*{\aclabelfont}[1]{\acsfont{#1}}
% ****************************************************************************************************

% ****************************************************************************************************
% 4. Setup floats: tables, (sub)figures, and captions
% ****************************************************************************************************
\usepackage{tabularx} % better tables
    \setlength{\extrarowheight}{3pt} % increase table row height
\newcommand{\tableheadline}[1]{\multicolumn{1}{c}{\spacedlowsmallcaps{#1}}}
\newcommand{\myfloatalign}{\centering} % to be used with each float for alignment
\usepackage{caption}
% Thanks to cgnieder and Claus Lahiri
% http://tex.stackexchange.com/questions/69349/spacedlowsmallcaps-in-caption-label
% [REMOVED DUE TO OTHER PROBLEMS, SEE ISSUE #82]    
%\DeclareCaptionLabelFormat{smallcaps}{\bothIfFirst{#1}{~}\MakeTextLowercase{\textsc{#2}}}
%\captionsetup{font=small,labelformat=smallcaps} % format=hang,
\captionsetup{font=small} % format=hang,
\usepackage{subfig}  
% ****************************************************************************************************


% ****************************************************************************************************
% 5. Setup code listings
% ****************************************************************************************************
\usepackage{listings} 
%\lstset{emph={trueIndex,root},emphstyle=\color{BlueViolet}}%\underbar} % for special keywords
\lstset{language=[LaTeX]Tex,%C++,
    morekeywords={PassOptionsToPackage,selectlanguage},
    keywordstyle=\color{RoyalBlue},%\bfseries,
    basicstyle=\small\ttfamily,
    %identifierstyle=\color{NavyBlue},
    commentstyle=\color{Green}\ttfamily,
    stringstyle=\rmfamily,
    numbers=none,%left,%
    numberstyle=\scriptsize,%\tiny
    stepnumber=5,
    numbersep=8pt,
    showstringspaces=false,
    breaklines=true,
    %frameround=ftff,
    %frame=single,
    belowcaptionskip=.75\baselineskip
    %frame=L
} 
% ****************************************************************************************************             


% ****************************************************************************************************
% 6. PDFLaTeX, hyperreferences and citation backreferences
% ****************************************************************************************************
% ********************************************************************
% Using PDFLaTeX
% ********************************************************************
\PassOptionsToPackage{pdftex,hyperfootnotes=false,pdfpagelabels}{hyperref}
    \usepackage{hyperref}  % backref linktocpage pagebackref
\pdfcompresslevel=9
\pdfadjustspacing=1 
\PassOptionsToPackage{pdftex}{graphicx}
    \usepackage{graphicx} 
 

% ********************************************************************
% Hyperreferences
% ********************************************************************
\hypersetup{%
    %draft, % = no hyperlinking at all (useful in b/w printouts)
    colorlinks=true, linktocpage=true, pdfstartpage=3, pdfstartview=FitV,%
    % uncomment the following line if you want to have black links (e.g., for printing)
    %colorlinks=false, linktocpage=false, pdfstartpage=3, pdfstartview=FitV, pdfborder={0 0 0},%
    breaklinks=true, pdfpagemode=UseNone, pageanchor=true, pdfpagemode=UseOutlines,%
    plainpages=false, bookmarksnumbered, bookmarksopen=true, bookmarksopenlevel=1,%
    hypertexnames=true, pdfhighlight=/O,%nesting=true,%frenchlinks,%
    urlcolor=webbrown, linkcolor=RoyalBlue, citecolor=webgreen, %pagecolor=RoyalBlue,%
    %urlcolor=Black, linkcolor=Black, citecolor=Black, %pagecolor=Black,%
    pdftitle={\myTitle},%
    pdfauthor={\textcopyright\ \myName, \myUni, \myFaculty},%
    pdfsubject={},%
    pdfkeywords={},%
    pdfcreator={pdfLaTeX},%
    pdfproducer={LaTeX with hyperref and classicthesis}%
}   

% ********************************************************************
% Setup autoreferences
% ********************************************************************
% There are some issues regarding autorefnames
% http://www.ureader.de/msg/136221647.aspx
% http://www.tex.ac.uk/cgi-bin/texfaq2html?label=latexwords
% you have to redefine the makros for the 
% language you use, e.g., american, ngerman
% (as chosen when loading babel/AtBeginDocument)
% ********************************************************************
\makeatletter
\@ifpackageloaded{babel}%
    {%
       \addto\extrasamerican{%
			\renewcommand*{\figureautorefname}{Figure}%
			\renewcommand*{\tableautorefname}{Table}%
			\renewcommand*{\partautorefname}{Part}%
			\renewcommand*{\chapterautorefname}{Chapter}%
			\renewcommand*{\sectionautorefname}{Section}%
			\renewcommand*{\subsectionautorefname}{Section}%
			\renewcommand*{\subsubsectionautorefname}{Section}%     
                }%
       \addto\extrasngerman{% 
			\renewcommand*{\paragraphautorefname}{Absatz}%
			\renewcommand*{\subparagraphautorefname}{Unterabsatz}%
			\renewcommand*{\footnoteautorefname}{Fu\"snote}%
			\renewcommand*{\FancyVerbLineautorefname}{Zeile}%
			\renewcommand*{\theoremautorefname}{Theorem}%
			\renewcommand*{\appendixautorefname}{Anhang}%
			\renewcommand*{\equationautorefname}{Gleichung}%        
			\renewcommand*{\itemautorefname}{Punkt}%
                }%  
            % Fix to getting autorefs for subfigures right (thanks to Belinda Vogt for changing the definition)
            \providecommand{\subfigureautorefname}{\figureautorefname}%             
    }{\relax}
\makeatother


% ****************************************************************************************************
% 7. Last calls before the bar closes
% ****************************************************************************************************
% ********************************************************************
% Development Stuff
% ********************************************************************
\listfiles
%\PassOptionsToPackage{l2tabu,orthodox,abort}{nag}
%   \usepackage{nag}
%\PassOptionsToPackage{warning, all}{onlyamsmath}
%   \usepackage{onlyamsmath}

% ********************************************************************
% Last, but not least...
% ********************************************************************
\usepackage{classicthesis} 
% ****************************************************************************************************


% ****************************************************************************************************
% 8. Further adjustments (experimental)
% ****************************************************************************************************
% ********************************************************************
% Changing the text area
% ********************************************************************
%\linespread{1.05} % a bit more for Palatino
%\areaset[current]{312pt}{761pt} % 686 (factor 2.2) + 33 head + 42 head \the\footskip
%\setlength{\marginparwidth}{7em}%
%\setlength{\marginparsep}{2em}%

% ********************************************************************
% Using different fonts
% ********************************************************************
%\usepackage[oldstylenums]{kpfonts} % oldstyle notextcomp
%\usepackage[osf]{libertine}
%\usepackage[light,condensed,math]{iwona}
%\renewcommand{\sfdefault}{iwona}
%\usepackage{lmodern} % <-- no osf support :-(
%\usepackage{cfr-lm} % 
%\usepackage[urw-garamond]{mathdesign} <-- no osf support :-(
%\usepackage[default,osfigures]{opensans} % scale=0.95 
%\usepackage[sfdefault]{FiraSans}
% ****************************************************************************************************

% ***********************************************************************
% Additional WD41 packages
% ***********************************************************************
\usepackage{lipsum}
\usepackage{array}
\usepackage{ragged2e}
\usepackage{multirow}
\usepackage{chngcntr}
\usepackage{rotating}
\usepackage{algorithm}
\usepackage{algorithmic}
\usepackage[style=long,nolist,nonumberlist,toc,acronym]{glossaries}
\usepackage[roman]{parnotes}
\renewcommand{\thefootnote}{\fnsymbol{footnote}}
%%%%%%%%%%%%%%%%%%%%%%%%%%%%%%%%%%%%%%%%%%%%%%%%%%%%%%%%%%%%%%%%%%%%%%%%%%%%%%
%
% FIGURE TOOLS 
% (From Bejamin Hopfer's Thesis)
% http://benjaminhopfer.com/2014/04/16/typesetting-my-masters-thesis-in-latex/
%
%%%%%%%%%%%%%%%%%%%%%%%%%%%%%%%%%%%%%%%%%%%%%%%%%%%%%%%%%%%%%%%%%%%%%%%%%%%%%%
%%helper functions, no direct use
\usepackage{ifoddpage}

\newlength\fullmarginwidth
\fullmarginwidth=\marginparwidth
\advance\fullmarginwidth by \marginparsep

\newlength\fullwidth
\fullwidth=\textwidth
\advance\fullwidth by \fullmarginwidth


%                   #1   #2       #3     #4   #5
%\subFloatCorrCaps{Cap}{ShortCap}{Label}{Opt}{path}
\newcommand{\subFloatCorrCaps}[5]{
	\ifthenelse{\equal{#1}{}}%
  {
    %subfloat without captions
    \subfloat{%
      \expandafter\includegraphics\expandafter[#4]{#5}
      \label{#3}
    }
  }
  {
    \ifthenelse{\equal{#2}{}}%
    {
      %subfloat with same captions
      \subfloat[#1][#1]{%
        \expandafter\includegraphics\expandafter[#4]{#5}
        \label{#3}
      }
    }
    {
      %subfloat with different captions
      \subfloat[#2][#1]{%
        \expandafter\includegraphics\expandafter[#4]{#5}
        \label{#3}
      }
    }
  }
}

%Parameters: [keyvals (see below)]{file}{caption}
\makeatletter
  \define@key[FigTools]{OneFig}{pos}{\def\FtPos{#1}}
  \define@key[FigTools]{OneFig}{opt}{\def\FtOpt{#1}}
  \define@key[FigTools]{OneFig}{label}{\def\FtLabel{#1}}
  \define@key[FigTools]{OneFig}{shortcaption}{\def\FtShortCaption{#1}}
  
  \define@key[FigTools]{TwoFig}{pos}{\def\FtPos{#1}}
  \define@key[FigTools]{TwoFig}{mainlabel}{\def\FtMainLabel{#1}}
  \define@key[FigTools]{TwoFig}{maincaption}{\def\FtMainCaption{#1}}
  \define@key[FigTools]{TwoFig}{mainshortcaption}{\def\FtMainShortCaption{#1}}
  \define@key[FigTools]{TwoFig}{leftopt}{\def\FtLeftOpt{#1}}
  \define@key[FigTools]{TwoFig}{leftlabel}{\def\FtLeftLabel{#1}}
  \define@key[FigTools]{TwoFig}{leftcaption}{\def\FtLeftCaption{#1}}
  \define@key[FigTools]{TwoFig}{leftshortcaption}{\def\FtLeftShortCaption{#1}}
  \define@key[FigTools]{TwoFig}{rightopt}{\def\FtRightOpt{#1}}
  \define@key[FigTools]{TwoFig}{rightlabel}{\def\FtRightLabel{#1}}
  \define@key[FigTools]{TwoFig}{rightcaption}{\def\FtRightCaption{#1}}  
  \define@key[FigTools]{TwoFig}{rightshortcaption}{\def\FtRightShortCaption{#1}}    
  \define@key[FigTools]{TwoFig}{spacing}{\def\FtSpacing{#1}}    
  \define@key[FigTools]{TwoFig}{topspace}{\def\FtTopSpace{#1}}    
    

  \define@key[FigTools]{ThreeFig}{pos}{\def\FtPos{#1}}
  \define@key[FigTools]{ThreeFig}{mainlabel}{\def\FtMainLabel{#1}}
  \define@key[FigTools]{ThreeFig}{maincaption}{\def\FtMainCaption{#1}}
  \define@key[FigTools]{ThreeFig}{mainshortcaption}{\def\FtMainShortCaption{#1}}
  \define@key[FigTools]{ThreeFig}{leftopt}{\def\FtLeftOpt{#1}}
  \define@key[FigTools]{ThreeFig}{leftlabel}{\def\FtLeftLabel{#1}}
  \define@key[FigTools]{ThreeFig}{leftcaption}{\def\FtLeftCaption{#1}}
  \define@key[FigTools]{ThreeFig}{leftshortcaption}{\def\FtLeftShortCaption{#1}}
  \define@key[FigTools]{ThreeFig}{midopt}{\def\FtMidOpt{#1}}
  \define@key[FigTools]{ThreeFig}{midlabel}{\def\FtMidLabel{#1}}
  \define@key[FigTools]{ThreeFig}{midcaption}{\def\FtMidCaption{#1}}
  \define@key[FigTools]{ThreeFig}{midshortcaption}{\def\FtMidShortCaption{#1}}
  \define@key[FigTools]{ThreeFig}{rightopt}{\def\FtRightOpt{#1}}
  \define@key[FigTools]{ThreeFig}{rightlabel}{\def\FtRightLabel{#1}}
  \define@key[FigTools]{ThreeFig}{rightcaption}{\def\FtRightCaption{#1}}  
  \define@key[FigTools]{ThreeFig}{rightshortcaption}{\def\FtRightShortCaption{#1}}
  \define@key[FigTools]{ThreeFig}{spacing}{\def\FtSpacing{#1}}
  \define@key[FigTools]{ThreeFig}{spacingtwo}{\def\FtSpacingTwo{#1}}
\makeatother

\newcommand{\bigplot}[3][]{%
  \bigfloatinternal[#1]{#2}{#3}{plot}
}

\newcommand{\bigfigure}[3][]{%
  \bigfloatinternal[#1]{#2}{#3}{graph}
}

\newcommand{\bigfloatinternal}[4][]{%
  \begingroup
    \setkeys[FigTools]{OneFig}{ pos=!htp,
                                label={fig:#2},
                                shortcaption={#3},
                                opt={}%width=0.95\textwidth}
                              }
    \setkeys[FigTools]{OneFig}{#1}
    \def\efigure{\begin{figure}}%
    \expandafter\efigure\expandafter[\FtPos]
      \checkoddpage
      \edef\side{\ifoddpage l\else r\fi}
      \makebox[\textwidth][\side]{% 
        \begin{minipage}[t]{\fullwidth}
          \centering
          \ifthenelse{\equal{#4}{plot}}
      		{
          	\input{\evalDir{#2}}
      		}{
          	\expandafter\includegraphics\expandafter[\FtOpt]{#2}
          }
          \caption[\FtShortCaption]{#3}
          \label{\FtLabel}
        \end{minipage}
      }%
    \end{figure}
  \endgroup
}

\newcommand{\bigdoublefigure}[3][]{%
  \begingroup
    %Default values:
    \setkeys[FigTools]{TwoFig}{ pos=!htp,
                                mainlabel={fig:#2-#3},
                                maincaption={#2-#3},
                                mainshortcaption={},%
                                leftopt={},%0.45\textwidth,
                                leftlabel={fig:#2-left},
                                leftcaption={},
                                leftshortcaption={},%
                                rightopt={},%0.45\textwidth,
                                rightlabel={fig:#3-right},
                                rightcaption={},
                                rightshortcaption={},
                                spacing={\hfill},
                                topspace={}
                              }
    %User provided values:
    \setkeys[FigTools]{TwoFig}{#1}
    \def\efigure{\begin{figure}}%
    \expandafter\efigure\expandafter[\FtPos]
      \FtTopSpace
      %Check on which side whe are (right or left)
      \checkoddpage
      \edef\side{\ifoddpage l\else r\fi}
      %Ensure there will be no overfull box message
      \makebox[\textwidth][\side]{% 
        \begin{minipage}{\fullwidth}
          \centering
					\subFloatCorrCaps{\FtLeftCaption}
			                     {\FtLeftShortCaption}
			                     {\FtLeftLabel}
			                     {\FtLeftOpt}
			                     {#2}
			    \FtSpacing
					\subFloatCorrCaps{\FtRightCaption}
			                     {\FtRightShortCaption}
			                     {\FtRightLabel}
			                     {\FtRightOpt}
			                     {#3}
          \caption[\FtMainShortCaption]{\FtMainCaption}
          \label{\FtMainLabel}
        \end{minipage}
      }
    \end{figure}
  \endgroup
}

\newcommand{\bigtriplefigure}[4][]{%
  \begingroup
    %Default values:
    \setkeys[FigTools]{ThreeFig}{ pos=!htp,
	                                mainlabel={fig:#2-#3},
	                                maincaption={#2-#3},
	                                mainshortcaption={},%
	                                leftopt={},%0.45\textwidth,
	                                leftlabel={fig:#2-left},
	                                leftcaption={},
	                                leftshortcaption={},%
	                                midopt={},%0.45\textwidth,
	                                midlabel={fig:#3-mid},
	                                midcaption={},
	                                midshortcaption={},%
	                                rightopt={},%0.45\textwidth,
	                                rightlabel={fig:#4-right},
	                                rightcaption={},
	                                rightshortcaption={},
	                                spacing={\hfill},
	                                spacingtwo={}
	                              }
    %User provided values:
    \setkeys[FigTools]{ThreeFig}{#1}
    \def\efigure{\begin{figure}}%
    \expandafter\efigure\expandafter[\FtPos]
      %Check on which side whe are (right or left)
      \checkoddpage
      \edef\side{\ifoddpage l\else r\fi}
      %Ensure there will be no overfull box message
      \makebox[\textwidth][\side]{% 
        \begin{minipage}{\fullwidth}
          \centering
					\subFloatCorrCaps{\FtLeftCaption}
			                     {\FtLeftShortCaption}
			                     {\FtLeftLabel}
			                     {\FtLeftOpt}
			                     {#2}
			    \FtSpacing
					\subFloatCorrCaps{\FtMidCaption}
			                     {\FtMidShortCaption}
			                     {\FtMidLabel}
			                     {\FtMidOpt}
			                     {#3}
			    \FtSpacing
			    \FtSpacingTwo
					\subFloatCorrCaps{\FtRightCaption}
			                     {\FtRightShortCaption}
			                     {\FtRightLabel}
			                     {\FtRightOpt}
			                     {#4}
          \caption[\FtMainShortCaption]{\FtMainCaption}
          \label{\FtMainLabel}
        \end{minipage}
      }
    \end{figure}
  \endgroup
}

\newcommand{\normplot}[3][]{%
  \normfloatinternal[#1]{#2}{#3}{plot}
}

\newcommand{\normfigure}[3][]{%
  \normfloatinternal[#1]{#2}{#3}{graph}
}

\newcommand{\normfloatinternal}[4][]{%
  \begingroup
    \setkeys[FigTools]{OneFig}{ pos=!htp,
                                label={fig:#2},
                                shortcaption={#3},
                                opt={}%{width=0.95\textwidth}
                              }
    \setkeys[FigTools]{OneFig}{#1}
    \def\efigure{\begin{figure}}%
    \expandafter\efigure\expandafter[\FtPos]
      \centering
      \ifthenelse{\equal{#4}{plot}}
      {
        \input{\evalDir{#2}}
      }{
      	\expandafter\includegraphics\expandafter[\FtOpt]{#2}
      }
      \caption[\FtShortCaption]{#3}
      \label{\FtLabel}
    \end{figure}
  \endgroup
}

\newcommand{\normdoublefigure}[3][]{%
  \begingroup
    %Default values:
    \setkeys[FigTools]{TwoFig}{ pos=!htp,
                                mainlabel={fig:#2-#3},
                                maincaption={},
                                mainshortcaption={},%
                                leftopt={},%width=0.45\textwidth},
                                leftlabel={fig:#2-left},
                                leftcaption={},
                                leftshortcaption={},%
                                rightopt={},%width=0.45\textwidth},
                                rightlabel={fig:#3-right},
                                rightcaption={},
                                rightshortcaption={},
                                spacing={\hfill}
                              }
    %User provided values:
    \setkeys[FigTools]{TwoFig}{#1}
    \def\efigure{\begin{figure}}%
    \expandafter\efigure\expandafter[\FtPos]
      \centering
			\subFloatCorrCaps{\FtLeftCaption}
	                     {\FtLeftShortCaption}
	                     {\FtLeftLabel}
	                     {\FtLeftOpt}
	                     {#2}
			\FtSpacing
			\subFloatCorrCaps{\FtRightCaption}
	                     {\FtRightShortCaption}
	                     {\FtRightLabel}
	                     {\FtRightOpt}
	                     {#3}
      \caption[\FtMainShortCaption]{\FtMainCaption}
      \label{\FtMainLabel}
    \end{figure}
  \endgroup
}

\newcommand{\normtriplefigure}[4][]{%
  \begingroup
    %Default values:
    \setkeys[FigTools]{ThreeFig}{ pos=!htp,
	                                mainlabel={fig:#2-#3},
	                                maincaption={},
	                                mainshortcaption={},%
	                                leftopt={},%width=0.45\textwidth},
	                                leftlabel={fig:#2-left},
	                                leftcaption={},
	                                leftshortcaption={},%
	                                midopt={},%width=0.45\textwidth},
	                                midlabel={fig:#3-mid},
	                                midcaption={},
	                                midshortcaption={},%
	                                rightopt={},%width=0.45\textwidth},
	                                rightlabel={fig:#4-right},
	                                rightcaption={},
	                                rightshortcaption={},
	                                spacing={\hfill},
	                                spacingtwo={}
	                              }
    %User provided values:
    \setkeys[FigTools]{ThreeFig}{#1}
    \def\efigure{\begin{figure}}%
    \expandafter\efigure\expandafter[\FtPos]
      \centering
			\subFloatCorrCaps{\FtLeftCaption}
	                     {\FtLeftShortCaption}
	                     {\FtLeftLabel}
	                     {\FtLeftOpt}
	                     {#2}
			\FtSpacing
			\subFloatCorrCaps{\FtMidCaption}
	                     {\FtMidShortCaption}
	                     {\FtMidLabel}
	                     {\FtMidOpt}
	                     {#3}
			\FtSpacing
			\FtSpacingTwo
			\subFloatCorrCaps{\FtRightCaption}
	                     {\FtRightShortCaption}
	                     {\FtRightLabel}
	                     {\FtRightOpt}
	                     {#4}
      \caption[\FtMainShortCaption]{\FtMainCaption}
      \label{\FtMainLabel}
    \end{figure}
  \endgroup
}
% ****************************************************************************************************  
% If you like the classicthesis, then I would appreciate a postcard. 
% My address can be found in the file ClassicThesis.pdf. A collection 
% of the postcards I received so far is available online at 
% http://postcards.miede.de
% ****************************************************************************************************


% ****************************************************************************************************
% 0. Set the encoding of your files. UTF-8 is the only sensible encoding nowadays. If you can't read
% äöüßáéçèê∂åëæƒÏ€ then change the encoding setting in your editor, not the line below. If your editor
% does not support utf8 use another editor!
% ****************************************************************************************************
\PassOptionsToPackage{utf8}{inputenc}
	\usepackage{inputenc}

% ****************************************************************************************************
% 1. Configure classicthesis for your needs here, e.g., remove "drafting" below 
% in order to deactivate the time-stamp on the pages
% ****************************************************************************************************
\PassOptionsToPackage{eulerchapternumbers,listings,drafting,%
					 pdfspacing,%floatperchapter,%linedheaders,%
					 subfig,beramono,eulermath,parts}{classicthesis}                                        
% ********************************************************************
% Available options for classicthesis.sty 
% (see ClassicThesis.pdf for more information):
% drafting
% parts nochapters linedheaders
% eulerchapternumbers beramono eulermath pdfspacing minionprospacing
% tocaligned dottedtoc manychapters
% listings floatperchapter subfig
% ********************************************************************


% ****************************************************************************************************
% 2. Personal data and user ad-hoc commands
% ****************************************************************************************************
\newcommand{\myTitle}{Bayesian Uncertainty Quantification of Physical Models in Thermal-Hydraulics System Codes\xspace}
\newcommand{\mySubtitle}{\xspace}
\newcommand{\myDegree}{ing. nucl. dipl. EPF\xspace}
\newcommand{\myName}{Damar Canggih Wicaksono\xspace}
\newcommand{\myProf}{Prof. Andreas Pautz\xspace}
\newcommand{\myOtherProf}{Put name here\xspace}
\newcommand{\mySupervisor}{Omar Zerkak\xspace}
\newcommand{\myFaculty}{Put data here\xspace}
\newcommand{\myDepartment}{Laboratory for Reactor Physics and Systems Behaviour\xspace}
\newcommand{\myUni}{École polytechnique fédérale de Lausanne\xspace}
\newcommand{\myLocation}{Lausanne\xspace}
\newcommand{\myTime}{April 2017\xspace}
\newcommand{\myVersion}{version 4.2\xspace}

% ********************************************************************
% Setup, finetuning, and useful commands
% ********************************************************************
\newcounter{dummy} % necessary for correct hyperlinks (to index, bib, etc.)
\newlength{\abcd} % for ab..z string length calculation
\providecommand{\mLyX}{L\kern-.1667em\lower.25em\hbox{Y}\kern-.125emX\@}
\newcommand{\ie}{i.\,e.}
\newcommand{\Ie}{I.\,e.}
\newcommand{\eg}{e.\,g.}
\newcommand{\Eg}{E.\,g.} 
% ****************************************************************************************************


% ****************************************************************************************************
% 3. Loading some handy packages
% ****************************************************************************************************
% ******************************************************************** 
% Packages with options that might require adjustments
% ******************************************************************** 
%\PassOptionsToPackage{ngerman,american}{babel}   % change this to your language(s)
% Spanish languages need extra options in order to work with this template
%\PassOptionsToPackage{spanish,es-lcroman}{babel}
	\usepackage{babel}                  

\usepackage{csquotes}
\PassOptionsToPackage{%
    %backend=biber, %instead of bibtex
	backend=bibtex8,bibencoding=ascii,%
	language=auto,%
	style=numeric-comp,%
    %style=authoryear-comp, % Author 1999, 2010
    %bibstyle=authoryear,dashed=false, % dashed: substitute rep. author with ---
    sorting=none, %nyts name, year, title
    maxbibnames=10, % default: 3, et al.
    %backref=true,%
    natbib=true % natbib compatibility mode (\citep and \citet still work)
}{biblatex}
    \usepackage{biblatex}

\PassOptionsToPackage{fleqn}{amsmath}       % math environments and more by the AMS 
    \usepackage{amsmath}

% ******************************************************************** 
% General useful packages
% ******************************************************************** 
\PassOptionsToPackage{T1}{fontenc} % T2A for cyrillics
    \usepackage{fontenc}     
\usepackage{textcomp} % fix warning with missing font shapes
\usepackage{scrhack} % fix warnings when using KOMA with listings package          
\usepackage{xspace} % to get the spacing after macros right  
\usepackage{mparhack} % get marginpar right
\usepackage{fixltx2e} % fixes some LaTeX stuff --> since 2015 in the LaTeX kernel (see below)
%\usepackage[latest]{latexrelease} % will be used once available in more distributions (ISSUE #107)
\PassOptionsToPackage{printonlyused,smaller}{acronym} 
    \usepackage{acronym} % nice macros for handling all acronyms in the thesis
    %\renewcommand{\bflabel}[1]{{#1}\hfill} % fix the list of acronyms --> no longer working
    %\renewcommand*{\acsfont}[1]{\textsc{#1}} 
    \renewcommand*{\aclabelfont}[1]{\acsfont{#1}}
% ****************************************************************************************************

% ****************************************************************************************************
% 4. Setup floats: tables, (sub)figures, and captions
% ****************************************************************************************************
\usepackage{tabularx} % better tables
    \setlength{\extrarowheight}{3pt} % increase table row height
\newcommand{\tableheadline}[1]{\multicolumn{1}{c}{\spacedlowsmallcaps{#1}}}
\newcommand{\myfloatalign}{\centering} % to be used with each float for alignment
\usepackage{caption}
% Thanks to cgnieder and Claus Lahiri
% http://tex.stackexchange.com/questions/69349/spacedlowsmallcaps-in-caption-label
% [REMOVED DUE TO OTHER PROBLEMS, SEE ISSUE #82]    
%\DeclareCaptionLabelFormat{smallcaps}{\bothIfFirst{#1}{~}\MakeTextLowercase{\textsc{#2}}}
%\captionsetup{font=small,labelformat=smallcaps} % format=hang,
\captionsetup{font=small} % format=hang,
\usepackage{subfig}  
% ****************************************************************************************************


% ****************************************************************************************************
% 5. Setup code listings
% ****************************************************************************************************
\usepackage{listings} 
%\lstset{emph={trueIndex,root},emphstyle=\color{BlueViolet}}%\underbar} % for special keywords
\lstset{language=[LaTeX]Tex,%C++,
    morekeywords={PassOptionsToPackage,selectlanguage},
    keywordstyle=\color{RoyalBlue},%\bfseries,
    basicstyle=\small\ttfamily,
    %identifierstyle=\color{NavyBlue},
    commentstyle=\color{Green}\ttfamily,
    stringstyle=\rmfamily,
    numbers=none,%left,%
    numberstyle=\scriptsize,%\tiny
    stepnumber=5,
    numbersep=8pt,
    showstringspaces=false,
    breaklines=true,
    %frameround=ftff,
    %frame=single,
    belowcaptionskip=.75\baselineskip
    %frame=L
} 
% ****************************************************************************************************             


% ****************************************************************************************************
% 6. PDFLaTeX, hyperreferences and citation backreferences
% ****************************************************************************************************
% ********************************************************************
% Using PDFLaTeX
% ********************************************************************
\PassOptionsToPackage{pdftex,hyperfootnotes=false,pdfpagelabels}{hyperref}
    \usepackage{hyperref}  % backref linktocpage pagebackref
\pdfcompresslevel=9
\pdfadjustspacing=1 
\PassOptionsToPackage{pdftex}{graphicx}
    \usepackage{graphicx} 
 

% ********************************************************************
% Hyperreferences
% ********************************************************************
\hypersetup{%
    %draft, % = no hyperlinking at all (useful in b/w printouts)
    colorlinks=true, linktocpage=true, pdfstartpage=3, pdfstartview=FitV,%
    % uncomment the following line if you want to have black links (e.g., for printing)
    %colorlinks=false, linktocpage=false, pdfstartpage=3, pdfstartview=FitV, pdfborder={0 0 0},%
    breaklinks=true, pdfpagemode=UseNone, pageanchor=true, pdfpagemode=UseOutlines,%
    plainpages=false, bookmarksnumbered, bookmarksopen=true, bookmarksopenlevel=1,%
    hypertexnames=true, pdfhighlight=/O,%nesting=true,%frenchlinks,%
    urlcolor=webbrown, linkcolor=RoyalBlue, citecolor=webgreen, %pagecolor=RoyalBlue,%
    %urlcolor=Black, linkcolor=Black, citecolor=Black, %pagecolor=Black,%
    pdftitle={\myTitle},%
    pdfauthor={\textcopyright\ \myName, \myUni, \myFaculty},%
    pdfsubject={},%
    pdfkeywords={},%
    pdfcreator={pdfLaTeX},%
    pdfproducer={LaTeX with hyperref and classicthesis}%
}   

% ********************************************************************
% Setup autoreferences
% ********************************************************************
% There are some issues regarding autorefnames
% http://www.ureader.de/msg/136221647.aspx
% http://www.tex.ac.uk/cgi-bin/texfaq2html?label=latexwords
% you have to redefine the makros for the 
% language you use, e.g., american, ngerman
% (as chosen when loading babel/AtBeginDocument)
% ********************************************************************
\makeatletter
\@ifpackageloaded{babel}%
    {%
       \addto\extrasamerican{%
			\renewcommand*{\figureautorefname}{Figure}%
			\renewcommand*{\tableautorefname}{Table}%
			\renewcommand*{\partautorefname}{Part}%
			\renewcommand*{\chapterautorefname}{Chapter}%
			\renewcommand*{\sectionautorefname}{Section}%
			\renewcommand*{\subsectionautorefname}{Section}%
			\renewcommand*{\subsubsectionautorefname}{Section}%     
                }%
       \addto\extrasngerman{% 
			\renewcommand*{\paragraphautorefname}{Absatz}%
			\renewcommand*{\subparagraphautorefname}{Unterabsatz}%
			\renewcommand*{\footnoteautorefname}{Fu\"snote}%
			\renewcommand*{\FancyVerbLineautorefname}{Zeile}%
			\renewcommand*{\theoremautorefname}{Theorem}%
			\renewcommand*{\appendixautorefname}{Anhang}%
			\renewcommand*{\equationautorefname}{Gleichung}%        
			\renewcommand*{\itemautorefname}{Punkt}%
                }%  
            % Fix to getting autorefs for subfigures right (thanks to Belinda Vogt for changing the definition)
            \providecommand{\subfigureautorefname}{\figureautorefname}%             
    }{\relax}
\makeatother


% ****************************************************************************************************
% 7. Last calls before the bar closes
% ****************************************************************************************************
% ********************************************************************
% Development Stuff
% ********************************************************************
\listfiles
%\PassOptionsToPackage{l2tabu,orthodox,abort}{nag}
%   \usepackage{nag}
%\PassOptionsToPackage{warning, all}{onlyamsmath}
%   \usepackage{onlyamsmath}

% ********************************************************************
% Last, but not least...
% ********************************************************************
\usepackage{classicthesis} 
% ****************************************************************************************************


% ****************************************************************************************************
% 8. Further adjustments (experimental)
% ****************************************************************************************************
% ********************************************************************
% Changing the text area
% ********************************************************************
%\linespread{1.05} % a bit more for Palatino
%\areaset[current]{312pt}{761pt} % 686 (factor 2.2) + 33 head + 42 head \the\footskip
%\setlength{\marginparwidth}{7em}%
%\setlength{\marginparsep}{2em}%

% ********************************************************************
% Using different fonts
% ********************************************************************
%\usepackage[oldstylenums]{kpfonts} % oldstyle notextcomp
%\usepackage[osf]{libertine}
%\usepackage[light,condensed,math]{iwona}
%\renewcommand{\sfdefault}{iwona}
%\usepackage{lmodern} % <-- no osf support :-(
%\usepackage{cfr-lm} % 
%\usepackage[urw-garamond]{mathdesign} <-- no osf support :-(
%\usepackage[default,osfigures]{opensans} % scale=0.95 
%\usepackage[sfdefault]{FiraSans}
% ****************************************************************************************************

% ***********************************************************************
% Additional WD41 packages
% ***********************************************************************
\usepackage{lipsum}
\usepackage{array}
\usepackage{ragged2e}
\usepackage{multirow}
\usepackage{chngcntr}
\usepackage{rotating}
\usepackage{algorithm}
\usepackage{algorithmic}
\usepackage[style=long,nolist,nonumberlist,toc,acronym]{glossaries}
\usepackage[roman]{parnotes}
\renewcommand{\thefootnote}{\fnsymbol{footnote}}
%%%%%%%%%%%%%%%%%%%%%%%%%%%%%%%%%%%%%%%%%%%%%%%%%%%%%%%%%%%%%%%%%%%%%%%%%%%%%%
%
% FIGURE TOOLS 
% (From Bejamin Hopfer's Thesis)
% http://benjaminhopfer.com/2014/04/16/typesetting-my-masters-thesis-in-latex/
%
%%%%%%%%%%%%%%%%%%%%%%%%%%%%%%%%%%%%%%%%%%%%%%%%%%%%%%%%%%%%%%%%%%%%%%%%%%%%%%
%%helper functions, no direct use
\usepackage{ifoddpage}

\newlength\fullmarginwidth
\fullmarginwidth=\marginparwidth
\advance\fullmarginwidth by \marginparsep

\newlength\fullwidth
\fullwidth=\textwidth
\advance\fullwidth by \fullmarginwidth


%                   #1   #2       #3     #4   #5
%\subFloatCorrCaps{Cap}{ShortCap}{Label}{Opt}{path}
\newcommand{\subFloatCorrCaps}[5]{
	\ifthenelse{\equal{#1}{}}%
  {
    %subfloat without captions
    \subfloat{%
      \expandafter\includegraphics\expandafter[#4]{#5}
      \label{#3}
    }
  }
  {
    \ifthenelse{\equal{#2}{}}%
    {
      %subfloat with same captions
      \subfloat[#1][#1]{%
        \expandafter\includegraphics\expandafter[#4]{#5}
        \label{#3}
      }
    }
    {
      %subfloat with different captions
      \subfloat[#2][#1]{%
        \expandafter\includegraphics\expandafter[#4]{#5}
        \label{#3}
      }
    }
  }
}

%Parameters: [keyvals (see below)]{file}{caption}
\makeatletter
  \define@key[FigTools]{OneFig}{pos}{\def\FtPos{#1}}
  \define@key[FigTools]{OneFig}{opt}{\def\FtOpt{#1}}
  \define@key[FigTools]{OneFig}{label}{\def\FtLabel{#1}}
  \define@key[FigTools]{OneFig}{shortcaption}{\def\FtShortCaption{#1}}
  
  \define@key[FigTools]{TwoFig}{pos}{\def\FtPos{#1}}
  \define@key[FigTools]{TwoFig}{mainlabel}{\def\FtMainLabel{#1}}
  \define@key[FigTools]{TwoFig}{maincaption}{\def\FtMainCaption{#1}}
  \define@key[FigTools]{TwoFig}{mainshortcaption}{\def\FtMainShortCaption{#1}}
  \define@key[FigTools]{TwoFig}{leftopt}{\def\FtLeftOpt{#1}}
  \define@key[FigTools]{TwoFig}{leftlabel}{\def\FtLeftLabel{#1}}
  \define@key[FigTools]{TwoFig}{leftcaption}{\def\FtLeftCaption{#1}}
  \define@key[FigTools]{TwoFig}{leftshortcaption}{\def\FtLeftShortCaption{#1}}
  \define@key[FigTools]{TwoFig}{rightopt}{\def\FtRightOpt{#1}}
  \define@key[FigTools]{TwoFig}{rightlabel}{\def\FtRightLabel{#1}}
  \define@key[FigTools]{TwoFig}{rightcaption}{\def\FtRightCaption{#1}}  
  \define@key[FigTools]{TwoFig}{rightshortcaption}{\def\FtRightShortCaption{#1}}    
  \define@key[FigTools]{TwoFig}{spacing}{\def\FtSpacing{#1}}    
  \define@key[FigTools]{TwoFig}{topspace}{\def\FtTopSpace{#1}}    
    

  \define@key[FigTools]{ThreeFig}{pos}{\def\FtPos{#1}}
  \define@key[FigTools]{ThreeFig}{mainlabel}{\def\FtMainLabel{#1}}
  \define@key[FigTools]{ThreeFig}{maincaption}{\def\FtMainCaption{#1}}
  \define@key[FigTools]{ThreeFig}{mainshortcaption}{\def\FtMainShortCaption{#1}}
  \define@key[FigTools]{ThreeFig}{leftopt}{\def\FtLeftOpt{#1}}
  \define@key[FigTools]{ThreeFig}{leftlabel}{\def\FtLeftLabel{#1}}
  \define@key[FigTools]{ThreeFig}{leftcaption}{\def\FtLeftCaption{#1}}
  \define@key[FigTools]{ThreeFig}{leftshortcaption}{\def\FtLeftShortCaption{#1}}
  \define@key[FigTools]{ThreeFig}{midopt}{\def\FtMidOpt{#1}}
  \define@key[FigTools]{ThreeFig}{midlabel}{\def\FtMidLabel{#1}}
  \define@key[FigTools]{ThreeFig}{midcaption}{\def\FtMidCaption{#1}}
  \define@key[FigTools]{ThreeFig}{midshortcaption}{\def\FtMidShortCaption{#1}}
  \define@key[FigTools]{ThreeFig}{rightopt}{\def\FtRightOpt{#1}}
  \define@key[FigTools]{ThreeFig}{rightlabel}{\def\FtRightLabel{#1}}
  \define@key[FigTools]{ThreeFig}{rightcaption}{\def\FtRightCaption{#1}}  
  \define@key[FigTools]{ThreeFig}{rightshortcaption}{\def\FtRightShortCaption{#1}}
  \define@key[FigTools]{ThreeFig}{spacing}{\def\FtSpacing{#1}}
  \define@key[FigTools]{ThreeFig}{spacingtwo}{\def\FtSpacingTwo{#1}}
\makeatother

\newcommand{\bigplot}[3][]{%
  \bigfloatinternal[#1]{#2}{#3}{plot}
}

\newcommand{\bigfigure}[3][]{%
  \bigfloatinternal[#1]{#2}{#3}{graph}
}

\newcommand{\bigfloatinternal}[4][]{%
  \begingroup
    \setkeys[FigTools]{OneFig}{ pos=!htp,
                                label={fig:#2},
                                shortcaption={#3},
                                opt={}%width=0.95\textwidth}
                              }
    \setkeys[FigTools]{OneFig}{#1}
    \def\efigure{\begin{figure}}%
    \expandafter\efigure\expandafter[\FtPos]
      \checkoddpage
      \edef\side{\ifoddpage l\else r\fi}
      \makebox[\textwidth][\side]{% 
        \begin{minipage}[t]{\fullwidth}
          \centering
          \ifthenelse{\equal{#4}{plot}}
      		{
          	\input{\evalDir{#2}}
      		}{
          	\expandafter\includegraphics\expandafter[\FtOpt]{#2}
          }
          \caption[\FtShortCaption]{#3}
          \label{\FtLabel}
        \end{minipage}
      }%
    \end{figure}
  \endgroup
}

\newcommand{\bigdoublefigure}[3][]{%
  \begingroup
    %Default values:
    \setkeys[FigTools]{TwoFig}{ pos=!htp,
                                mainlabel={fig:#2-#3},
                                maincaption={#2-#3},
                                mainshortcaption={},%
                                leftopt={},%0.45\textwidth,
                                leftlabel={fig:#2-left},
                                leftcaption={},
                                leftshortcaption={},%
                                rightopt={},%0.45\textwidth,
                                rightlabel={fig:#3-right},
                                rightcaption={},
                                rightshortcaption={},
                                spacing={\hfill},
                                topspace={}
                              }
    %User provided values:
    \setkeys[FigTools]{TwoFig}{#1}
    \def\efigure{\begin{figure}}%
    \expandafter\efigure\expandafter[\FtPos]
      \FtTopSpace
      %Check on which side whe are (right or left)
      \checkoddpage
      \edef\side{\ifoddpage l\else r\fi}
      %Ensure there will be no overfull box message
      \makebox[\textwidth][\side]{% 
        \begin{minipage}{\fullwidth}
          \centering
					\subFloatCorrCaps{\FtLeftCaption}
			                     {\FtLeftShortCaption}
			                     {\FtLeftLabel}
			                     {\FtLeftOpt}
			                     {#2}
			    \FtSpacing
					\subFloatCorrCaps{\FtRightCaption}
			                     {\FtRightShortCaption}
			                     {\FtRightLabel}
			                     {\FtRightOpt}
			                     {#3}
          \caption[\FtMainShortCaption]{\FtMainCaption}
          \label{\FtMainLabel}
        \end{minipage}
      }
    \end{figure}
  \endgroup
}

\newcommand{\bigtriplefigure}[4][]{%
  \begingroup
    %Default values:
    \setkeys[FigTools]{ThreeFig}{ pos=!htp,
	                                mainlabel={fig:#2-#3},
	                                maincaption={#2-#3},
	                                mainshortcaption={},%
	                                leftopt={},%0.45\textwidth,
	                                leftlabel={fig:#2-left},
	                                leftcaption={},
	                                leftshortcaption={},%
	                                midopt={},%0.45\textwidth,
	                                midlabel={fig:#3-mid},
	                                midcaption={},
	                                midshortcaption={},%
	                                rightopt={},%0.45\textwidth,
	                                rightlabel={fig:#4-right},
	                                rightcaption={},
	                                rightshortcaption={},
	                                spacing={\hfill},
	                                spacingtwo={}
	                              }
    %User provided values:
    \setkeys[FigTools]{ThreeFig}{#1}
    \def\efigure{\begin{figure}}%
    \expandafter\efigure\expandafter[\FtPos]
      %Check on which side whe are (right or left)
      \checkoddpage
      \edef\side{\ifoddpage l\else r\fi}
      %Ensure there will be no overfull box message
      \makebox[\textwidth][\side]{% 
        \begin{minipage}{\fullwidth}
          \centering
					\subFloatCorrCaps{\FtLeftCaption}
			                     {\FtLeftShortCaption}
			                     {\FtLeftLabel}
			                     {\FtLeftOpt}
			                     {#2}
			    \FtSpacing
					\subFloatCorrCaps{\FtMidCaption}
			                     {\FtMidShortCaption}
			                     {\FtMidLabel}
			                     {\FtMidOpt}
			                     {#3}
			    \FtSpacing
			    \FtSpacingTwo
					\subFloatCorrCaps{\FtRightCaption}
			                     {\FtRightShortCaption}
			                     {\FtRightLabel}
			                     {\FtRightOpt}
			                     {#4}
          \caption[\FtMainShortCaption]{\FtMainCaption}
          \label{\FtMainLabel}
        \end{minipage}
      }
    \end{figure}
  \endgroup
}

\newcommand{\normplot}[3][]{%
  \normfloatinternal[#1]{#2}{#3}{plot}
}

\newcommand{\normfigure}[3][]{%
  \normfloatinternal[#1]{#2}{#3}{graph}
}

\newcommand{\normfloatinternal}[4][]{%
  \begingroup
    \setkeys[FigTools]{OneFig}{ pos=!htp,
                                label={fig:#2},
                                shortcaption={#3},
                                opt={}%{width=0.95\textwidth}
                              }
    \setkeys[FigTools]{OneFig}{#1}
    \def\efigure{\begin{figure}}%
    \expandafter\efigure\expandafter[\FtPos]
      \centering
      \ifthenelse{\equal{#4}{plot}}
      {
        \input{\evalDir{#2}}
      }{
      	\expandafter\includegraphics\expandafter[\FtOpt]{#2}
      }
      \caption[\FtShortCaption]{#3}
      \label{\FtLabel}
    \end{figure}
  \endgroup
}

\newcommand{\normdoublefigure}[3][]{%
  \begingroup
    %Default values:
    \setkeys[FigTools]{TwoFig}{ pos=!htp,
                                mainlabel={fig:#2-#3},
                                maincaption={},
                                mainshortcaption={},%
                                leftopt={},%width=0.45\textwidth},
                                leftlabel={fig:#2-left},
                                leftcaption={},
                                leftshortcaption={},%
                                rightopt={},%width=0.45\textwidth},
                                rightlabel={fig:#3-right},
                                rightcaption={},
                                rightshortcaption={},
                                spacing={\hfill}
                              }
    %User provided values:
    \setkeys[FigTools]{TwoFig}{#1}
    \def\efigure{\begin{figure}}%
    \expandafter\efigure\expandafter[\FtPos]
      \centering
			\subFloatCorrCaps{\FtLeftCaption}
	                     {\FtLeftShortCaption}
	                     {\FtLeftLabel}
	                     {\FtLeftOpt}
	                     {#2}
			\FtSpacing
			\subFloatCorrCaps{\FtRightCaption}
	                     {\FtRightShortCaption}
	                     {\FtRightLabel}
	                     {\FtRightOpt}
	                     {#3}
      \caption[\FtMainShortCaption]{\FtMainCaption}
      \label{\FtMainLabel}
    \end{figure}
  \endgroup
}

\newcommand{\normtriplefigure}[4][]{%
  \begingroup
    %Default values:
    \setkeys[FigTools]{ThreeFig}{ pos=!htp,
	                                mainlabel={fig:#2-#3},
	                                maincaption={},
	                                mainshortcaption={},%
	                                leftopt={},%width=0.45\textwidth},
	                                leftlabel={fig:#2-left},
	                                leftcaption={},
	                                leftshortcaption={},%
	                                midopt={},%width=0.45\textwidth},
	                                midlabel={fig:#3-mid},
	                                midcaption={},
	                                midshortcaption={},%
	                                rightopt={},%width=0.45\textwidth},
	                                rightlabel={fig:#4-right},
	                                rightcaption={},
	                                rightshortcaption={},
	                                spacing={\hfill},
	                                spacingtwo={}
	                              }
    %User provided values:
    \setkeys[FigTools]{ThreeFig}{#1}
    \def\efigure{\begin{figure}}%
    \expandafter\efigure\expandafter[\FtPos]
      \centering
			\subFloatCorrCaps{\FtLeftCaption}
	                     {\FtLeftShortCaption}
	                     {\FtLeftLabel}
	                     {\FtLeftOpt}
	                     {#2}
			\FtSpacing
			\subFloatCorrCaps{\FtMidCaption}
	                     {\FtMidShortCaption}
	                     {\FtMidLabel}
	                     {\FtMidOpt}
	                     {#3}
			\FtSpacing
			\FtSpacingTwo
			\subFloatCorrCaps{\FtRightCaption}
	                     {\FtRightShortCaption}
	                     {\FtRightLabel}
	                     {\FtRightOpt}
	                     {#4}
      \caption[\FtMainShortCaption]{\FtMainCaption}
      \label{\FtMainLabel}
    \end{figure}
  \endgroup
}
% ****************************************************************************************************  
% If you like the classicthesis, then I would appreciate a postcard. 
% My address can be found in the file ClassicThesis.pdf. A collection 
% of the postcards I received so far is available online at 
% http://postcards.miede.de
% ****************************************************************************************************


% ****************************************************************************************************
% 0. Set the encoding of your files. UTF-8 is the only sensible encoding nowadays. If you can't read
% äöüßáéçèê∂åëæƒÏ€ then change the encoding setting in your editor, not the line below. If your editor
% does not support utf8 use another editor!
% ****************************************************************************************************
\PassOptionsToPackage{utf8}{inputenc}
	\usepackage{inputenc}

% ****************************************************************************************************
% 1. Configure classicthesis for your needs here, e.g., remove "drafting" below 
% in order to deactivate the time-stamp on the pages
% ****************************************************************************************************
\PassOptionsToPackage{eulerchapternumbers,listings,drafting,%
					 pdfspacing,%floatperchapter,%linedheaders,%
					 subfig,beramono,eulermath,parts}{classicthesis}                                        
% ********************************************************************
% Available options for classicthesis.sty 
% (see ClassicThesis.pdf for more information):
% drafting
% parts nochapters linedheaders
% eulerchapternumbers beramono eulermath pdfspacing minionprospacing
% tocaligned dottedtoc manychapters
% listings floatperchapter subfig
% ********************************************************************


% ****************************************************************************************************
% 2. Personal data and user ad-hoc commands
% ****************************************************************************************************
\newcommand{\myTitle}{Bayesian Uncertainty Quantification of Physical Models in Thermal-Hydraulics System Codes\xspace}
\newcommand{\mySubtitle}{\xspace}
\newcommand{\myDegree}{ing. nucl. dipl. EPF\xspace}
\newcommand{\myName}{Damar Canggih Wicaksono\xspace}
\newcommand{\myProf}{Prof. Andreas Pautz\xspace}
\newcommand{\myOtherProf}{Put name here\xspace}
\newcommand{\mySupervisor}{Omar Zerkak\xspace}
\newcommand{\myFaculty}{Put data here\xspace}
\newcommand{\myDepartment}{Laboratory for Reactor Physics and Systems Behaviour\xspace}
\newcommand{\myUni}{École polytechnique fédérale de Lausanne\xspace}
\newcommand{\myLocation}{Lausanne\xspace}
\newcommand{\myTime}{April 2017\xspace}
\newcommand{\myVersion}{version 4.2\xspace}

% ********************************************************************
% Setup, finetuning, and useful commands
% ********************************************************************
\newcounter{dummy} % necessary for correct hyperlinks (to index, bib, etc.)
\newlength{\abcd} % for ab..z string length calculation
\providecommand{\mLyX}{L\kern-.1667em\lower.25em\hbox{Y}\kern-.125emX\@}
\newcommand{\ie}{i.\,e.}
\newcommand{\Ie}{I.\,e.}
\newcommand{\eg}{e.\,g.}
\newcommand{\Eg}{E.\,g.} 
% ****************************************************************************************************


% ****************************************************************************************************
% 3. Loading some handy packages
% ****************************************************************************************************
% ******************************************************************** 
% Packages with options that might require adjustments
% ******************************************************************** 
%\PassOptionsToPackage{ngerman,american}{babel}   % change this to your language(s)
% Spanish languages need extra options in order to work with this template
%\PassOptionsToPackage{spanish,es-lcroman}{babel}
	\usepackage{babel}                  

\usepackage{csquotes}
\PassOptionsToPackage{%
    %backend=biber, %instead of bibtex
	backend=bibtex8,bibencoding=ascii,%
	language=auto,%
	style=numeric-comp,%
    %style=authoryear-comp, % Author 1999, 2010
    %bibstyle=authoryear,dashed=false, % dashed: substitute rep. author with ---
    sorting=none, %nyts name, year, title
    maxbibnames=10, % default: 3, et al.
    %backref=true,%
    natbib=true % natbib compatibility mode (\citep and \citet still work)
}{biblatex}
    \usepackage{biblatex}

\PassOptionsToPackage{fleqn}{amsmath}       % math environments and more by the AMS 
    \usepackage{amsmath}

% ******************************************************************** 
% General useful packages
% ******************************************************************** 
\PassOptionsToPackage{T1}{fontenc} % T2A for cyrillics
    \usepackage{fontenc}     
\usepackage{textcomp} % fix warning with missing font shapes
\usepackage{scrhack} % fix warnings when using KOMA with listings package          
\usepackage{xspace} % to get the spacing after macros right  
\usepackage{mparhack} % get marginpar right
\usepackage{fixltx2e} % fixes some LaTeX stuff --> since 2015 in the LaTeX kernel (see below)
%\usepackage[latest]{latexrelease} % will be used once available in more distributions (ISSUE #107)
\PassOptionsToPackage{printonlyused,smaller}{acronym} 
    \usepackage{acronym} % nice macros for handling all acronyms in the thesis
    %\renewcommand{\bflabel}[1]{{#1}\hfill} % fix the list of acronyms --> no longer working
    %\renewcommand*{\acsfont}[1]{\textsc{#1}} 
    \renewcommand*{\aclabelfont}[1]{\acsfont{#1}}
% ****************************************************************************************************

% ****************************************************************************************************
% 4. Setup floats: tables, (sub)figures, and captions
% ****************************************************************************************************
\usepackage{tabularx} % better tables
    \setlength{\extrarowheight}{3pt} % increase table row height
\newcommand{\tableheadline}[1]{\multicolumn{1}{c}{\spacedlowsmallcaps{#1}}}
\newcommand{\myfloatalign}{\centering} % to be used with each float for alignment
\usepackage{caption}
% Thanks to cgnieder and Claus Lahiri
% http://tex.stackexchange.com/questions/69349/spacedlowsmallcaps-in-caption-label
% [REMOVED DUE TO OTHER PROBLEMS, SEE ISSUE #82]    
%\DeclareCaptionLabelFormat{smallcaps}{\bothIfFirst{#1}{~}\MakeTextLowercase{\textsc{#2}}}
%\captionsetup{font=small,labelformat=smallcaps} % format=hang,
\captionsetup{font=small} % format=hang,
\usepackage{subfig}  
% ****************************************************************************************************


% ****************************************************************************************************
% 5. Setup code listings
% ****************************************************************************************************
\usepackage{listings} 
%\lstset{emph={trueIndex,root},emphstyle=\color{BlueViolet}}%\underbar} % for special keywords
\lstset{language=[LaTeX]Tex,%C++,
    morekeywords={PassOptionsToPackage,selectlanguage},
    keywordstyle=\color{RoyalBlue},%\bfseries,
    basicstyle=\small\ttfamily,
    %identifierstyle=\color{NavyBlue},
    commentstyle=\color{Green}\ttfamily,
    stringstyle=\rmfamily,
    numbers=none,%left,%
    numberstyle=\scriptsize,%\tiny
    stepnumber=5,
    numbersep=8pt,
    showstringspaces=false,
    breaklines=true,
    %frameround=ftff,
    %frame=single,
    belowcaptionskip=.75\baselineskip
    %frame=L
} 
% ****************************************************************************************************             


% ****************************************************************************************************
% 6. PDFLaTeX, hyperreferences and citation backreferences
% ****************************************************************************************************
% ********************************************************************
% Using PDFLaTeX
% ********************************************************************
\PassOptionsToPackage{pdftex,hyperfootnotes=false,pdfpagelabels}{hyperref}
    \usepackage{hyperref}  % backref linktocpage pagebackref
\pdfcompresslevel=9
\pdfadjustspacing=1 
\PassOptionsToPackage{pdftex}{graphicx}
    \usepackage{graphicx} 
 

% ********************************************************************
% Hyperreferences
% ********************************************************************
\hypersetup{%
    %draft, % = no hyperlinking at all (useful in b/w printouts)
    colorlinks=true, linktocpage=true, pdfstartpage=3, pdfstartview=FitV,%
    % uncomment the following line if you want to have black links (e.g., for printing)
    %colorlinks=false, linktocpage=false, pdfstartpage=3, pdfstartview=FitV, pdfborder={0 0 0},%
    breaklinks=true, pdfpagemode=UseNone, pageanchor=true, pdfpagemode=UseOutlines,%
    plainpages=false, bookmarksnumbered, bookmarksopen=true, bookmarksopenlevel=1,%
    hypertexnames=true, pdfhighlight=/O,%nesting=true,%frenchlinks,%
    urlcolor=webbrown, linkcolor=RoyalBlue, citecolor=webgreen, %pagecolor=RoyalBlue,%
    %urlcolor=Black, linkcolor=Black, citecolor=Black, %pagecolor=Black,%
    pdftitle={\myTitle},%
    pdfauthor={\textcopyright\ \myName, \myUni, \myFaculty},%
    pdfsubject={},%
    pdfkeywords={},%
    pdfcreator={pdfLaTeX},%
    pdfproducer={LaTeX with hyperref and classicthesis}%
}   

% ********************************************************************
% Setup autoreferences
% ********************************************************************
% There are some issues regarding autorefnames
% http://www.ureader.de/msg/136221647.aspx
% http://www.tex.ac.uk/cgi-bin/texfaq2html?label=latexwords
% you have to redefine the makros for the 
% language you use, e.g., american, ngerman
% (as chosen when loading babel/AtBeginDocument)
% ********************************************************************
\makeatletter
\@ifpackageloaded{babel}%
    {%
       \addto\extrasamerican{%
			\renewcommand*{\figureautorefname}{Figure}%
			\renewcommand*{\tableautorefname}{Table}%
			\renewcommand*{\partautorefname}{Part}%
			\renewcommand*{\chapterautorefname}{Chapter}%
			\renewcommand*{\sectionautorefname}{Section}%
			\renewcommand*{\subsectionautorefname}{Section}%
			\renewcommand*{\subsubsectionautorefname}{Section}%     
                }%
       \addto\extrasngerman{% 
			\renewcommand*{\paragraphautorefname}{Absatz}%
			\renewcommand*{\subparagraphautorefname}{Unterabsatz}%
			\renewcommand*{\footnoteautorefname}{Fu\"snote}%
			\renewcommand*{\FancyVerbLineautorefname}{Zeile}%
			\renewcommand*{\theoremautorefname}{Theorem}%
			\renewcommand*{\appendixautorefname}{Anhang}%
			\renewcommand*{\equationautorefname}{Gleichung}%        
			\renewcommand*{\itemautorefname}{Punkt}%
                }%  
            % Fix to getting autorefs for subfigures right (thanks to Belinda Vogt for changing the definition)
            \providecommand{\subfigureautorefname}{\figureautorefname}%             
    }{\relax}
\makeatother


% ****************************************************************************************************
% 7. Last calls before the bar closes
% ****************************************************************************************************
% ********************************************************************
% Development Stuff
% ********************************************************************
\listfiles
%\PassOptionsToPackage{l2tabu,orthodox,abort}{nag}
%   \usepackage{nag}
%\PassOptionsToPackage{warning, all}{onlyamsmath}
%   \usepackage{onlyamsmath}

% ********************************************************************
% Last, but not least...
% ********************************************************************
\usepackage{classicthesis} 
% ****************************************************************************************************


% ****************************************************************************************************
% 8. Further adjustments (experimental)
% ****************************************************************************************************
% ********************************************************************
% Changing the text area
% ********************************************************************
%\linespread{1.05} % a bit more for Palatino
%\areaset[current]{312pt}{761pt} % 686 (factor 2.2) + 33 head + 42 head \the\footskip
%\setlength{\marginparwidth}{7em}%
%\setlength{\marginparsep}{2em}%

% ********************************************************************
% Using different fonts
% ********************************************************************
%\usepackage[oldstylenums]{kpfonts} % oldstyle notextcomp
%\usepackage[osf]{libertine}
%\usepackage[light,condensed,math]{iwona}
%\renewcommand{\sfdefault}{iwona}
%\usepackage{lmodern} % <-- no osf support :-(
%\usepackage{cfr-lm} % 
%\usepackage[urw-garamond]{mathdesign} <-- no osf support :-(
%\usepackage[default,osfigures]{opensans} % scale=0.95 
%\usepackage[sfdefault]{FiraSans}
% ****************************************************************************************************

% ***********************************************************************
% Additional WD41 packages
% ***********************************************************************
\usepackage{lipsum}
\usepackage{array}
\usepackage{ragged2e}
\usepackage{multirow}
\usepackage{chngcntr}
\usepackage{rotating}
\usepackage{algorithm}
\usepackage{algorithmic}
\usepackage[style=long,nolist,nonumberlist,toc,acronym]{glossaries}
\usepackage[roman]{parnotes}
\usepackage{bm}
% Booktabs -----
\usepackage{booktabs}
\newcommand{\ra}[1]{\renewcommand{\arraystretch}{#1}}
% colortbl, for shading cells -----
\usepackage{colortbl}
% bbind, for checkmark symbol -----
\usepackage{bbding}
% changepage, to stretch booktabs table to the margin -----
\usepackage[strict]{changepage}%
%\renewcommand{\thefootnote}{\fnsymbol{footnote}}
%%%%%%%%%%%%%%%%%%%%%%%%%%%%%%%%%%%%%%%%%%%%%%%%%%%%%%%%%%%%%%%%%%%%%%%%%%%%%%
%
% FIGURE TOOLS 
% (From Bejamin Hopfer's Thesis)
% http://benjaminhopfer.com/2014/04/16/typesetting-my-masters-thesis-in-latex/
%
%%%%%%%%%%%%%%%%%%%%%%%%%%%%%%%%%%%%%%%%%%%%%%%%%%%%%%%%%%%%%%%%%%%%%%%%%%%%%%
%%helper functions, no direct use
\usepackage{ifoddpage}

\newlength\fullmarginwidth
\fullmarginwidth=\marginparwidth
\advance\fullmarginwidth by \marginparsep

\newlength\fullwidth
\fullwidth=\textwidth
\advance\fullwidth by \fullmarginwidth


%                   #1   #2       #3     #4   #5
%\subFloatCorrCaps{Cap}{ShortCap}{Label}{Opt}{path}
\newcommand{\subFloatCorrCaps}[5]{
	\ifthenelse{\equal{#1}{}}%
  {
    %subfloat without captions
    \subfloat{%
      \expandafter\includegraphics\expandafter[#4]{#5}
      \label{#3}
    }
  }
  {
    \ifthenelse{\equal{#2}{}}%
    {
      %subfloat with same captions
      \subfloat[#1][#1]{%
        \expandafter\includegraphics\expandafter[#4]{#5}
        \label{#3}
      }
    }
    {
      %subfloat with different captions
      \subfloat[#2][#1]{%
        \expandafter\includegraphics\expandafter[#4]{#5}
        \label{#3}
      }
    }
  }
}

%Parameters: [keyvals (see below)]{file}{caption}
\makeatletter
  \define@key[FigTools]{OneFig}{pos}{\def\FtPos{#1}}
  \define@key[FigTools]{OneFig}{opt}{\def\FtOpt{#1}}
  \define@key[FigTools]{OneFig}{label}{\def\FtLabel{#1}}
  \define@key[FigTools]{OneFig}{shortcaption}{\def\FtShortCaption{#1}}
  
  \define@key[FigTools]{TwoFig}{pos}{\def\FtPos{#1}}
  \define@key[FigTools]{TwoFig}{mainlabel}{\def\FtMainLabel{#1}}
  \define@key[FigTools]{TwoFig}{maincaption}{\def\FtMainCaption{#1}}
  \define@key[FigTools]{TwoFig}{mainshortcaption}{\def\FtMainShortCaption{#1}}
  \define@key[FigTools]{TwoFig}{leftopt}{\def\FtLeftOpt{#1}}
  \define@key[FigTools]{TwoFig}{leftlabel}{\def\FtLeftLabel{#1}}
  \define@key[FigTools]{TwoFig}{leftcaption}{\def\FtLeftCaption{#1}}
  \define@key[FigTools]{TwoFig}{leftshortcaption}{\def\FtLeftShortCaption{#1}}
  \define@key[FigTools]{TwoFig}{rightopt}{\def\FtRightOpt{#1}}
  \define@key[FigTools]{TwoFig}{rightlabel}{\def\FtRightLabel{#1}}
  \define@key[FigTools]{TwoFig}{rightcaption}{\def\FtRightCaption{#1}}  
  \define@key[FigTools]{TwoFig}{rightshortcaption}{\def\FtRightShortCaption{#1}}    
  \define@key[FigTools]{TwoFig}{spacing}{\def\FtSpacing{#1}}    
  \define@key[FigTools]{TwoFig}{topspace}{\def\FtTopSpace{#1}}    
    

  \define@key[FigTools]{ThreeFig}{pos}{\def\FtPos{#1}}
  \define@key[FigTools]{ThreeFig}{mainlabel}{\def\FtMainLabel{#1}}
  \define@key[FigTools]{ThreeFig}{maincaption}{\def\FtMainCaption{#1}}
  \define@key[FigTools]{ThreeFig}{mainshortcaption}{\def\FtMainShortCaption{#1}}
  \define@key[FigTools]{ThreeFig}{leftopt}{\def\FtLeftOpt{#1}}
  \define@key[FigTools]{ThreeFig}{leftlabel}{\def\FtLeftLabel{#1}}
  \define@key[FigTools]{ThreeFig}{leftcaption}{\def\FtLeftCaption{#1}}
  \define@key[FigTools]{ThreeFig}{leftshortcaption}{\def\FtLeftShortCaption{#1}}
  \define@key[FigTools]{ThreeFig}{midopt}{\def\FtMidOpt{#1}}
  \define@key[FigTools]{ThreeFig}{midlabel}{\def\FtMidLabel{#1}}
  \define@key[FigTools]{ThreeFig}{midcaption}{\def\FtMidCaption{#1}}
  \define@key[FigTools]{ThreeFig}{midshortcaption}{\def\FtMidShortCaption{#1}}
  \define@key[FigTools]{ThreeFig}{rightopt}{\def\FtRightOpt{#1}}
  \define@key[FigTools]{ThreeFig}{rightlabel}{\def\FtRightLabel{#1}}
  \define@key[FigTools]{ThreeFig}{rightcaption}{\def\FtRightCaption{#1}}  
  \define@key[FigTools]{ThreeFig}{rightshortcaption}{\def\FtRightShortCaption{#1}}
  \define@key[FigTools]{ThreeFig}{spacing}{\def\FtSpacing{#1}}
  \define@key[FigTools]{ThreeFig}{spacingtwo}{\def\FtSpacingTwo{#1}}
\makeatother

\newcommand{\bigplot}[3][]{%
  \bigfloatinternal[#1]{#2}{#3}{plot}
}

\newcommand{\bigfigure}[3][]{%
  \bigfloatinternal[#1]{#2}{#3}{graph}
}

\newcommand{\bigfloatinternal}[4][]{%
  \begingroup
    \setkeys[FigTools]{OneFig}{ pos=!htp,
                                label={fig:#2},
                                shortcaption={#3},
                                opt={}%width=0.95\textwidth}
                              }
    \setkeys[FigTools]{OneFig}{#1}
    \def\efigure{\begin{figure}}%
    \expandafter\efigure\expandafter[\FtPos]
      \checkoddpage
      \edef\side{\ifoddpage l\else r\fi}
      \makebox[\textwidth][\side]{% 
        \begin{minipage}[t]{\fullwidth}
          \centering
          \ifthenelse{\equal{#4}{plot}}
      		{
          	\input{\evalDir{#2}}
      		}{
          	\expandafter\includegraphics\expandafter[\FtOpt]{#2}
          }
          \caption[\FtShortCaption]{#3}
          \label{\FtLabel}
        \end{minipage}
      }%
    \end{figure}
  \endgroup
}

\newcommand{\bigdoublefigure}[3][]{%
  \begingroup
    %Default values:
    \setkeys[FigTools]{TwoFig}{ pos=!htp,
                                mainlabel={fig:#2-#3},
                                maincaption={#2-#3},
                                mainshortcaption={},%
                                leftopt={},%0.45\textwidth,
                                leftlabel={fig:#2-left},
                                leftcaption={},
                                leftshortcaption={},%
                                rightopt={},%0.45\textwidth,
                                rightlabel={fig:#3-right},
                                rightcaption={},
                                rightshortcaption={},
                                spacing={\hfill},
                                topspace={}
                              }
    %User provided values:
    \setkeys[FigTools]{TwoFig}{#1}
    \def\efigure{\begin{figure}}%
    \expandafter\efigure\expandafter[\FtPos]
      \FtTopSpace
      %Check on which side whe are (right or left)
      \checkoddpage
      \edef\side{\ifoddpage l\else r\fi}
      %Ensure there will be no overfull box message
      \makebox[\textwidth][\side]{% 
        \begin{minipage}{\fullwidth}
          \centering
					\subFloatCorrCaps{\FtLeftCaption}
			                     {\FtLeftShortCaption}
			                     {\FtLeftLabel}
			                     {\FtLeftOpt}
			                     {#2}
			    \FtSpacing
					\subFloatCorrCaps{\FtRightCaption}
			                     {\FtRightShortCaption}
			                     {\FtRightLabel}
			                     {\FtRightOpt}
			                     {#3}
          \caption[\FtMainShortCaption]{\FtMainCaption}
          \label{\FtMainLabel}
        \end{minipage}
      }
    \end{figure}
  \endgroup
}

\newcommand{\bigtriplefigure}[4][]{%
  \begingroup
    %Default values:
    \setkeys[FigTools]{ThreeFig}{ pos=!htp,
	                                mainlabel={fig:#2-#3},
	                                maincaption={#2-#3},
	                                mainshortcaption={},%
	                                leftopt={},%0.45\textwidth,
	                                leftlabel={fig:#2-left},
	                                leftcaption={},
	                                leftshortcaption={},%
	                                midopt={},%0.45\textwidth,
	                                midlabel={fig:#3-mid},
	                                midcaption={},
	                                midshortcaption={},%
	                                rightopt={},%0.45\textwidth,
	                                rightlabel={fig:#4-right},
	                                rightcaption={},
	                                rightshortcaption={},
	                                spacing={\hfill},
	                                spacingtwo={}
	                              }
    %User provided values:
    \setkeys[FigTools]{ThreeFig}{#1}
    \def\efigure{\begin{figure}}%
    \expandafter\efigure\expandafter[\FtPos]
      %Check on which side whe are (right or left)
      \checkoddpage
      \edef\side{\ifoddpage l\else r\fi}
      %Ensure there will be no overfull box message
      \makebox[\textwidth][\side]{% 
        \begin{minipage}{\fullwidth}
          \centering
					\subFloatCorrCaps{\FtLeftCaption}
			                     {\FtLeftShortCaption}
			                     {\FtLeftLabel}
			                     {\FtLeftOpt}
			                     {#2}
			    \FtSpacing
					\subFloatCorrCaps{\FtMidCaption}
			                     {\FtMidShortCaption}
			                     {\FtMidLabel}
			                     {\FtMidOpt}
			                     {#3}
			    \FtSpacing
			    \FtSpacingTwo
					\subFloatCorrCaps{\FtRightCaption}
			                     {\FtRightShortCaption}
			                     {\FtRightLabel}
			                     {\FtRightOpt}
			                     {#4}
          \caption[\FtMainShortCaption]{\FtMainCaption}
          \label{\FtMainLabel}
        \end{minipage}
      }
    \end{figure}
  \endgroup
}

\newcommand{\normplot}[3][]{%
  \normfloatinternal[#1]{#2}{#3}{plot}
}

\newcommand{\normfigure}[3][]{%
  \normfloatinternal[#1]{#2}{#3}{graph}
}

\newcommand{\normfloatinternal}[4][]{%
  \begingroup
    \setkeys[FigTools]{OneFig}{ pos=!htp,
                                label={fig:#2},
                                shortcaption={#3},
                                opt={}%{width=0.95\textwidth}
                              }
    \setkeys[FigTools]{OneFig}{#1}
    \def\efigure{\begin{figure}}%
    \expandafter\efigure\expandafter[\FtPos]
      \centering
      \ifthenelse{\equal{#4}{plot}}
      {
        \input{\evalDir{#2}}
      }{
      	\expandafter\includegraphics\expandafter[\FtOpt]{#2}
      }
      \caption[\FtShortCaption]{#3}
      \label{\FtLabel}
    \end{figure}
  \endgroup
}

\newcommand{\normdoublefigure}[3][]{%
  \begingroup
    %Default values:
    \setkeys[FigTools]{TwoFig}{ pos=!htp,
                                mainlabel={fig:#2-#3},
                                maincaption={},
                                mainshortcaption={},%
                                leftopt={},%width=0.45\textwidth},
                                leftlabel={fig:#2-left},
                                leftcaption={},
                                leftshortcaption={},%
                                rightopt={},%width=0.45\textwidth},
                                rightlabel={fig:#3-right},
                                rightcaption={},
                                rightshortcaption={},
                                spacing={\hfill}
                              }
    %User provided values:
    \setkeys[FigTools]{TwoFig}{#1}
    \def\efigure{\begin{figure}}%
    \expandafter\efigure\expandafter[\FtPos]
      \centering
			\subFloatCorrCaps{\FtLeftCaption}
	                     {\FtLeftShortCaption}
	                     {\FtLeftLabel}
	                     {\FtLeftOpt}
	                     {#2}
			\FtSpacing
			\subFloatCorrCaps{\FtRightCaption}
	                     {\FtRightShortCaption}
	                     {\FtRightLabel}
	                     {\FtRightOpt}
	                     {#3}
      \caption[\FtMainShortCaption]{\FtMainCaption}
      \label{\FtMainLabel}
    \end{figure}
  \endgroup
}

\newcommand{\normtriplefigure}[4][]{%
  \begingroup
    %Default values:
    \setkeys[FigTools]{ThreeFig}{ pos=!htp,
	                                mainlabel={fig:#2-#3},
	                                maincaption={},
	                                mainshortcaption={},%
	                                leftopt={},%width=0.45\textwidth},
	                                leftlabel={fig:#2-left},
	                                leftcaption={},
	                                leftshortcaption={},%
	                                midopt={},%width=0.45\textwidth},
	                                midlabel={fig:#3-mid},
	                                midcaption={},
	                                midshortcaption={},%
	                                rightopt={},%width=0.45\textwidth},
	                                rightlabel={fig:#4-right},
	                                rightcaption={},
	                                rightshortcaption={},
	                                spacing={\hfill},
	                                spacingtwo={}
	                              }
    %User provided values:
    \setkeys[FigTools]{ThreeFig}{#1}
    \def\efigure{\begin{figure}}%
    \expandafter\efigure\expandafter[\FtPos]
      \centering
			\subFloatCorrCaps{\FtLeftCaption}
	                     {\FtLeftShortCaption}
	                     {\FtLeftLabel}
	                     {\FtLeftOpt}
	                     {#2}
			\FtSpacing
			\subFloatCorrCaps{\FtMidCaption}
	                     {\FtMidShortCaption}
	                     {\FtMidLabel}
	                     {\FtMidOpt}
	                     {#3}
			\FtSpacing
			\FtSpacingTwo
			\subFloatCorrCaps{\FtRightCaption}
	                     {\FtRightShortCaption}
	                     {\FtRightLabel}
	                     {\FtRightOpt}
	                     {#4}
      \caption[\FtMainShortCaption]{\FtMainCaption}
      \label{\FtMainLabel}
    \end{figure}
  \endgroup
}