\newpage
%*****************************************************************************************************************
\section[\texttt{trace-simexp}]{\texttt{trace-simexp}: Computer Experiment for TRACE code}\label{app:trace_simexp}
%*****************************************************************************************************************

% Introductory paragraph
A computer experiment, loosely defined, is a multiple computer model runs using different values of the model parameters.
Its design, in particular the selection of the design points at which the model will be evaluated; as well as its analysis, in particular the analysis of the output variation in relation to the inputs variation, are useful for sensitivity and uncertainty analyses of the model subjected to the experimentation.

An important prerequisite of carrying out such experiment is the availability of a supporting tool able to handle the related logistical aspects.
A Python3-based scripting utility has been developed to assist in carrying such experiments for the thermal-hydraulics system code \gls[hyper=false]{trace}.
The scope of the utility is ranging from the pre-processing of the \gls[hyper=false]{trace} input deck amenable for batch parallel execution to the post-treatment of the resulting binary \texttt{xtv} / \texttt{dmx} file amenable to subsequent sensitivity and uncertainty analyses.

The general flow chart of the processes involved in \texttt{trace-simexp} package is Fig.~\ref{fig:ch1_trace_simexp}.
\bigfigure[pos=tbhp,
           opt={width=1.0\textwidth},
           label={fig:ch1_trace_simexp},
           shortcaption={Flowchart of \texttt{trace-simexp}.}]
{../figures/appendices/trace-simexp/trace_simexp}
{Flowchart of \texttt{trace-simexp}.}