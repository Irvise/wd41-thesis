%*****************************************************************************************
\section{Landmark Registration and Time Warping Function}\label{app:landmark_registration}
%*****************************************************************************************

The most straightforward curve registration procedure is the landmark registration/marker registration \cite{Ramsay2009}.
\emph{Landmarks} are salient features of a curve that can be observed or expected to occur in a set of curves.
In the context of reflood simulations, examples of such landmarks are the time of maximum temperature and the time of quenching.
A transformation of time for each curve is carried out such that these features are aligned with respect to a reference curve.

The landmark registration problem can be expressed as the following: 
Let $\{y_i(t); i = 1, 2, \cdots, N; t \in [t_a,t_b]\}$ be a set of continuous functions defined over the domain $t_a$ to $t_b$.
Let $\{y_{ref}(t^{ref}_j); j = 1, \cdots, M\}$ be a set of $M$ landmarks of a given reference function $y_{ref}(t)$.
Then a set of time warping functions $\{h_i(t); i = 1, 2, \cdots, N; t \in [t_a, t_b]\}$ for each curve in the data set can be defined.
These functions have the following properties:

\begin{enumerate}
	
	\item Each $h_i(t)$ is defined in the same domain as the domain of the original curve $y_i(t)$. That is, $t \in [t_a, t_b]$;
  
	\item Each $h_i(t)$ satisfies the boundary conditions,
    \begin{equation}
      \begin{split}
        h_i(t_a) & = t_a \\
        h_i(t_b) & = t_b \text{;}
      \end{split}
    \label{eq:warping_boundaries}
    \end{equation}
   
	\item Each $h_i(t)$ is a strictly increasing function. The first implication of this property is that the time transformation process cannot alter the ordering of the landmarks.
         In other words, time is strictly increasing both in the original and the transformed frames.
         The second implication is that the time warping function is an invertible function 
         such that for the same event there exists a unique pair of time and its transformed value.
  
	\item Each $h_i(t)$ transforms the reference time for each curve such that the timing of the $M$ reference landmarks are aligned with respect to the landmarks of each curve,
          \begin{equation}
            \begin{split}
              y_i [h_i(t)]				& \equiv y_i(t^*_i) \equiv y^*_i(t) ; \, \forall i = 1, \cdots, N  \\
              h_i (t^{ref}_j) & \equiv t^*_{ij} \iff h_i^{-1}[t^*_{ij}] = t^{ref}_j  ; \, \forall j = 1, \cdots, M 
            \end{split}
          \label{eq:warping_transformation}
          \end{equation}
          where $t_i^*$ and $y_i^*$ are the warped time and the \emph{registered} function of curve $i$, respectively;
					$t_{ij}^*$ is the timing of the landmark $j$ of curve $i$;
					and $h^{-1}_i(t)$, the inverse of the warping function, is the \emph{aligning} function. 
					The argument $t$ appeared in $y_i^*(t)$ implies that the registered curve is in the time scale of the reference, which is the same for all the curves.
          For instance, if $h_i(t^{ref}_j) > t^{ref}_j$ then the landmark $j$ for curve $i$ is delayed 
          and the aligning function accelerates the time for curve $i$ to conform to the reference timing.
          On the other hand, if $h_i(t^{ref}_j) < t^{ref}_j$ then the landmark $j$ for curve $i$ occurs earlier
          and the aligning function retards the time for curve $i$ to conform to the reference timing.
\end{enumerate}

The registration problem can then be posed as an estimation problem of each time warping function $h_i(t)$ constrained by the above properties.
Following \cite{Ramsay1998}, it is solved by using penalized least square regression method.
In accordance to the \glsfirst[hyper=false]{fda} framework, the warping function is also represented as a linear combination of B-spline basis functions.