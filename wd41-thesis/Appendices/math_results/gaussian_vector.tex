%------------------------------------------------------------------------------------------------
\section{Gaussian Random Vector (Multivariate Normal Random Variable)}\label{app:gaussian_vector}
%------------------------------------------------------------------------------------------------

% Introductory Paragraph, Why Gaussian
Gaussian random vector is a vector with random elements that are \emph{jointly} Gaussian.
That is, the random variables have an \gls[hyper=false]{mvn} distribution. 
It is the most widely studied and applied multivariate random variable.
There are a couple of reasons for this.
From a practical viewpoint, the distribution of MVN is tractable and its special properties are well known \cite{Jensen2014}.
From an epistemological viewpoint, modeling a quif only the mean and variance are what's to be then MVN distribution is fully characterized by its mean and covariance, only these wo parameters are of interest \cite{McElreath2015}.
This section of the appendices reviews the definition and some of the most important properties of MVN random variable relevant in the present study.

A $D$-dimensional random vector $\bm{\mathcal{X}}$ whose elements are random variables, $\bm{\mathcal{X}} = [\mathcal{X}_1, \cdots, \mathcal{X}_D] \in \mathbb{R}^D$, 
is said to have an \gls[hyper=false]{mvn} distribution with mean vector $\boldsymbol{\mu} \in \mathbb{R}^D$ and variance-covariance matrix $\boldsymbol{\Sigma} \in S_{++}^D$,
\marginpar{multivariate normal distribution}
if its joint probability density function is given by,
\begin{equation}
	p(\mathbf{x};\boldsymbol{\mu},\boldsymbol{\Sigma}) = \frac{1}{2\pi^{D/2}|\boldsymbol{\Sigma}|^{1/2}} \exp{\left[-\frac{1}{2}(\boldsymbol{x}-\boldsymbol{\mu})^T\boldsymbol{\Sigma}^{-1}(\boldsymbol{x}-\boldsymbol{\mu})\right]}
	\label{eq:gaussian_joint_density}
\end{equation}
The joint distribution of a Gaussian random vector is parameterized and fully specified by the mean vector $\boldsymbol{\mu}$ and the variance-covariance matrix $\boldsymbol{\Sigma}$. 
The symbol ``;'' separates the value of the variates $\mathbf{x}$ from the parameters of the distribution. 
A $D$-dimensional random vector $\bm{\mathcal{X}}$ distributed as a joint Gaussian is denoted by,
\begin{equation}
	\bm{\mathcal{X}} \sim \mathcal{N}_D\left(\boldsymbol{\mu}, \boldsymbol{\Sigma}\right)
\end{equation}

% Mean vector
The mean vector $\boldsymbol{\mu}$ is defined as,
\marginpar{mean vector}
\begin{equation}
	\boldsymbol{\mu} = [\mathbb{E}[\mathcal{X}_1], \cdots, \mathbb{E}[\mathcal{X}_D]]^T
\label{eq:mean_vector}
\end{equation}
where $\mathbb{E}[\circ]$ is the expectation operator, such that $\mathbb{E}[\mathcal{X}] = \int_x x p(x) dx$.

% Variance-Covariance Matrix
The variance-covariance matrix $\boldsymbol{\Sigma}$ is an element in the space of symmetric positive semi-definite (PSD) $D \times D$ matrices, $S_{++}^D$ defined as,
\begin{equation}
	S_{++}^D = \{\boldsymbol{\Sigma} \in \mathbb{R}^{D\times D}: \boldsymbol{\Sigma} = \boldsymbol{\Sigma}^T, \mathbf{x}^T \boldsymbol{\Sigma} \mathbf{x} \geq 0, \forall \mathbf{x} \in \mathbb{R}^D \textnormal{and } \mathbf{x} \neq 0 \}
	\label{eq:def_covariance_matrix}
\end{equation}

% Diagonal Elements of the Variance-Covariance Matrix
The diagonal elements of the variance-covariance matrix, $\Sigma_{i,i}$, describe the variance of a single random variable,
\marginpar{variance-covariance matrix}
while the off-diagonal elements, $\Sigma_{i,j}$, describe the covariation between a pair of random variables,
\begin{equation}
	\boldsymbol{\Sigma} =
	\begin{pmatrix}
			\mathbb{V}[\mathcal{X}_1] 							 & \cdots		& \text{Cov}[\mathcal{X}_1, \mathcal{X}_D] \\
			\vdots                   								 & \ddots   & \vdots \\
			\text{Cov}[\mathcal{X}_D, \mathcal{X}_1] & \cdots 	& \mathbb{V}[\mathcal{X}_D]
	\end{pmatrix}
\label{eq:covariance_matrix}
\end{equation}
where $\mathbb{V} [\circ]$ is the variance operator, such that $\mathbb{V} [\mathcal{X}] = \mathbb{E}[(\mathcal{X} - \mathbb{E}[\mathcal{X}])^2]$;
and $\text{Cov} [\circ, \circ]$ is the covariance operators, such that $\text{Cov} [\mathcal{X}, \mathcal{X}^*] = \mathbb{E}[(\mathcal{X}-\mathbb{E}[\mathcal{X}])(\mathcal{X}^*-\mathbb{E}[\mathcal{X}^*])]$.

% Partitioning Gaussian Random Vector
Suppose that the $D$-dimensional random vector $\bm{\mathcal{X}}$ is partitioned into two sub-vectors (disjoint sets) of $D_1$-dimensional random vector $\bm{\mathcal{X}}_A$ and $D_2$-dimensional random vector $\bm{\mathcal{X}}_B$, such that $\bm{\mathcal{X}} = [\bm{\mathcal{X}}_A, \bm{\mathcal{X}}_B]$ and $D = D_1 + D_2$ (see Appendix~\ref{app:probability}).
\marginpar{Gaussian random vector partition}
Then the Gaussian random vector $[\bm{\mathcal{X}}_A, \bm{\mathcal{X}}_B]$ is written,
\begin{equation}
	\begin{bmatrix}
		 \bm{\mathcal{X}}_A \\
		 \bm{\mathcal{X}}_B
	\end{bmatrix} \sim \mathcal{N} \left (
	\begin{bmatrix}
			\boldsymbol{\mu}_A \\
			\boldsymbol{\mu}_B
	\end{bmatrix}, \begin{pmatrix}
		  \boldsymbol{\Sigma}_{A,A}   & \boldsymbol{\Sigma}_{A,B} \\
			\boldsymbol{\Sigma}_{B,A} & \boldsymbol{\Sigma}_{B,B} \\
	\end{pmatrix} \right)
\label{eq:gaussian_random_vector}
\end{equation}
where $\boldsymbol{\mu}_A$ and $\boldsymbol{\mu}_B$ are the $D_1$-dimensional and $D_2$-dimensional mean vectors of $\bm{\mathcal{X}}_A$ and $\bm{\mathcal{X}}_B$, respectively;
and $\boldsymbol{\Sigma}_{A,A}$, $\boldsymbol{\Sigma}_{A,B}$, $\boldsymbol{\Sigma}_{B,A}$, $\boldsymbol{\Sigma}_{B,B}$ are the $D_1 \times D_1$, $D_1 \times D_2$, $D_2 \times D_1$, and $D_2 \times D_2$ sub-matrices of the partitioned covariance matrix, respectively. 
So for instance, 
\begin{equation*}
	\boldsymbol{\Sigma}_{A,B} =
	\begin{pmatrix}
			\text{Cov}[\mathcal{X}_1, \mathcal{X}_{D_1+1}] 			& \cdots	& \text{Cov}[\mathcal{X}_1, \mathcal{X}_{D}] \\
			\vdots                   								 			 			& \ddots  & \vdots \\
			\text{Cov}[\mathcal{X}_{D_1}, \mathcal{X}_{D_1+1}]  & \cdots 	& \text{Cov}[\mathcal{X}_{D_1}, \mathcal{X}_{D}]
	\end{pmatrix}
\label{eq:covariance_matrix}
\end{equation*}

% Marginal Distribution of Gaussian Random Vector
The marginal density of both $\bm{\mathcal{X}}_A$ and $\bm{\mathcal{X}}_B$ follows an \gls[hyper=false]{mvn} distribution given by,
\marginpar{Gaussian identity: marginal density}
\begin{equation}
		p(\mathbf{x}_A) = \frac{1}{(2\pi)^{D_1/2}|\boldsymbol{\Sigma_{A,A}}|^{1/2}} \exp{\left[-\frac{1}{2}(\mathbf{x}_A-\boldsymbol{\mu}_A)^T\boldsymbol{\Sigma^{-1}_{A,A}}(\mathbf{x}_A-\boldsymbol{\mu}_B)\right]}
\label{eq:gaussian_marginal}
\end{equation}
The results are analogous for the marginal of $\bm{\mathcal{X}}_B$, $p(\mathbf{x}_B)$ of which all the subscripts $A$ is replaced by $B$.

% Conditional Distribution of Gaussian Random Vector
The conditional density of $\bm{\mathcal{X}}_A$ conditioned on $\bm{\mathcal{X}}_B$, again, also follows an \gls[hyper=false]{mvn} distribution given as,
\marginpar{Gaussian identity: conditional density}
\begin{equation}
	\begin{split}
		& p(\mathbf{x}_A|\mathbf{x}_B) = \frac{1}{2\pi^{D_1/2}|\boldsymbol{\Sigma_{A,A}}|^{1/2}} \exp{\left[-\frac{1}{2}(\mathbf{x}_A-\boldsymbol{\mu}_A)^T\boldsymbol{\Sigma^{-1}_{A,A}}(\mathbf{x}_A-\boldsymbol{\mu}_B)\right]} \\
		& \boldsymbol{\mu}_A = \boldsymbol{\mu}_A + \boldsymbol{\Sigma}_{A,B} \boldsymbol{\Sigma}^{-1}_{B,B} \left(\mathbf{x}_B - \boldsymbol{\mu}_B \right) \\
		& \boldsymbol{\Sigma}^*_{A,A} = \boldsymbol{\Sigma}_{A,A} - \boldsymbol{\Sigma}_{A,B} \boldsymbol{\Sigma}^{-1}_{B,B} \boldsymbol{\Sigma}^T_{A,B} 
	\end{split}
\label{eq:gaussian_conditional}
\end{equation}
The results are analogous for the conditional $\bm{\mathcal{X}}_B$ given $\bm{\mathcal{X}}_A$, $p(\mathbf{x}_B|\mathbf{x}_A)$ of which all the subscripts $A$ is replaced by $B$, and vice versa.
