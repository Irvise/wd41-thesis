\newpage
%***********************************************************************
\section{Convergence of the Sobol' Indices}\label{app:sobol_convergence}
%***********************************************************************

% TRACE plot of total effect indices estimation
The convergence of the Sobol' indices' estimator can be investigated from their evolutions as functions of the number of samples.
Shown in Fig.~\ref{fig:ch3_plot_running_indices} is the evolution (trace plot) of the estimated main-effect indices with the maximum cladding temperature as the \gls[hyper=false]{qoi}.
The Saltelli et al. estimator performs poorly compared to the Janon et al. estimator for this particular output.
This means that a larger number of samples are required to obtain a stable ranking.
\normfigure[pos=tbhp,
           opt={width=1.0\textwidth},
           label={fig:ch3_plot_running_indices},
           shortcaption={Trace plot of the main-effect sensitivity indices estimations.}]
{../figures/chapter3/figures/plotRunningIndices}
{Evolution of the main-effect sensitivity indices for all input paramters using two different estimators as a function of the number of Sobol' sequence samples. The \gls[hyper=false]{qoi} is the maximum cladding temperature}

% Uncertainty in the estimation
If the main purpose of the \gls[hyper=false]{sa} is simply to rank the parameter importance with respect to a particular \gls[hyper=false]{qoi},
then the Janon et al. estimator visibly requires fewer samples and the ranking can be reliably constructed.
However, the apparent stabilization of the indices' estimator is not sufficient to establish a robust estimate of the indices since \gls[hyper=false]{mc} estimation entails uncertainty due to finite number of samples.
Such uncertainty needs to be addressed for all the estimates.

% Empirical convergence study
In this work, an empirical convergence study was established using three different sample sizes ($250$; $500$; $1000$),
and for each size, the $95\%$ \gls[hyper=false]{ci} length (the difference between the upper and lower bounds) is determined using the bootstrap technique \cite{Efron1986} using $10'000$ replications.
The results are shown if Fig.~\ref{fig:ch3_plot_convergence_maxtemp} for the maximum cladding temperature as the \gls[hyper=false]{qoi}.
Note that the comparisons between \gls[hyper=false]{ci} lengths of different estimates can be made directly as the Sobol' index itself is dimensionless.
As can be seen, with respect to this \gls[hyper=false]{qoi}, the Janon et al. estimator is further confirmed as the more efficient estimator.
The uncertainty of indices estimated by the Saltelli et al. estimator is still high for numbers of samples in the range of thousands.
The efficiency of the Saltelli et al. estimator is also found to be more sensitive to the choice of estimand (i.e., Sobol' index of a given input parameter). 
\normfigure[pos=tbhp,
	         opt={width=1.0\textwidth},
           label={fig:ch3_plot_convergence_maxtemp},
           shortcaption={Convergence of the Sobol' main effect indices estimators w.r.t maximum cladding temperature.}]
{../figures/chapter3/figures/plotConvergenceMaxTemp}
{The $95\%$ percentile boostrapped CI length as a function of the number of samples for five selected estimated Sobol' main-effect indices, with respect to the maximum cladding temperature using two different estimators. The lines shown are the regression through the origin lines.}

% Difference in output results in different convergence
However, further investigation also revealed that the efficiency of an estimator depended on the \gls[hyper=false]{qoi} in a more complex manner than initially considered.
As can be seen in Fig.~\ref{fig:ch3_plot_convergence_pc1score}, where the first principal component scores were taken as the \gls[hyper=false]{qoi}, both estimators were found to be comparable,
with the Saltelli et al. estimator being even slightly more efficient.
And as before, the Janon et al. estimator shows less sensitivity to the choice of estimand in its convergence as compared to the Saltelli et al. estimator.
\normfigure[pos=!tbhp,
					 opt={width=1.0\textwidth},
           label={fig:ch3_plot_convergence_pc1score},
           shortcaption={Convergence of the Sobol' main effect indices estimators w.r.t first principal component.}]
{../figures/chapter3/figures/plotConvergencePC1Score}
{The $95\%$ percentile boostrapped CI length as a function of the number of samples for five selected estimated Sobol' main-effect indices, with respect to the first principal component using two different estimators. The lines shown are the regression through the origin lines.}

% The use of convergence study
The convergence analysis plots shown in Figs.~\ref{fig:ch3_plot_convergence_maxtemp} and \ref{fig:ch3_plot_convergence_pc1score} can be useful in the planning of the simulation experiments.
As can be inferred from both figures, the \gls[hyper=false]{ci} length of a given estimator depends on the \gls[hyper=false]{qoi}, the estimand, the estimator used, and the number of samples.
The regression lines also indicate the projection of the reduction in the \gls[hyper=false]{ci} length with increasing number of samples.

% Total-effect Indices convergence
As for the total-effect indices, the results obtained using the Jansen's estimator confirmed the good efficiency of the estimator,
reaching below $10\%$ \gls[hyper=false]{ci} length for $1'000$ \gls[hyper=false]{mc} samples across all \gls[hyper=false]{qoi}.
